% Choose the language of your thesis passing 'french' or 'english' as
% \documentclass option.
% Note1: The 'page de garde' will always be written in French.
% Note2: You will have an error if you change the language of the document and
%        compile it without cleaning the auxiliary files. Compiling it again
%        should solve the problem.
\documentclass[english,a4paper,11pt,twoside]{StyleThese}
\newcommand{\included}{}
\usepackage{pdfpages}

\include{formatAndDefs}
\include{Symbols}
\usepackage[acronym]{glossaries}

\makenoidxglossaries

\newacronym{hri}{HRI}{Human-Robot Interaction}
\newacronym{han}{HAN}{Human-Aware Navigation}
\newacronym{cohan}{CoHAN}{Cooperative Human-Aware Navigation}
\newacronym{hateb}{HATEB}{Human-Aware Timed Elastic Band}
\newacronym{rrt}{RRT}{Rapidly exploring Random Tree}
\newacronym{prm}{PRM}{Probablistic Road Map}
\newacronym{slam}{SLAM}{Simultaneous Localization and Mapping}
\newacronym{ekf}{EKF}{Extended Kalman Filter}
\newacronym{lidar}{LIDAR}{Light Detection and Ranging}
\newacronym{dwa}{DWA}{Dynamic Window Approach}
\newacronym{rvo}{RVO}{Reciprocal Velocity Obstacle}
\newacronym{orca}{ORCA}{Optimal Reciprocal Collision Avoidance}
\newacronym{calu}{CALU}{Collision Avoidance with Localization Uncertainty}
\newacronym{cadrl}{CADRL}{Collision Avoidance with Deep Reinforcement Learning}
\newacronym{sacadrl}{SA-CADRL}{Socially Aware Collision Avoidance with Deep Reinforcement Learning}
\newacronym{crl}{CRL}{Composite Reinforcement Learning}
\newacronym{sfm}{SFM}{Social Force Model}
\newacronym{pomdp}{POMDP}{Partially Observable Markov Decision Process}
\newacronym{irl}{IRL}{Inverse Reinforcement Learning}
\newacronym{frp}{FRP}{Freezing Robot Problem}
\newacronym{mpc}{MPC}{Model Predictive Control}
\newacronym{rl}{RL}{Reinforcement Learning}
\newacronym{hrvo}{HRVO}{Hybrid Reciprocal Velocity Obstacle}
\newacronym{teb}{TEB}{Timed Elastic Band}
\newacronym{ros}{ROS}{Robot Operating System}
\newacronym{ttc}{TTC}{Time-to-Collision}
\newacronym{rgl}{RGL}{Relational Graph Learning}

%%%%%%%%%%%%%%%%%%%%%%%%%%%%%%%%% Previous %%%%%%%%%%%
% \newacronym{hatp}{HATP}{Hierarchical Agent-based Task Planner}
% \newacronym{htn}{HTN}{Hierarchical Task Network}
% \newacronym{het}{HET}{Hierarchical Execution Trace}

% \newacronym{reg}{REG}{Referring Expression Generation}
% \newacronym{re}{RE}{Referring Expression}
% \newacronym{kb}{KB}{Knowledge Base}
% \newacronym{mummer}{MuMMER}{MultiModal Mall Entertainment Robot}

% \newacronym{ce}{CE}{Compound Entity}
% \newacronym{cr}{CR}{Compound Relation}
% \newacronym{ct}{CT}{Compound Tree}

% \newacronym{ucs}{UCS}{Uniform-Cost Seach}
% \newacronym{ia}{IA}{Incremental Algorithm}
% \newacronym{gba}{GBA}{Graph-Based Algorithm}

% \newacronym{ssr}{SSR}{Semantic Spatial Representation}
\setlength{\belowcaptionskip}{-6pt}

%%%%%%%%%%%%%%% Uncomment this to generate new Cover page%%%%%%%%%
\usepackage[ED=MITT-InfoTel, Ets=INP]{tlsflyleaf}
%% This is file `example.tex',
%% Copyright 2013 Tristan GREGOIRE
%% Copyright 2015 Yann BACHY
%
% This work may be distributed and/or modified under the
% conditions of the LaTeX Project Public License, either version 1.3
% of this license or (at your option) any later version.
% The latest version of this license is in
%   http://www.latex-project.org/lppl.txt
% and version 1.3 or later is part of all distributions of LaTeX
% version 2005/12/01 or later.
%
%
% This work has the LPPL maintenance status `maintained'.
% 
% The Current Maintainer of this work is T. GREGOIRE
%

%\documentclass{book}

% Loading the tlsflyleaf.sty package require some option to define the
% establishment name, the doctoral school and the PhD speciality.
% In that aim you have 2 key-value option:
%   - Ets=<value> : define the establishment name
%   - ED=<value>  : define the doctoral school and speciality
%   - ED2=<value> : define the second speciality ("double mention"). OPTIONAL.
% The full list of accepted values for each option could be find either
% in the documentation or in ED-list.txt and Ets-list.txt files provide with the package.
%\usepackage[ED=MITT - STICRT, Ets=INSA]{tlsflyleaf}
%\usepackage[ED=SDU2E-Ast, ED2=SDU2E-Eco, Ets=UT3]{tlsflyleaf}
%\usepackage[ED=MITT - STICRT, Ets=UT3]{tlsflyleaf}

% ==================
% Setup basic string
% - PhD Title
% - author
% - defence date
% - laboratory
% - cotutelle
\title{\textbf{\large Human-Aware Robot Task Planning:\\Theory of Mind and Anticipation of Human Decisions and Actions}}
% \title{\textbf{\large Planification de Tâches pour un Robot Collaboratif :\\Théorie de l'Esprit et Anticipation des Décisions et Actions de l'Humain}}
\author{Anthony Favier}
\defencedate{30/04/2024}
\lab{Laboratoire d'Analyse et d'Architecture des Systèmes (LAAS-CNRS)}
%\cotutelle{}

% ==================
% Setup people like your boss, the jury team and the referees
% - First you need to define how number they will be in each category
%   It is done with the commands \nboss{n}, \nreferee{n} and \njudge{n}.
%   You can define more people in each category than the number given 
%   but only the first "\npeople" will be print.
% - Then use the command \makesomeone{<category>}{<number>}{<name>}{<status>}{<other>}
%   where:
%     <category> should be select in ['boss', 'referee', 'judge']
%     <number>   is the rank for printing the person. 
%                Only number <= "\npeople" will be printed
%     <name>     First name and las name of the people
%     <status>   Is (s)he a "charg\'e de recher" ou un "professeur d'universit\'e"...
%     <other>    What ever string you want to add (laboratory, jury member place...).
%% Boss
\nboss{1}
\makesomeone{boss}{1}{Rachid ALAMI}{}{} % Sera afiche en premier
%% Referee
\nreferee{2}
\makesomeone{referee}{1}{Mohamed CHETOUANI}{}{}
\makesomeone{referee}{2}{Julie SHAH}{}{}
%% Jury
\njudge{7}
\makesomeone{judge}{1}{Mohamed CHETOUANI}{Professeur}{Rapporteur}
\makesomeone{judge}{2}{Julie SHAH}{Professeure}{Rapporteur}
\makesomeone{judge}{3}{Jean-Charles FABRE}{Professeur}{Membre du Jury}
\makesomeone{judge}{4}{Luca IOCCHI}{Professeur}{Membre du Jury}
\makesomeone{judge}{5}{Simon LACROIX}{Directeur de Recherche}{Membre du Jury}
\makesomeone{judge}{6}{Catherine PELACHAUD}{Directrice de Recherche}{Membre du Jury}
\makesomeone{judge}{7}{Rachid ALAMI}{Directeur de Recherche}{Directeur de Thèse}

% ============================================================
% DOCUMENT
% \begin{document}
%   \makeflyleaf
% \end{document}

%%%%%%%%%%%%%%% Uncomment this to generate new Cover page%%%%%%%%%

\sloppy
\begin{document}

%%%%%%%%%%%%%%% Uncomment this to generate new Cover page %%%%%%%%%
\makeflyleaf
%%%%%%%%%%%%%%% Uncomment this to generate new Cover page %%%%%%%%%

%%%%%%%%%%%%%%% Uncomment this to use custom Cover page %%%%%%%%%
% \includepdf[pages=-]{custom_cover_new.pdf}
%%%%%%%%%%%%%%% Uncomment this to use custom Cover page %%%%%%%%%


\cleardoublepage

\dominitoc

\pagenumbering{roman}

\cleardoublepage

%%%% Acknowledgments %%%%%
%%% Needs an update before and after the defence, write last
% \section*{Remerciemments}
 

% General thèse, LAAS

% Jean Charles Fabre: 
%     - m'a aidé pour divers pb pendant l'école
%     - M'a proposé un stage qui m'a fait découvrir le LAAS 
%     - Puis m'a finalement recommandé à Rachid Alami qui m'a lui même permis d'obtenir un stage au Japon ainsi qui qu'une thèse au dans l'équipe RIS
%     - Le covid a changé mes plans, le japon est tombé à l'eau, et j'ai fini par faire un stage avec Rachid Alami qui est finalement devenu d'un manière un peu flou le début de ma thèse


Le Docteur qui sort de cette aventure est bien différent de l'étudiant qui l'a commencé. Car oui, une thèse c'est une aventure dans laquelle on s'engage en quête de découverte par curiosité et passion.
Mais on s'engage également sans savoir où cela nous mène, sans savoir ce que l'on va trouver, sans savoir si l'on va dans la bonne direction, et sans savoir si l'on arrivera au bout.
Finalement et soudainement, cette aventure se termine et l'on comprend que ça en valait la peine. 
Cette expérience m'a fait grandir, à la fois sur le plan scientifique, mais aussi et surtout sur le plan personnel.
J'ai énormément appris et j'ai rencontré des personnes formidables.
Ainsi, je souhaite remercier tous ceux qui ont participé de près ou de loin à mon aventure.

Tout d'abord, je souhaite remercier mon directeur de thèse Rachid Alami qui a joué un rôle crucial dans ce parcours.
Merci de m'avoir fait confiance, que ce soit en mes idées, ma technique ou mon écriture.
Nous avons très vite été sur la même longueur d'onde concernant le sujet, ce qui a rendu nos échanges et notre travail assez naturel concernant la direction à prendre.
Ainsi, merci pour ces discussions et nombreuses idées qui elles même en font naître d'autres permettant toujours d'aller plus loin. 
Merci pour tes remarques et critiques, toujours pertinentes, qui permettent d'affiner et toujours mieux faire.
Merci pour ta disponibilité, qu'il soit midi ou minuit, que tu sois au labo ou à l'autre bout du monde, tu trouves toujours un moment pour répondre aux emails de dernière minute.

Je souhaite ensuite remercier tous ceux qui m'ont accueilli et intégré lors de mes débuts. Je pense notamment à Amandine, Antoine, Yannick, Kathleen, Phani, Guilhem, Guillaume.

Un merci particulier à Guilhem pour ton legs d'HATP/EHDA et le temps que tu m'as consacré pour expliquer son fonctionnement. 
Ce système m'a beaucoup inspiré, et sans vraiment le savoir, tu m'as fourni les fondations de la majorité de ma thèse. C'est aussi ça la recherche, avancer pas par pas, construire brique après brique. D'une certaine manière, tu m'as donné la première brique sur laquelle j'ai pu en poser beaucoup d'autres, et donc je t'en remercie.

Merci également à Phani, tu es surement la première personne avec qui j'ai collaboré dans l'équipe. C'est avec toi que j'ai écrit mes premiers articles et c'est grâce à tes conseils et relectures que j'ai pu apprendre à écrire dans ce format. 

Merci aussi à Guillaume. Tu as l'habitude mais je dois le répéter, impressionnant, toujours costume, pas langue dans sa poche. Mais c'est ça qui allié à ta vision juste et critique permet lors de brève discussion de ce rendre compte que l'on avance de travers et que l'on peut mieux faire.   

Guilhem, Amandine, Antoine, Kathleen, Yannick

Phani

Shashank

Guillaume: qui 

Simon

Philippe

Jeremy

Collègues en général, liste

Permanent (Simon Lacroix, Arthur, Felix, Aurélie)



Merci à ANITI d'avoir permis de financer mon doctorat et au LAAS-CNRS d'avoir permis qu'il s'y déroule.


%%%% Abstract %%%%%
% \chapter*{Abstract}

% 4000 caractères

Human-robot collaboration and coordination must be seamless for robots to become significant parts of our everyday lives. 
Despite human-robot collaboration being proven beneficial, most of today's robots work in a physically separated workspace from humans or their abilities are severely restricted in close vicinity of humans.
The goal of Human-Robot Collaboration (HRC), a growing field in robotics and AI research, is to enable humans to collaborate safely and effectively with robots to achieve any industrial, service or household tasks.
This work aims to bridge the gap between robotic capabilities and human expectations, fostering a new era of seamless and intuitive collaboration between humans and robots in shared environments. More precisely, this manuscript presents an exploration of decision-making in the context of Human-Robot Collaboration, especially in the subfields of human-aware navigation and task planning.

First, we discuss various related fields and works of HRC to better understand the context of my work. 
After being familiarized with the HATP/EHDA task planner I present my first contribution which incorporates some Theory of Mind concepts in HRC task planning. Some models and algorithms are proposed and evaluated to better estimate and maintain human beliefs in order to better anticipate their behavior. As a result, we can identify when the human has a false belief about a fact evaluated as relevant for the task. In such cases, the robot can proactively inform the human to correct the false belief or the robot can purposely delay its action to make sure the human sees its execution and infer the corresponding fact, avoiding verbal communication. Results show that this scheme allows to effectively maintain human beliefs in false beliefs task scenarios and allows for solving a broader class of problems than HATP/EHDA while not communicating systematically.

My second contribution is a new task planning approach producing a behavioral robot policy ensuring smooth collaboration where the human always has a full latitude of decision and where the robot always complies concurrently with the human actions.
This approach is based on a model of concurrent and compliant joint action that we designed. This model in the form of an automaton captures the uncontrollability factor of humans and social signals. We also propose a new plan evaluation and selection based on estimations of the human inner preferences regarding the task. Empirical results show that this approach allows a concurrent robot behavior compliant with human online decisions and preferences.

To validate this approach we conducted a user study using an interactive collaboration simulator specifically developed for this. Participants were asked to collaborate in several BlocksWorld scenarios with a simulated robot following the policies produced by our approach. We used a baseline opposite to our approach for comparison where the human is forced to comply with the robot choices. We showed through statistical analysis that our approach allowed satisfying significantly better human preferences. Likewise, we also showed, compared to the baseline, that our approach induces a significantly more positive interaction, adaptive and efficient collaboration, and significantly more adaptive and accommodating robot decisions.

Eventually, my third contribution concerns the decision-making in navigation. I propose a system producing an ``intelligent'' human avatar that on top of being reactive can make rational decisions about navigation tasks. This system serves as a benchmarking and testing tool for robot navigation systems to be challenged. This way, robot navigation systems can be evaluated, tuned and stress tested in simulation allowing them to run mature real-life experiments faster.

% \chapter*{Résumé}

% 4000 char

Bien que la collaboration humain-robot puisse être bénéfique, la plupart des robots actuels travaillent dans des espaces physiquement séparés de l'humain ou alors leurs capacités sont drastiquement limitées à proximité d'un humain. Ce travail vise à combler le fossé entre les capacités robotiques et les attentes humaines, en favorisant une nouvelle ère de collaboration transparente et intuitive entre les humains et les robots dans des environnements partagés pour réaliser à la fois des tâches industrielles, de services ou domestiques. Plus précisément, ce manuscrit présente une étude sur la prise de décision dans le contexte de la collaboration humain-robot, en particulier dans les domaines et de la planification des tâches et de la simulation d'agents intelligents.

D'abord, nous discutons de divers travaux en lien avec la Collaboration Humain-Robot afin de mieux comprendre le contexte de mon travail. Après une familiarisation avec le planificateur de tâches HATP/EHDA, je présente ma première contribution qui incorpore certains concepts de la Théorie De l'Esprit dans la planification de tâches. Certains modèles et algorithmes sont proposés et évalués pour mieux estimer et anticiper les connaissances de l'humain et son comportement. Ainsi, nous pouvons identifier les potentiellement néfastes fausses croyances de l'humain and ainsi l'informer proactivement pour corriger les fausses informations, ou volontairement retarder les actions du robot pour qu'elles soient vu par l'humain. Les résultats montrent que ce schéma permet de maintenir efficacement les connaissances de l'humain et permet de résoudre une classe de problèmes plus large que HATP/EHDA tout en ne communiquant pas systématiquement.

Ma deuxième contribution est une nouvelle approche de planification des tâches produisant une politique comportementale du robot assurant une collaboration fluide où l'humain a toujours une latitude de décision totale et où le robot se conforme toujours en parallèle à ces décisions. Cette approche est basée sur un modèle d'action conjointe simultanée et accommodante que nous avons conçu. Ce modèle, sous la forme d'un automate, tient compte de l'incontrôlabilité de l'humain et des signaux sociaux. Nous proposons également une nouvelle méthode d'évaluation et de sélection des plans basée sur l'estimation des préférences internes de l'humain concernant la tâche. Les résultats empiriques montrent que cette approche permet un comportement concourant du robot qui se conforme aux décisions et aux préférences en temps réel de l'humain.

Dans une autre contribution validant l'approche précédente, nous avons implémenté notre modèle d'action conjointe en tant que schéma d'exécution dans un simulateur. Nous avons ainsi mené une étude utilisateur où les participants ont collaboré dans plusieurs scénarios avec un robot simulé suivant les politiques produites par notre approche. En opposition avec notre approche, notre avons utilisé comme référence un robot imposant continuellement ses décisions à l'humain. Nous avons montré par une analyse statistique que notre approche permettait de nettement mieux satisfaire les préférences des humains. De plus, les differences les plus significatives sont que les participants ont perçu une interaction plus positive, une collaboration plus adaptative et efficace, et des décisions du robot plus adéquates et accommodantes.   

Enfin, mes dernières contributions concernent la simulation d'agents humains intelligents. Ses agents simulés dotés de processus de prise de décision peuvent aider à tester, évaluer et robustifier des systèmes robotiques intéractifs et collaboratifs. Nous proposons une architecture générique visant à simuler d'un tel agent intelligent et nous présentons une version implémentée pour le cas de la navigation. Nous présentons aussi une contribution capable de simuler plusieurs agents sociaux mobiles.

\textbf{Mot clés : } Collaboration Humain-Robot, Interaction Humain-Robot, Décision, False beliefs, Adaptation, Parallélisme


\tableofcontents

% \printnomenclature
% \printnoidxglossary[type=\acronymtype]
% \listoffigures
% \listoftables
% Use \mtcfixnomenclature below if you have a glossary (added with
% \printnomenclature above) and you're see a shift in the mini-table of
% contents at the begining of each chapter (example: no mini-toc in chapter 1;
% mini-toc of chapter 1 appearing in chapter 2; and so on).
%
% You should not use \mtcfixnomenclature if you have no glossary (that means,
% if you don't use \printnomenclature or if your glossary is empty).
%\mtcfixnomenclature

\mainmatter
% \chapter*{Introduction}
\addstarredchapter{Introduction}
\markboth{Introduction}{Introduction}

\newcommand{\mysection}[1]{%
    \section*{#1}%
    \addcontentsline{toc}{section}{#1}%
    \markright{#1}%
}


\minitoc

\acrfull{hrc} is a growing field in robotics and \acrfull{ai} research that aims to enable safe and effective teamwork between humans and robots.
This field mostly concerns fully autonomous robots. Hence, for instance, it excludes exoskeletons or teleoperated robots such as surgery manipulators or remotely operated (aerial) vehicles. 

In this manuscript, robots are considered autonomous tools able to interact physically with their environment. As tools, robots should facilitate human tasks by reducing both the required physical effort and mental workload. 
Industrial robots are already popular in factories because they are fast, accurate, reliable, and never tire, which makes them ideal for repetitive factory tasks. 
However, such robots are usually contained in dedicated areas where humans cannot enter for safety reasons. 
Hence, it is still an open challenge to endow robots with enough reliable reasoning capabilities and compliant motion control to allow efficient and trusted direct collaboration between humans and robots. 

This work aims to design autonomous robots able to make explainable, acceptable, and efficient decisions to collaborate with humans. Moreover, \acrshort{hrc} can also occur in various contexts that must be taken into account which range from co-worker robots in factories to householder robots for our everyday lives and include service robots in public places like restaurants and shops.


\mysection{Human-Robot Collaboration Challenges}

\acrshort{hrc} opens several challenges to address, each corresponding to a different aspect or functionality that the robot should be endowed with to fulfill its role. Hence, the ideal general collaborative robot should be the aggregation of all the notions below: 

\begin{itemize}
    \item \textbf{Navigation}: The robot should be able to acceptably and efficiently move in a human-populated environment. This implies being mechanically designed for it, anticipating and planning correct trajectories, and being able to adapt and follow these trajectories in real time. The robot should not move threateningly and should account for humans.

    \item \textbf{Manipulation}: The robot should be able to manipulate objects to interact with its environment. Hence, the robot should have an actuator like an arm and a gripper and should be able to exhibit motions that are efficient and safe to nearby humans.

    \item \textbf{Decision-making}: The robot should be able to make relevant decisions to be collaborative. This implies being able to plan its actions to solve a collaborative task. It also implies being able to supervise the task execution and make online decisions to adapt to uncertainties in real time, which includes human actions and commands. 

    \item \textbf{Communication}: To achieve congruent interaction and collaboration, collaborative agents must communicate. This implies that the robot should be able to communicate information to the human and understand the one received from the latter. These communications can be of various types, e.g., verbal, using natural language, prompting text on a screen, and signaling through robot movements (arms, head, mobile base). More innovative techniques can also be mentioned such as projecting arrows or desired paths on the ground, and using Augmented Reality to show internal information on the robot (decision, next action, etc...).

    \item \textbf{Perception}: Eventually, the robot must be able to perceive its environment. This is mandatory for the robot to have a reliable perception scheme to know the position of near objects, obstacles, and humans. First, relevant sensors must be used and placed on the robot or in the environment itself. After, from sensory data, some analysis and reasoning processes must extract relevant facts about the robot's environment such as objects' positions, spatial relations, reachable objects, human knowledge and intentions, the state of the current goal, and more. This constitutes the knowledge of the robot which may include an estimation of the near human knowledge. Perception is critical because most of the other challenges rely on the robot's knowledge.

\end{itemize}
    

\mysection{Contributions and manuscript organization}

The main challenge of HRC/HRI addressed in my PhD is decision-making. My work led to three main contributions concerning two distinguishable subfields. My two first contributions concern decision-making during task planning, to decide and plan the robot's action in order to solve a task collaboratively or simply in the presence of humans. My third contribution addresses decision-making in navigation, especially, how to simulate interactive social navigating agents endowed with decision-making processes to challenge robot navigation schemes.
As a result, the structure of my manuscript is as follows.

Chapter~\ref{chap:1} provides more details about the context of my PhD by discussing related fields and works of HRC. This chapter eventually highlights the gap in the literature that motivates my PhD work.

Part~\ref{part:1} gathers all my work concerning task planning for HRC. It represents a major portion of my PhD work and includes the chapters~\ref{chap:2}, \ref{chap:3}, \ref{chap:4}, \ref{chap:5}, and \ref{chap:6}.

Chapter~\ref{chap:2} familiarizes the reader with the HATP/EHDA task planner. Indeed, I participated in its development and this planner has been the keystone of my two contributions to task-planning for HRC. Hence, the reader should understand both this task planner's motivation and methods.

Chapter~\ref{chap:3} introduces my first main contribution which incorporates Theory of Mind concepts in HRC task-planning, more precisely, in the HATP/EHDA planning process.
Some models and algorithms are proposed and evaluated to better estimate human beliefs in order to better anticipate their potential actions. As a result, we can identify when the human has a false belief about a fact evaluated as relevant for the task. In such cases, the robot can proactively inform the human to correct the false belief or the robot can purposely delay its action to make sure the human sees its execution and infer the corresponding fact, avoiding verbal communication. 

In Chapter~\ref{chap:4}, as my second main contribution, I address the lack of concurrent actions of HATP/EHDA to improve fluency in collaboration. Inspired by the Joint Action literature, we designed a model of concurrent and compliant execution for HRC in the form of an automaton. We also propose to evaluate plans based on an estimation of the human inner preferences. A novel task planning approach taking into account the mentioned execution model and plan evaluation is proposed. This approach generates concurrent robot policies compliant with human online decisions and preferences. 

Chapter~\ref{chap:5} is a technical description of an interactive simulator I developed to execute the robot policy generated by the approach described in the previous chapter. This simulator proposes an execution scheme based on the model of execution to run and supervise the robot policy in a 3D simulator. It also allows a human operator to perform actions through intuitive mouse control. Hence, this simulator offers a way to collaborate with a robot following the approach we designed and is used in the next chapter to evaluate the approach.    

Chapter~\ref{chap:6} presents a user study validating the approach proposed in Chapter~\ref{chap:4} using the simulator described in Chapter~\ref{chap:5}. For this purpose, several scenarios have been designed using a BlocksWorld task, and human participants were asked to collaborate with the simulated robot to evaluate its behavior. We compared our approach with a baseline consisting of a robot following a simpler version of the model of execution. 

Conclusions regarding part~\ref{part:1} follow this chapter before starting the second part of this thesis. 

Part~\ref{part:2} concerns decision-making in navigation and how to simulate interactive social navigating agents. Despite not being my main research topic, this subject represents significant work in my PhD. This part includes the two last chapters: \ref{chap:7} and \ref{chap:8}.

Chapter~\ref{chap:7} describes my third contribution in which I address the decision-making challenge in navigation to allow the robot to move in a human-populated environment acceptably and efficiently. I developed a system producing an ``intelligent'' human avatar that, while being reactive, can make rational decisions about navigation tasks. This system serves as a benchmarking and testing tool for robot navigation systems to be challenged. This way, robot navigation systems can be evaluated, tuned, and stress tested in simulation allowing them to run mature real-life experiments faster.  

Chapter~\ref{chap:8} presents an additional work with the same rationales as the approach described in Chapter~\ref{chap:7} and is largely inspired by it. However, this work addresses a major limitation of the previous one which is not being able to simulate several human agents. This other approach can choreograph several agents with group movements and social behaviors. Despite having limited individual decision-making compared to the previous approach, this additional work generates pertinent and challenging intricate situations with several agents which is beneficial to the social robotic research.

Conclusions concerning both Chapter \ref{chap:7} and \ref{chap:8} end part~\ref{part:2}.

Eventually, I share general conclusions regarding all of my PhD work in a dedicated part before providing additional materials in the appendix.  

\mysection{List of Publications}

\subsubsection*{As main author}
\begin{itemize}

    \item Anthony Favier, Phani-Teja Singamaneni, Rachid Alami. Simulating Intelligent Human Agents for Intricate Social Robot Navigation. Social Robot Navigation workshop - Robotics: Science and Systems (RSS'21), Jul 2021, Washington, United States. 
    \item Anthony Favier, Phani-Teja Singamaneni, Rachid Alami. An Intelligent Human Avatar to Debug and Challenge Human-aware Robot Navigation Systems. Late Breaking Report - 2022 ACM/IEEE International Conference on Human-Robot Interaction (HRI'22), Mar 2022, Sapporo, Japan. 
    \item Anthony Favier, Shashank Shekhar, Rachid Alami. Robust Planning for Human-Robot Joint Tasks with Explicit Reasoning on Human Mental State. AI-HRI Symposium at AAAI Fall Symposium Series (FSS'22), Nov 2022, Arlington, United States. 
    \item Anthony Favier, Shashank Shekhar, Rachid Alami. Anticipating False Beliefs and Planning Pertinent Reactions in Human-Aware Task Planning with Models of Theory of Mind. PlanRob Workshop - International Conference on Automated Planning and Scheduling (ICAPS'23), Jul 2023, Prague, Czech Republic. 
    \item Anthony Favier, Shashank Shekhar, Rachid Alami. Models and Algorithms for Human-Aware Task Planning with Integrated Theory of Mind. IEEE International Conference on Robot and Human Interactive Communication (RO-MAN'23), Aug 2023, Busan, South Korea. 
    \item Anthony Favier, Phani Teja Singamaneni, Rachid Alami. Challenging Human-Aware Robot Navigation with an Intelligent Human Simulation System. Social Simulation Conference (SSC'23), Sep 2023, Glasgow, France. 
    \item Anthony Favier, Rachid Alami. Planning Concurrent Actions and Decisions in Human-Robot Joint Action Context. Symbiotic Society with Avatars workshop - ACM/IEEE International Conference on Human-Robot Interaction (HRI'24), Mar 2024, Boulder, United States.

\end{itemize}
    
\subsubsection*{As co-author}
\begin{itemize}
    
    \item Guilhem Buisan, Anthony Favier, Amandine Mayima, Rachid Alami. HATP/EHDA: A Robot Task Planner Anticipating and Eliciting Human Decisions and Actions. IEEE International Conference On Robotics and Automation (ICRA 2022), May 2022, Philadelphia, United States. ⟨10.1109/ICRA46639.2022.9812227⟩. 
    
    \item Phani-Teja Singamaneni, Anthony Favier, Rachid Alami. Towards Benchmarking Human-Aware Social Robot Navigation: A New Perspective and Metrics. IEEE International Conference on Robot and Human Interactive Communication (RO-MAN), 2023, Aug 2023, Busan, South Korea.
    \item Phani-Teja Singamaneni, Anthony Favier, Rachid Alami. Human-Aware Navigation Planner for Diverse Human-Robot Contexts. 2021 IEEE/RSJ International Conference on Intelligent Robots and Systems (IROS), Sep 2021, Prague (online), Czech Republic. 
    \item Phani-Teja Singamaneni, Anthony Favier, Rachid Alami. Invisible Humans in Human-aware Robot Navigation. IEEE International Conference on Robotics and Automation (ICRA 2022), May 2022, Philadelphia, United States.
    \item Phani-Teja Singamaneni, Anthony Favier, Rachid Alami. Watch out! There may be a Human. Addressing Invisible Humans in Social Navigation. 2022 IEEE/RSJ International Conference on Intelligent Robots and Systems (IROS 2022), Oct 2022, Kyoto, Japan. 

    \item Olivier Hauterville, Camino Fernández, Phani-Teja Singamaneni, Anthony Favier, Vicente Matellán, et al.. IMHuS: Intelligent Multi-Human Simulator. IROS2022 Workshop: Artificial Intelligence for Social Robots Interacting with Humans in the Real World, Oct 2022, Kyoto, Japan. 
    \item Olivier Hauterville, Camino Fernández, Phani-Teja Singamaneni, Anthony Favier, Vicente Matellán, et al.. Interactive Social Agents Simulation Tool for Designing Choreographies for Human-Robot-Interaction Research. ROBOT2022: Fifth Iberian Robotics Conference, Nov 2022, Zaragoza, Spain. 
\end{itemize}
    
    
    
% \ifdefined\included
\else
\setcounter{chapter}{0}
\dominitoc
\faketableofcontents
\fi

\chapter{Human-Robot Collaboration Context}
\chaptermark{Human-Robot Collaboration Context}
\label{chap:1}
\minitoc

%%%%%%%%%%%%%%%%%%%%%%%%%%%%%%%%%%%%%%%%%%%%%%%%%%%%%%%%%%%%%%%%%%%%%%%%%%%%%%%%%%%%%%

\chapabstract{This chapter provides more details about the context of my PhD by discussing related fields and works of HRC. This chapter eventually highlights the gap in the literature that motivates my PhD work.}

%%%%%%%%%%%%%%%%%%%%%%%%%%%%%%%%%%%%%%%%%%%%%%%%%%%%%%%%%%%%%%%%%%%%%%%%%%%%%%%%%%%%%%

\section{Multidisciplinarity of Human-Robot Interaction}

\begin{figure}
    \center
    \includegraphics[width=\linewidth]{Chapter1/hri_multi.pdf}
    \caption{Multidisciplinarity of the Human-Robot Interaction field.}
    \label{fig:hri_multi}
\end{figure}

In \cite{bartneck_human_robot_2020}, the \acrfull{hri} is considered unique because of the interaction of humans with social robots, which is at the core of this multidisciplinary research field. These interactions usually include physically embodied robots, and their embodiment makes them inherently different from other computing technologies. Moreover, social robots are perceived as social actors bearing cultural meaning and strongly impacting contemporary and future societies. Saying that a robot is embodied does not mean it is simply a computer on legs or wheels. Instead, we must understand how to design that embodiment, both in terms of software and hardware, as it is commonplace in robotics, and in terms of its effects on people and the kinds of interactions they can have with such a robot.

Overall, HRI focuses on developing robots that can interact with people in various everyday environments. This opens up technical challenges resulting from the dynamics and complexities of humans and the social environment. This also opens up design challenges—related to robotic appearance, behavior, and sensing capabilities—to inspire and guide interaction. From a psychological perspective, HRI offers the unique opportunity to study human affect, cognition, and behavior when confronted with social agents other than humans. In this context, social robots can be research tools to study psychological mechanisms and theories.

As a result, by taking inspiration from Human-Human Interaction (HHI) and Human-Computer Interaction (HCI), HRI is an endeavor that brings together ideas from a wide range of disciplines such as Engineering, Computer Science, Robotics, Psychology, Sociology, and Design by taking inspiration from Human-Human and Human-Computer Interaction.
In the following, we discuss some related aspects and works of the mentioned disciplines by categorizing them in HHI, HCI, and HRI as depicted in figure~\ref{fig:hri_multi}.

\subsection{Human-Human Interaction}

Many works dealing with interacting with humans take inspiration from Human-Human Interaction (HHI), including research in sociology and psychology. HHI refers to the communication and collaboration between two or more individuals, where humans engage in various forms of social, cognitive, and emotional exchanges. Such interaction can occur through verbal and non-verbal communication, such as speech, gestures, facial expressions, and body language.   

\textbf{TODO: detail more each aspect with words}

Communication theories, both verbal or not: Albert Meharbian's 7-38-55 rule \cite{mehrabian1967decoding}, and the Grice's four maxims of conversation called the Gricean maxims: quantity, quality, relation, and manner. These four maxims describe specific rational principles observed by people who follow the cooperative principle in pursuit of effective communication \cite{grice1975logic}.
\cite{smith_designing_1998}, \cite{cherry_human_nodate})

Joint Action theories, including collaboration and teamwork (\cite{cohen_teamwork_1991,cohen_team_1970,levesque_acting_1990})

Social psychology (Stanley Milgram, Philip Zimbardo, and Solomon Asch)

conflict resolution and negotiation (Roger Fisher and William Ury)

emotional intelligence (Daniel Goleman)

cross-cultural communication.


We must first understand how humans interact with each other before making robots able to interact correctly with humans. Nevertheless, mimicking humans perfectly is questionable since robots fundamentally differ from humans. Robots are created by humans to be helped and assisted. Thus, HHI should inspire robot design, but additional research is mandatory to determine how to create appropriate interactive and collaborative robots.

\subsection{Human-Computer Interaction}

A first step of artificial interaction and collaboration is the field of Human-Computer Interaction. Human-Computer Interaction (HCI) is the field of study that focuses on optimizing how users and computers interact by designing interactive computer interfaces that satisfy users' needs. It is a multidisciplinary subject covering computer science, behavioral sciences, cognitive science, ergonomics, psychology, and design principles.
Today, HCI focuses on designing, implementing, and evaluating interactive interfaces that enhance user experience using computing devices. This includes user interface design, user-centered design, and user experience design. 

This field is made up of four key components. 
The User along with their needs, goals, interaction patterns, cognitive capabilities, emotions, and experiences.
The Goal-Oriented Task which is the objective or goal the user has in mind.
The Interface is about the overall user interaction experience through senses such as touch, click, gesture, voice, display size, colors.
The Context must be taken into account because it influences the interaction. 

To produce easy to interact with robots, the study of HCI is relevant and also serves as an inspiration to design intuitive, user-friendly interactive robots.

\subsection{Human-Robot Interaction}

Human-Robot Interaction (HRI) is a field of study that explores the design, development, and evaluation of robots that interact with humans in various settings. HRI aims to create robots that can effectively and seamlessly collaborate with humans in domestic environments, workplaces, or other contexts. 
HRI can be categorized in several domains, not necessarily exclusive. Here are some examples:

\uline{Social robotics} focuses on social interactions with humans and, thus, explores how robots can understand and respond to human emotions, social cues, and communication styles.
A significant amount of work is dedicated to HRI in Healthcare to assist patients, especially the elderly and children with conditions. Those works are also usually linked to emotion-aware robotics focused on recognizing and responding to human emotions using affective computing techniques. A common application is storytelling for children to convey ideas, feelings, or culture. 

\uline{Human-Centered Robotics} emphasizes the importance of considering human needs and preferences. This subfield often involves user studies to ensure and identify if and how robots are user-friendly and can seamlessly integrate into human environments.

\uline{Robot Ethics} is another central subfield focused on considerations such as privacy, safety, responsibility/accountability, and the impact of robots on society.

\uline{Explainable AI and transparency} are a growing interest in making decision-making processes more understandable to humans, and thus, help robots be legible, predictable, and acceptable.

\uline{Computational HRI}, as described in~\cite{thomaz_computational_2016}, is the subset of HRI concerned explicitly with the algorithms, techniques, models, and frameworks necessary to build robotic systems that engage in social interactions with humans. This thesis is part of this category because it is focused on developing task-planning algorithms and models relevant to a collaborative robot. 

\uline{Human-Robot Collaboration} or \uline{Collaborative Robotics} focuses on developing robots that work alongside humans in shared workspaces, usually as a team. HRC is the main topic of my thesis and is a vast subject worth delving into. Hence, the following section is dedicated to providing more details about HRC.

%%%%%%%%%%%%%%%%%%%%%%%%%%%%%%%%%%%%%%%%%%%%%%%%%%%%%%%%%%%%%%%%%%%%%%%%%%%%%%%%%%%%%%
\section{Human-Robot Collaboration}

Human-Robot Collaboration (HRC) refers to the synergy and cooperation between humans and robots in shared environments to achieve common goals. In HRC, humans and robots work together, often leveraging their complementary strengths to enhance overall performance and efficiency. According to human desires, the robot can also act in a way that eases and facilitates the human part of the task. This collaborative approach involves close interaction, communication, and coordination between human and robotic agents.

\subsection{Inspirations \& Theories informing HRC}

This interdisciplinary field takes inspiration from various theories and fields as introduced earlier. Nevertheless, three main inspirations can be highlighted:

\subsubsection*{Belief Desire Intention Model:} The belief-desire-intention (BDI) model was originally developed by Michael Bratman~\cite{Bratman1987_BRAIPA}. This model is used in intelligent agents research to describe and model intelligent agents. Straightforwardly, the BDI model is characterized by the implementation of the three notions appearing in its name, i.e., an agent's beliefs (knowledge of the work in the perspective of the agent), desires (objective or goal to accomplish), and intentions (the planned course of actions to achieve the agent's desire). 

\subsubsection*{Shared Cooperative Activity:} Shared cooperative Activity defines prerequisites for an activity to be considered shared and cooperative. The main ones are mutual responsiveness, commitment to the joint activity, and commitment to mutual support. A good example to clarify these prerequisites is a scenario where agents move a table together. Mutual responsiveness ensures that the agents' movements are synchronized. The commitment to the joint activity reassures each agent that the others will not drop their side and quit the joint activity. Finally, the commitment to mutual support deals with possible breakdowns due to one agent's inability to perform part of the plan.  

\subsubsection*{Joint Intention \& Action Theory:} 
Joint Intention Theory proposes that for joint action to emerge, team members must communicate to maintain a set of shared beliefs and to coordinate their actions toward the shared plan~\cite{cohen_teamwork_1991}. In collaborative work, agents should be able to count on the commitment of other members. Therefore, each agent should inform the others when they conclude that a goal is achievable, impossible, or irrelevant~\cite{hoffman2004collaboration}.

\subsection{Key Aspects}

% More concretely, some key aspects of a seamless collaboration are listed and commented on below to better picture what the theories above involve in practice. 

In order to better picture the implications of the above theories, some key aspects of a seamless collaboration are listed and commented on below. This list is not exhaustive, but it highlights some skills that humans naturally exhibit and that a robot must be endowed with to collaborate with them. 

\textbf{Specialization of Roles:} There are several human-robot relationships, including supervisor-subordinate, partner-partner, teacher-learner, and leader-follower. These roles can be predefined and fixed during the whole collaboration. The role distribution can also be flexible using weighting functions that allow a continuous change between the roles to adapt to every context and situation.

\textbf{Establishing shared goal(s):} Through direct discussion or inference, agents must determine and agree on the shared goals they are trying to achieve. However, a shared goal isn't always necessary and can be established in the middle of a task execution either by the human or the robot.

\textbf{Allocation of subtasks:} After deciding how to achieve their goals, agents must determine what actions and subtasks will be done by each agent and how to coordinate each other. This can either be done explicitly before starting the task or be reactively done on the fly.

\textbf{Progression tracking:} Agents must be able to track progress toward their goals. That is, they must be able to determine what has been achieved, by whom, and what remains to be done. 

\textbf{Communication:} Any collaboration requires communication, verbal or not. Most of the mentioned aspects can or must involve communication. However, it is essential to identify what and how to communicate during the collaboration. Communicating too much can be annoying, while not enough can induce confusion and harm collaboration.

\textbf{Adaption and learning:} On a short-term scale, agents must adapt to each other and the environment. In the longer term, agents must also learn from other partners and the acquired experience.

\textbf{Ergonomics:} It should be intuitive to collaborate and communicate with the robot. This aspect must be taken into account when designing both the hardware and the software of the robot. Ergonomics is a central aspect of Human-Compute Interaction. Thus, many works from this field can be used in our context or serve as inspiration.

\textbf{Explainability:} This aspect is important for seamless collaboration as the human should be able to understand what the robot is doing and why. This topic is getting more and more attention and is often referred to as Explainable AI. This is especially relevant to counter the \textit{black box} tendency of machine learning where it's impossible to explain a specific decision. Being explainable often enhances predictability, which is also essential for a collaborative robot.

\subsection{Architectures \& Complete systems}

It is important to remember that since a collaborative robot is issued from an interdisciplinary field, its different functionalities and capabilities are usually separated into several dedicated components. These components interact and communicate with each other, forming a complete architecture. Such architectures cover all aspects relevant to exhibiting the robot's behavior, from sensory perception to physical motions, including reasoning processes. 
Despite developing distinct robotic components, the robot must be considered a whole. Each component cannot be studied entirely independently of other aspects of the complete system. As a result, optimizing by design the way these components interact has been the focus of several works proposing robotic architectures.

\begin{figure}
    \center
    \includegraphics[width=0.9\linewidth]{Chapter1/soar_architecture.png}
    \caption{The SOAR cognitive architecture.}
    \label{fig:soar}
\end{figure}

As Matthias Scheutz, Jack Harris, and Paul Schermerhorn put in \cite{scheutz_systematic_2013}, architectures for intelligent robots have improved steadily over the years. 
Early works like \cite{alami_designing_1993,alami_architecture_1998,chatila_integrated_1992} propose architectures to provide autonomy to mobile robots, focusing on three levels: decision, execution, and functional. Diverse components that let robots negotiate increasingly complex indoor and outdoor environments as been considered and improved such architectures over the years. As a result, current robot architectures integrate multiple sophisticated algorithms for real-time perceptual, planning, and action processing, from 3D object recognition to simultaneous localization and mapping to navigation and task planning, to action sequencing. However, classical robotic architectures like the ones mentioned above typically lack components for high-level cognition, such as general-purpose reasoning and problem solving. 
To address this issue, studies on cognitive robot architecture began mainly with the SOAR (depicted in figure~\ref{fig:soar}) and ACT-R architectures \cite{laird_soar_1987,anderson2004integrated}.
Often based on the structure of the human mind, such cognitive architectures aim to endow robots with high-level capabilities like learning, inferring, and reasoning about how to behave in response to complex goals in complex worlds. 

\cite{lemaignan_artificial_2017} identifies relevant collaborative cognitive skills and integrates them in a proposed architecture. The skills include geometric reasoning and situation assessment based on perspective-taking and affordance analysis; acquisition and representation of knowledge models for multiple agents (humans and robots, with their specificities); situated, natural, and multi-modal dialogue; human-aware task planning; human-robot joint task achievement.

\cite{thierauf_toward_2024} proposes another integrated cognitive robotic architecture more focused on self-awareness. It allows the robot to assess its own performance, identify task execution failures, communicate them to the humans, and resolve them, if possible. 



%%%%%%%%%%%%%%%%%%%%%%%%%%%%%%%%%%%%%%%%%%%%%%%%%%%%%%%%%%%%%%%%%%%%%%%%%%%%%%%%%%%%%%
\section{Models for interaction}


It has been shown previously that a robotic agent interacting with a human needs to coordinate its actions with them. Moreover, joint action theory exhibits that humans interacting together represent the task as a whole, and plan not only for their actions but also for the actions of other agents. Thus, we think that for a human to perform the most efficient and satisfactory joint task with a robot, this robot must explicitly model human actions and plan not only for its actions but also for the human ones. This is why we present in this section some notations to clarify the different models used in this thesis, and then we present in more details how to model tasks.

\subsection{Human and Robot agents}

\begin{figure}
    \centering
    \includegraphics[width=0.8\linewidth]{Chapter1/chakraborti_notations.jpg}
    \caption{Agent models from Chakraborti \textit{et al.} notations.}
    \label{fig:chakraborti_notations}
\end{figure}

Chakraborti \textit{et al.} introduced notations to differentiate between the models used in this thesis in their work~\cite{ChakrabortiBTZS15}, also used in Buisan's thesis~\cite{thesisBuisan21}. These notations summarize and differentiate elegantly the different models manipulated in the HRI/HRC field. These notations are depicted in figure~\ref{fig:chakraborti_notations}. At the bottom are depicted the human agent (on the left) and the robot agent (on the right). When solving a task alone, the robot uses its own model referred to as $\mathcal{M}^R$ and this is considered as Classical Planning. 
Then, $\mathcal{M}^H_r$ is an estimation by the robot of the model of the human. Finally, $\tilde{\mathcal{M}}^R_h$ is an estimation of the robot model the human has. These models are likely to include knowledge about the world in the agent's perspective, an action model describing the agent's capabilities, and an agenda capturing the goal and motivation of the agent.

It is important to keep in mind that when discussing a task planner it is considered as part of the robot. Thus, $\mathcal{M}^R$ is considered as the ground truth for the robot. As a consequence, if there is a belief divergence between $\mathcal{M}^H_r$ and $\mathcal{M}^R$, we always consider that $\mathcal{M}^R$ is the truth, otherwise, it would make no sense to keep this information in $\mathcal{M}^R$ while having access to the one in $\mathcal{M}^H_r$.

\subsection{Task modeling}

A common way of representing human activity (MrH ) and interaction with computers at a high abstraction level is by using task models. The hierarchical structure of human activity was first exploited by Annett and Duncan [Annett 1967]. They state that tasks can be described at several levels of abstraction until a certain criterion is met. Each task can thus be refined into subtasks detailing the procedure followed by the human to achieve the higher level task. Task modeling has then evolved to introduce interaction with systems, produced and needed information, potential errors and a wide variety of operator specifying how tasks interact with each other during their execution. Task models are now commonly used in user-centered and user-interface design processes. Most advanced notations include ConcurTaskTrees [Paternò 2004] and HAMSTERS [Martinie 2019]. These models are used to design or evaluate interactive systems. They allow the designer to better understand the user task or to study the user workflow using their system. However, these models contain too little information for a system to be able to reason and make decisions on them (either in planning or acting).


\subsection{Hierarchical models}

In classical planning, each action of an agent is atomic and needs some conditions to hold in the environment to be executed, and then it changes the environment when applied. The planning process has then to find the right sequence of actions, being applicable one after the other to change the environment to reach a specific goal state. 

However, humans tend to work more abstractly and decompose tasks hierarchically into smaller tasks until the action level is reached. In practice, using Hierarchical Task Networks (HTNs) allows the domain designer to help the plan search by inserting expert knowledge via a hierarchy linking the actions \cite{erol_complexity_1996}. A task network consists of tasks organized in a fully or partially ordered manner, and each task can be either abstract or primitive. Primitive tasks are elementary tasks that can be achieved by performing one associated action.
On the other hand, abstract tasks are tasks that first need to be decomposed into other subtasks, ``more primitive''. The planner's goal is not to find the sequence of actions to reach a goal but to select recursively for each task the suitable decomposition ending (if possible) with a network of actions applicable from the initial state. Ghallab, Nau, and Traverso name such a process as \textit{planning with refinement methods} \cite{ghallab2016automated}. This planning hierarchy allows the domain designer to guide the search by inserting some expertise into the model and enhancing explainability as the decompositions often offer a semantic to their subtasks. The 'why' of a subtask can usually be answered by going up in the hierarchy, while the 'how' is answered by going down. 

\begin{figure}
    \center
    \includegraphics{Chapter1/htn_example.pdf}
    \caption{An HTN example. Rectangles represent abstract tasks. Each method describes a possible way to decompose an abstract task if its preconditions are met. Methods can decompose abstract tasks into primitive tasks (ellipses) and/or into other abstract tasks. The obtained subtasks can be fully ordered, such as with \emph{method 1} (represented with a one-way arrow), or partially ordered, like \emph{method 2} (represented with a two-way arrow). Note that methods can also decompose a task into nothing like \emph{method 3}, for instance, when the task is already done, and they can be recursive like \emph{method 4}.
    }
    \label{fig:htn_example}
\end{figure}

To better picture HTN models, a symbolic example is depicted in figure~\ref{fig:htn_example}.
Rectangles represent abstract tasks. Each method describes a possible way to decompose an abstract task if its preconditions are met. Methods' preconditions are not necessarily mutually exclusive. Hence, as mentioned above, it is the planner's job to select the most suitable one when several ones are applicable.
Methods decompose abstract tasks into primitive tasks, represented with ellipses, and/or into other abstract tasks. The obtained subtasks can be fully ordered, such as with \emph{method 1} (represented with a one-way arrow) where \emph{Abstract Task 2} has to be completed before \emph{Primitive Task 1}. Methods can also be partially ordered as with \emph{method 2} (represented with a two-way arrow) where \emph{Primitive Task 2} and \emph{Primitive Task 3} can be achieved in any order. Note that methods can also decompose a task into nothing like \emph{method 3}, for instance, when the task is already done. Moreover, methods can be recursive like \emph{method 4}.

\begin{figure}
    \center
    \includegraphics[width=0.9\linewidth]{Chapter1/htn_decomposition_example.pdf}
    \caption{Two possible decompositions of a task network using the HTN described in fig~\ref{fig:htn_example}. Both \emph{method 1} and \emph{method 2} can be applied, leading to two different solutions.
    }
    \label{fig:htn_decomposition_example}
\end{figure}

Now, let us look at an example of a refinement process. We consider an initial task network only composed of \emph{Abstract Task 1}, and we refine it using the HTN described in fig~\ref{fig:htn_example} until the task network only contains primitive tasks (actions). Initially, \emph{method 1} and \emph{method 2} are both applicable and thus, are candidates to decompose \emph{Abstract Task 1}. Applying \emph{method 2} leads directly to a fully refined solution task network, including two partially ordered primitive tasks (\emph{Primitive Task 2} and \emph{Primitive Task 3}). On the other hand, \emph{method 1} can be applied, leading to a different task network that still includes abstract tasks. Then, \emph{Abstract Task 2} is recursively decomposed using M4 twice. This could correspond to scenarios like filling a box with two balls. Thus, M4 is only applicable until two more balls are in the box. Eventually, only M3 is applicable and refines \emph{Abstract Task 2} into nothing, leading to another solution task network.

%%%%%%%%%%%%%%%%%%%%%%%%%%%%%%%%%%%%%%%%%%%%%%%%%%%%%%%%%%%%%%%%%%%%%%%%%%%%%%%%%%%%%%
\section{Task Planning}

\subsection{Classical Planning}

As put by Ghallab, Nau and Traverso, “the purpose of planning is to synthesize an organized set of actions to carry out some activity”~\cite{ghallab2016automated}. 
Classical planning is a type of planning that assume deterministic and fully observable environments. It involves representing the world as a set of states and actions, with plans derived through state space search algorithms. Actions have preconditions and effects, and planning problems entail finding a sequence of actions transforming an initial state into a goal state. Classical planning algorithms, including STRIPS, Graphplan, and Fast Downward, utilize heuristics to guide the search efficiently. While well-suited for domains with clear and deterministic dynamics, classical planning may face challenges in handling uncertainty or partial observability, leading to the development of alternative planning approaches for such scenarios.

\subsection{Planning for HRC}

Classical planning has been vastly studied and can now solve efficiently various problems. Yet, the intricate nature of HRC scenarios demands sophisticated task-planning methodologies capable of adapting to dynamic environments, understanding human intent, and promoting a fluent exchange of information. Hence, several subfields of task-planning have emerged and are used in HRC. Here are a few examples:

\begin{itemize}
    \item \textbf{Hierarchical Task Planning} is a technique that organizes tasks in a hierarchical structure like presented in the previous section, allowing for the representation of complex tasks at various abstraction levels. This approach enhances modularity, flexibility and explainability in task planning, accommodating intricate collaborative scenarios.
    
    \item \textbf{Mixed-Initiative Planning} leverages the strengths of both humans and robots by allowing for a dynamic allocation of decision-making authority. This technique promotes collaborative decision-making, enabling the system to adapt to the expertise and preferences of each agent involved in the collaborative task.
    
    \item \textbf{Human-Centric Task Planning} focuses on incorporating human factors into the planning process. This involves understanding human capabilities, preferences, and cognitive load to optimize task plans that align with the natural workflows and expectations of human collaborators.
    
    \item \textbf{Learning-Based Task Planning} has emerged thanks to advancements in machine learning as a frontier in adapting to evolving environments. This technique involves training models to understand patterns in human behavior, enabling the robot to learn and adapt its task planning strategies over time. Such techniques can also be used to predict human behavior and consequently adapt the robot's actions.
    
    \item \textbf{Probabilistic Task Planning} integrates uncertainty into the planning process, acknowledging the inherent unpredictability of human behavior and environmental factors. By incorporating probabilistic models, this technique enhances the robustness of task plans in dynamic and uncertain collaborative settings.
    
    \item \textbf{Task and Motion Planning} combines symbolic and geometric reasoning to plan agents' actions. In our context, it can be helpful to consider safety spatial areas near humans and adapt both the robot's motion and decisions. 
    
\end{itemize}

Overall, \textbf{Human-Aware Task Planning} is the process of considering the presence and behavior of humans in the planning and execution of robot tasks. It involves taking into account cues from the shared environment and the dynamics of human-robot interaction. The goal is to generate robot policies that are adaptable, robust, and efficient in crowded and dynamic environments. A detailed presentation and discussion about existing task planning works for HRC is provided in the related work section (\ref{sec:ch2_related_work}) of Chapter~\ref{chap:2}.  

\textbf{TODO: remove?}\textit{This field includes having an explicit shared task or common goal between the human and the robot, implying that the two agents will collaborate to reach the goal. It also includes not having an established shared goal, and thus, is closer to what I would call an interaction instead of a collaboration. Yet, both problems are interesting and must be addressed, with a unified approach or not. }

\subsection{Other usage of task planning}

Despite being designed for Human-Robot Collaboration, the task planning techniques presented in this work can be used in other interactive contexts. 

For instance, instead of planning robot actions, we could plan verbal answers in a dialogue. 
This approach is used in \cite{de_carolis_verbal_2000} and \cite{de_carolis_behavior_2001}.
Hence, some algorithms and models proposed in this manuscript could be used to anticipate the possible human sentences and plan the relevant robot communications. 

Additionally, one could think about a smart environment, e.g., a domotic house, where various sensors and actuators are connected. When humans explore and operate in such an environment, it could be relevant to plan domotic actions according to human behavior, e.g., proactively making coffee, opening stores, activating the robot vacuum cleaner, etc. This context is addressed in \cite{pecora_constraint_based_2012} to provide proactive human support.


%%%%%%%%%%%%%%%%%%%%%%%%%%%%%%%%%%%%%%%%%%%%%%%%%%%%%%%%%%%%%%%%%%%%%%%%%%%%%%%%%%%%%%
\section{Human-Aware Navigation}

In my work, I studied the decision-making challenge mainly in the field of task-planning. 
Nevertheless, I also worked on decision-making processes for navigation, more precisely, on how to simulate a navigating interactive human agent endowed with decisional capabilities to challenge robotic navigation systems. Hence, to better understand this contribution, this section provides some context on robot navigation, especially on state-of-the-art techniques and existing benchmarking tools.

As stated in \cite{thesisBuisan21}, robot navigation aims to make the robot base (the whole robot) move from one place to another while avoiding static and moving obstacles. However, other constraints must be added when the robot has to move in an environment where humans are evolving. 
Humans should not just be avoided as other moving obstacles, and their psychological and mental state must be taken into account. Hence, the robot should neither move threateningly, block the humans, nor induce drastic changes in their motion. Taking all these aspects into account is what is called human-aware robot navigation.

\subsection{Robot Navigation techniques}

State-of-the-art techniques for robot navigation involve two kinds of motion planners: a global and a local planner. The global planner is in charge of finding the best overall trajectory leading the robot to its goal, producing a global plan. This planner usually only takes into account static obstacles described by a given map of the environment. Then, the local planner is in charge of producing velocity commands sent to the motor controllers to follow the produced global plan. To produce the velocity commands the local planner may produce a local plan with only a few seconds of time horizon that follows the global plan while taking into account obstacles detected in real-time by the robot sensors, including moving obstacles. This way the robot should reach its goal while being reactive to moving obstacles. 

However, as explained above, such approaches are insufficient in human-populated environments. Humans must be detected and treated differently during the motion planning process. Human-aware approaches detect and track nearby humans and try to estimate their trajectory to plan the robot one accordingly. This is achieved in works like \cite{singamaneni2021human}, where the human trajectory is estimated using goal recognition processes and elastic bands methods. Then, the robot's motion is planned using tuned elastic band methods to account for the robot's goal, the estimated human trajectory, and other social norms.   

\subsection{Benchmarking tools and metrics}

Where it is easy to benchmark robot navigation on objective metrics like the time to reach a goal, the distance traveled and the number of collisions \cite{perille2020benchmarking}, it is more challenging to benchmark their human-aware properties.
First, there is no consensus on the metrics to evaluate a navigation system's human-aware properties. State-of-the-art metrics involve proxemics \cite{samarakoon2022review}. However, other metrics can be relevant, such as the deviation imposed on human motion and the feeling of threat produced.
Also, it is challenging to find a usable system that will effectively challenge a HAN system. 

\begin{figure}
    \center
    \includegraphics[width=0.40\linewidth]{Chapter1/social_force.png}
    \includegraphics[width=0.54\linewidth]{Chapter1/crowd1.png}
    \caption{Social force model approximating pedestrian motion by a sum of forces.}
    \label{fig:social_force_model}
\end{figure}

Typical approaches involve reactive-only techniques such as social force models \cite{helbing1995social,chen_social_2018}. 
The social force model assumes that a sum of different forces can approximate pedestrians' acceleration, deceleration, and directional changes, each capturing a different desire or interaction effect. For instance, as depicted in figure \ref{fig:social_force_model}, one force corresponds to the acceleration towards the desired velocity of motion; second, repulsive forces reflect the agent keeping a certain distance from other agents and obstacles; third, attractive forces represent the goal and motivations of the agent. Eventually, using standard physics equations, the sum of these dynamic forces describes the agents' motion.
Such reactive models are easy to use and efficient for crowd simulations. Interestingly, in crowded or evacuation scenarios, social force models exhibit several so-called ``self-organization phenomena'' such as lane formation, zipper effect, intermittent flow, or turbulence.
Unfortunately, such a model can perform very poorly in intricate and non-crowded scenarios involving some decision-making. Thus, there was a lack of intelligent simulated agents to challenge effectively HAN system.
A few recent works also propose simulating human agents endowed with some reasoning processes, but I did not find the time to compare my contribution with them and it remains an interesting possible future work. Nevertheless, this shows that this is a subject of interest. Some related works will be discussed in Chapter~\ref{chap:6}.

% \ifdefined\included
\else
\setcounter{chapter}{1} %% Numéro du chapitre précédent ;)
\dominitoc
\faketableofcontents
\fi

\chapter{A Human Aware Task Planner Emulating Human Decisions and Actions (HATP/EHDA)}
\chaptermark{HATP/EHDA}
\label{chap:2}
\minitoc

\section{Introduction}

I was introduced to the field of task planning the work of a PhD student from my lab, Guilhem Buisan. I slightly contributed to the original version and then proposed two extensions of his work. Yet, Guilhem mainly designed and implemented this novel human-aware task planning approach dedicated to HRI which plans the robot actions while estimating and emulating the human decisions and actions, namely HATP/EHDA. 

We believe this planning approach suits well the needs of HRI scenarios, and thus, it became a laboratory to address relevant challenges of task-planning for HRC. 
Two of my main contributions consist of addressing such challenges and then implementing the solutions as extensions of HATP/EHDA.   
As a consequence, it is important to understand this work well, both its motivation and methods, before introducing my proper contributions. This section introduces, motivates and explains the HATP/EHDA approach as a background to the other chapters. 
Naturally, a detailed description of this prototypical planner is already given in Buisan's thesis~\cite{thesisBuisan21}. Thus, large parts of this section are directly retrieved from Buisan's thesis, but they are important to have in mind. Some notations are adapted to match my contributions' descriptions in the next chapters. 

\section{Related work}

\subsection{HATP}
\textit{HATP/EHDA is inspired by the hierarchical agent-based task planner HATP (citation). In addition to ``standard'' task planning metrics like plan length, HATP takes into account social rules and costs to produce the robot plan. This way, it aims to produce a plan which will be acceptable and appreciated without having to negotiate with the human. However, although the plan aims to be socially acceptable, the human must follow the plan produce and has no choice to make. 
In some scenario this can be frustrating, and it also doesn't account for contingencies in terms of human action. If the human diverge from the plan the execution must be stopped and the plan either repaired or even replan. 
In opposition, HATP/EHDA generates robot policies instead of plans. In a turn taking manner where the human usually starts, the generated policy indicates the best robot action to perform according the previously performed human action. Thus, the human is free to choose online the action they want to perform, and the robot will account for this decision. This process neither requires prior negotiation with the human nor an established shared goal.}


The Hierarchical Agent-based Task Planner (HATP) proposes a hierarchical approach to multi agents task planning. This HTN-based planner is able to elaborate a multi-agents plan based on a single HTN tree. Moreover, it maintains one beliefs base per agent allowing to write task decomposition rules and actions preconditions and effects in any agent beliefs base. Finally, HATP also computes costs for the plans found based on action costs and predefined social rules. However, HATP assumes that a shared goal has been established between the human and the robot prior to the planning process and that the generated plan will be shared with the human before the execution. Indeed, HATP does not represent the human as an agent having a separate decision process that may lead to diverging plans without robot communication. Other approaches are explicitly considering an external human model, which can be used to predict future human actions, and plan accordingly

*******

HATP extends the classical Hierarchical Task Network (HTN) planning by being able to produce shared plans to reach a joint goal. A HATP planning domain describes how to decompose tasks into subtasks down to atomic symbolic actions. Both the robot and human feasible tasks and actions are described in the domain. A contextdependent cost function is associated with each action. During the task decomposition, HATP will explore several applicable sub-tasks until the global task is totally refined into feasible actions, and will return the minimal cost plan. HATP also supports social rules, allowing to balance the effort of involved agents depending on human preferences and to penalize plans presenting certain undesirable sequences of actions. We will not use these social rules in what follows, but our approach stays totally compatible with them. Moreover, during the exploration of the task tree, HATP will assign actions to available agents, robot or human (when an action can be done by both). By doing so, HATP is able to elaborate one action stream per agent, together with causality and synchronization links. Besides, HATP domain syntax supports Multiple Values State Variables (MVSV) [Guitton 2012] which is used to represent and reason about each agent mental state. The value of each variable depends on the agent it is requested for. This allows to represent action preconditions depending on the knowledge of the agent performing the action and also to represent their effect on each agent mental state which can depend on the agent perspective. Finally, the last argument which motivated our choice was the previous integration of HATP with a Geometrical Task Planning (GTP) [Gharbi 2015a]. This work aimed at refining geometric and motion planning requests during the task planning process. The geometric planner would then compute, in context, the feasibility, the cost and the side effects of the action. In a similar way, we propose here to integrate and run REG, in context, to determine communication action feasibility and pertinence with respect to other courses of actions. The HATP Data Structures As presented in [De Silva 2015], the HATP data structures are organized around the so-called HATP entities. These entities are a collection of any number of attributes being manipulated during the planning process. The type of the attributes can either be a basic type (i.e. integer, floating point number, boolean or string) or another entity type. Besides, an attribute can either be a set, holding multiple values of the specified type or an atom, being a unique element of the specified type. Finally, an attribute can either be set as static or dynamic. Static attributes cannot be modified once they have been set in the world state initialization, they will only be read during the planning process, whereas dynamic attributes can also be changed by the effects of the primitive tasks (actions). The Listing 3.1 gives an example of entity definitions in order to represent the situation in Figure 3.1. // Agent entity type i s i m p l i c i t define entityType Cube , Area ; define entityAttributes Cube{ dynamic atom Area i s I n ; dynamic atom Agent isHeldBy ; } define entityAttributes Area{ dynamic set Cube h a s I n ; } define entityAttributes Agent{ s t a t i c atom s t r i n g type ; dynamic atom Cube i s H o l d i n g ; } Listing 3.1: Example of a part of the HATP domain describing the situation of Figure 3.1.


Although we saw that this approach needs a task planner able to maintain one set of beliefs per agent during the planning process, HATP, the planner used was only allocating tasks to the human without considering if the human was aware or not of the generated plan. HATP is provided with a human model (MrH ) in several ways. The planning domain contains the actions the humans can perform and their beliefs are updated all along the planning process. Moreover, human preferences can be set on the plan through the social cost, adding extra-cost to the plan if some criteria are met such as bad chaining of actions (e.g. the robot mopping the floor just before making a sandwich). However, HATP assumes that a shared goal has been established between the robot and the human. This shared goal is supposed to have been acquired previously in the interaction, for example via a human request. Besides, as stated before, it generates a plan which is unknown if it needs to be communicated to the human or if it can be easily guessed (i.e. which is predictable) by the human. Finally, the plan is “fixed” and does not account for whether the human chooses to perform a different path of actions or not. Any deviation of the human from their generated stream of action either needs the supervision to perform repair actions or to request for a replanning. This approach has been shown to be suitable and pertinent in some applications (e.g. when communication can easily be done at any point of the plan). 


\subsection{Other human-aware task planner}

\textbf{TODO: List planners and approaches, general and some specific ones? maybe not, already done in next chapters}

\section{A human aware task planner}

\textbf{TODO: Divide in Exploration and Selection (cost?), this would make the cost as a Section and not only a subsection (more visibility). Cost Chapter is too much I think...}

\subsection{Rationale}

HATP/EHDA has been developed to try to satisfy several objectives: \textbf{TODO: add Ghilhem text?}

\begin{enumerate}
    \item \textbf{Plan without assuming a prior shared goal.} In HRI scenarios, the robot and the human are not always sharing a goal. The robot can for example plan to perform a task around humans that are not involved at first, or it may be requested by a human to do a task without wanting to take a part in it. HATP/EHDA can balance between integrating the sharing of a goal with a human (assumed to be collaborative) in the plan and making the robot do the task alone, or integrate the eventuality to ask for punctual human help. 

    \item \textbf{Model the human decision processes.} When taking part in a task, a human (assumed wiling to collaborate with the robot) will also plan to reach their (potentially shared) goal. HATP/EHDA must be able to account for this to provide plans that are expected and explainable by the human partner.

    \item \textbf{Help the human decisions, but not compel them.} Unlike HATP, HATP/EHDA should account for the human flexibility in their decision. While by modeling the human decision processes it is possible to narrow down the possible human actions, the generated plans must be able to help the supervision (execution of the plan) to avoid replanning or repairing during the execution by considering several human actions.

    \item \textbf{Model the potential human reactions.} It is possible to predict that the human may react to some situations, interrupting or helping their current task. Two causes have been identified for these reactions. First, they can ensue from some specific world states, that have been perceived and interpreted by the human. Then, they can also originate from explicit communications issued by the robot. These communications can either be a belief alignment, updating the human knowledge and impacting their decisions; a request to perform a specific action or a request to help the robot along with a shared goal, needing the human to plan for it.

    \item \textbf{Act and decide on the different agents' beliefs.} It is important to be able to represent actions as having different effects on the beliefs of the robot or the human. Indeed, some robot actions are partially or not observable by the human, when performing them, the human has no way of knowing the complete new world state. Besides, these effects and their observability often depend on the current world state, which representation must be supported in the planner. Then, decisions made while planning may require to reason on both the robot and the human beliefs. This is especially the case with communication actions aiming at aligning knowledge or ask questions for example. Finally, some actions of pure decision have no direct effect on the world, but only on the internal beliefs of the agents. For example, observation actions will only update the beliefs of the agent doing it.

    \item \textbf{Decide not only on the world state but also on the decision processes of the agents.} Some decisions made during the planning process require access not only to the beliefs of the agents representing the world state, but also to the estimation of their planning processes. For example, the decomposition of a task by the robot may be impossible if some other task is already performed in its partial plan. Other decisions may also need the estimation of the human current planning process. For example, if it has been estimated earlier in the plan that the human will perform a certain decomposition of a task, the planner would assign a complementary task to the robot.

    \item \textbf{Adapt to the human experience, trust and preferences.} We also want the planning process to be adjusted depending on the actual human it is planning with. It must perform its plan search differently whether the human has the habit to perform this particular task with the robot or not. Moreover, the human model can be adjusted to the trust the human has in the robot and to their preferences.

\end{enumerate}

\subsection{Problem Specification}

\textbf{TODO: Give one general and overall problem specification notation. For instance, $P = \{ \mathcal{M}_H, \mathcal{M}_R \}$}

Use Distinct Agent models(beliefs, HTN, agenda, triggers)

\textbf{TODO: Specify notations for agent model. For instance, $M_\varphi = \{ val_{\varphi}, tn_{\varphi}, \Lambda_{\varphi}, T_{\varphi} \}$}

One of the strength of HATP/EHDA is that it utilizes two distinct agent models. Each model comprises the following:
\begin{itemize}
    \item \textbf{Beliefs}: estimation of the world state from the agent's perspective.
    
    \item \textbf{Agenda}: capturing the personal and/or shared goals of the agent, currently implemented as a task list/sequence but could be generalized to partially ordered task networks.
    
    \item \textbf{Action Model}: encoding the capabilities of the agent and used to estimate the next actions of the agent given a goal and a world state. Here described by a hierarchical task network (HTN)
    
    \item \textbf{Triggers}: describes the reactions the agent may have which might update their agenda. That is, the agent may react to a specific world state, event sequence, or an explicit communication. For instance,  consider a scenario where suddenly another agent is handing over an object to the agent. This event has nothing to do with the agent's goal, and thus, the next agent action extracted from the Action Model might not consider the other agent. However, a natural reaction to this situation is to grab the handed object. Thanks to the Triggers mechanic, we can in an automated way estimate that whatever the agent was doing, when being handed an object the agent will grab it.  

\end{itemize}

The planner uses two agent models, one for the human and one for the robot. Despite there same structure there is a fundamental difference between the two models: one is a controllable agent and not the other. Indeed, the human model is only used to speculate the human decision and actions in given situations. 
Then, the robot model is used to plan the robot action according to the estimated human actions.
Note that the human decisions can still be influenced by the robot actions, but they cannot be compelled.
It's also important to note that the two agents are not equivalent, the robot agent role is to help, assist and facilitate human, and thus, to synthesize pertinent, legible and acceptable behavior.


\textbf{TODO: Add HTN information, HTN can be undecidable? (strictly more expressive than STRIPS)}

To give more details, this is what HATP/EHDA formalization can look like.

We consider a \textit{classical planning domain} (\textit{state-transition system}) $\Sigma = (S,A,\gamma)$, s.t., $S$ is a finite set of states in which the system may be, $A$ is a finite set of actions that the actors may perform, $\gamma : S \times A \rightarrow S$ is a state-transition function. Each state $s \in S$ is a description of the properties of various objects in the planner's environment~\cite{naubooks0014222}. 

To represent the objects and their properties, we will use two sets $B$ and $X$: $B$ is a set of names for all the objects, plus any mathematical constants representing properties of those objects. $X$ is a set of syntactic terms called state variables, s.t. the value of each $x \in X$ depends solely on the state $s$.

A \textit{state-variable} over $B$ is a syntactic term $x = sv(b_1, ..., b_k)$, where $sv$ is a symbol called the state variable's name, and each $b_i$ is a member of $B$ and a parameter of $x$. Each state variable $x$ has a range, $\textit{Range}(x) \subseteq B$, which is the set of all possible values for $x$.



Here is the description of the sets $B$ and $X$ for the collaborative cooking example:
\textbf{TODO: Use other example, ICRA cube stacking?}
{\small
\begin{align*}
&B           = Entities \cup Places \cup Booleans \cup \{\textsf{nil}\} \\
&\quad Entities    = Agents \cup Objects\\
&\quad Agents      = \{ \textsf{R}, \textsf{H} \} ~~ \backslash\backslash~\textsf{R}:robot,~\textsf{H}:human\\
&\quad Objects     = \{ \textsf{salt}, \textsf{pasta}, \textsf{counter} \}\\
&\quad Places      = \{ \textsf{kitchen}, \textsf{room} \}\\
&\quad Booleans    = \{ \textsf{true},\textsf{false} \}\\
&\\
&X = \{ at(e), saltIn, stoveOn, counterClean ~ | ~ e \in Entities \}\\
&\quad \textit{Range}(saltIn ~|~ stoveOn ~|~ counterClean)=Booleans\\
&\quad \textit{Range}(at(\textsf{R} ~|~ \textsf{H} ~|~ \textsf{pasta})) = Places\\
&\quad \textit{Range}(at(\textsf{salt} ~|~ \textsf{counter})) = \{ \textsf{kitchen} \}
\end{align*}
}

A \textit{variable value assignment} function over $X$ is a function $val$ that maps each $x_i \in X$ into a value $z_i \in$ $\textit{Range}(x_i)$. With $X = \{ x_1, ..., x_n \}$, we will often write this function as a set of assertions: $val = \{ x_1=z_1, \ldots, x_n=z_n \}$. 


A \textit{state}, or \textit{planning state}, $s_i \in S$ is a 4-tuple composed of 2 functions over $X$ (agents beliefs) and 2 task networks (agents agenda)  s.t. $s_i = (val^R_i, val^H_i, tn^R_i, tn^H_i)$. 
The state of the world from the perspective of the robot is captured by the variable value assignment function $val^R_i$. Since the planner is assumed to be part of the robot, the beliefs of the robot are assumed to be the ground truth, and thus, are sometimes noted as $val_i$. 
Similarly, $val^H_i$ represents the estimation of $val_i$ in the perspective of the human, also referred to as the estimated human beliefs. 
We say that a state $s_i \in S$ contains \textit{false beliefs}, or \textit{belief divergences}, if $\exists x_j \in X, val^H_i(x_j) \neq val^R_i(x_j)$. 
Be careful not to get confused, a \textit{state} is a state in which the planning problem is while being solved, and each \textit{state} is connected to other states through agent actions. Where the \textit{beliefs} are the state of the world in the perspective of an agent, which is part of the \textit{planning state}.

For our example, the initial state $s_0$ would be as follow: 

{\small
\noindent
\begin{multline*}
s_0 = \{val^R_0, ~val^H_0, ~tn^R_0, ~tn^H_0\} \\ \quad
\end{multline*}
\bigvspace
\begin{multline*}
\quad val^R_0 = val^H_0 = \{at(\textsf{R}) = \textsf{kitchen}, at(\textsf{H}) = \textsf{kitchen},\\ 
at(\textsf{pasta}) = \textsf{room}, saltIn = \textsf{false}, stoveOn=\textsf{false} \}
\end{multline*}
\smallvspace
\begin{multline*}
\quad tn^R_0 = \{ CookPasta, CleanCounter \} \\ \quad
\end{multline*}
\bigvspace
\begin{multline*}
\quad tn^H_0 = \{ CookPasta \} \\ \quad
\end{multline*}
\smallvspace
}

An action is a tuple $\alpha = (\textit{head}(\alpha), \textit{pre}(\alpha), \textit{eff}(\alpha))$ where $\textit{head}(\alpha)$ is a syntactic expression of the form $\textit{act}(z_1, ..., z_k)$ where $act$ is a symbol called the \textit{action name} and $z_1,...,z_k$ are variables called parameters. $\textit{pre}(\alpha) = \{ p_1, ..., p_m \}$ is a set of preconditions, each of which is a literal. And $\textit{eff}(\alpha) = \{ e_1, ..., e_n \}$ is a set of effects, each of which is an expression of the form: $sv(t_1, ..., t_j) \leftarrow t_0$ with $t_0$ being either the value to assign to the state variable $sv(t_1, ..., t_j)$ or a new location/place for the state variable. We note $\textit{agt}(\alpha)$ the agent performing the action $\alpha$.

To estimate the next possible actions that an agent $\varphi \in Agents$ is likely to perform in a state $s_i \in S$, we proceed in the same way as in~\cite{buisan:hal-03684211}. We refine the agent's agenda $tn_{\varphi}$ based on its belief $val^\varphi_i$ and obtain a \textit{refinement} as follows $\textit{ref}(tn^\varphi_i, val^\varphi_i)= \{ (a_1,tn_1),...,(a_j,tn_j) \}$. 
A \textit{refinement} contains a tuple for each estimated possible action $a_j$ and the associated new agenda $tn_j$ after being refined. 

In our cooking example, we obtain the following refinement if the starting agent is the human:\\
{\small
$\textit{ref}(tn^H_0, val^H_0) = \{ (add\_salt(),tn_1), (move\_to(\textsf{kitchen}),tn_2) \}$
}

State transition function:

Thus, $\forall x \in X$, we always have,

\begin{equation}
    val_{i+1}(x) = \left\{ 
    \begin{array}{ll}
        w, & \mbox{if} ~ x \leftarrow w \in \textit{eff}(a)   \\ 
        val_i(x), & \mbox{otherwise}
    \end{array}\right.
\end{equation}

\textbf{TODO: Add trigger? Might be better to explain them in planning process}


\subsection{solution format}
AND/OR Tree resulting in a robot policy. Add figure of AND/OR Tree, we plan in a turn taking fashion where each human action is an AND edge and each robot action is a OR edge, and nodes planning states (beliefs + agendas).

\subsection{Planning process}
Planning process using HTNS action models

Especially the refinement process algorithm. Probably only description.

\subsection{Solution selection, costs}

Talk about cost in human-aware task planning. A trade-off between efficiency and social criteria. The robot should be efficient but also behave in an acceptable, legible, and accommodating manner.  


Cost evaluation is tricky, objective metrics are easy to use for efficiency, however, social criteria are harder to compute because they are hard to generalize. These social rules can be very context dependent which make them hard to generalize and thus to take them into account in a reliable manner.

Plan cost is a mix of all the following:

\begin{itemize}
    \item length of the plan (nb of action or temporal duration if available)
    \item sum of individual action cost: the cost of each action can be estimated to translate the effort required to perform it. It can reflect several aspect and constraints such as: pure physical strength required, duration of the action, energy consumption
    \item Undesired states: Common sense and social norms can be used to define several rules defining undesired state. This can cover various aspects such as hygiene or safety. For instance, despite being possible and maybe efficient, we wouldn't like a robot holding a dirty dripping mop in one hand and the sandwich we asked for in the other hand. Another example, we wouldn't like a robot dropping a knife just on the edge of a table or counter because it may fall and be dangerous. 
    \item Undesired sequences of actions: For the same reasons, we can also define undesired sequences of actions. This can express preferences regarding the ordering of different subtasks, e.g., since we don't want the robot to hold our sandwich while cleaning the house, we would also not like the robot to clean first and then make a sandwich because the robot is likely to be dirty while making the sandwich.  
\end{itemize}

I implemented in HATP/EHDA a way to specify and take into account undesired state and action sequences. Detecting any of them in a possible plan would penalize the plan cost of the specified amount. Here, the undesired elements must be specified in the problem specification and are abstracted in the planner. However, as stated above, it is hard to generalize undesired states and action sequences to integrate them directly in the planner to avoid specifying them in the problem specification.   

\section{Examples}

ICRA paper ?

\subsection{Problem description}
\subsection{How it is solved}

% \ifdefined\included
\else
\setcounter{chapter}{2} %% Numéro du chapitre précédent ;)
\dominitoc
\faketableofcontents
\fi

\chapter{Models and Algorithms for human-aware task planning with integrated theory of mind}
\chaptermark{Models and Algorithms of Theory of Mind}
\label{chap:3}
\minitoc

\chapabstract{This chapter presents my first main contribution, proposing models and algorithms to incorporate Theory of Mind concepts in HRC task-planning. An empirical evaluation is provided and discussed, demonstrating how this contribution allows solving a broader class of problems than HATP/EHDA, without systematicaly using communication.}

%%% SECTION %%%%%%%%%%%%%%%%%%%%%%%%%%%%%%%%%%%%%%%%%%%%%%%%%%%%
\section{Introduction}



\begin{figure}
    \centering
    \includegraphics[width=0.5\linewidth]{Chapter3/The-Sally-Anne-Task.jpeg}
    \caption{Sally and Anne Task}
    \label{fig:sally_and_anne_task}
\end{figure}

False beliefs tasks are commonly used as tests to acknowledge the presence of Theory of Mind reasoning. The most common test is the \textit{Sally and Anne} one, depicted in fig~\ref{fig:sally_and_anne_task}, and is used in developmental psychology to examine children's "theory of mind" understanding, which refers to their ability to understand how other people think, feel and behave. The test consist in describing the execution of a simple task involving non-observable objects and co-presence, then children are questioned about the belief of one of the character. The task consist in the following. Sally and Anne are two co-present characters near a basket and a box. Sally puts her ball in the basket before going away. Then, Anne moves the ball from the basket to the box, in hindsight of Sally. Eventually, the question is where will Sally look for the ball she left? As an observer of this scene, using Theory of Mind, we can naturally say that Sally will look for her ball in basket because she isn't aware that Anne moved the ball in the box. Very young children are not able to understand that Sally isn't aware of Anne's action and thus, they are likely to answer that Sally will look in the box. 

Theory of Mind reasoning is a very important skill for \textit{good} interaction and collaboration between agents. Thus, it is reasonable and desirable to aim at endowing robots with such skill to enhance their interactions with humans. This is the motivation of the contribution presented in this chapter. Here, we propose an extension of HATP/EHDA to be able to maintain in a principled way the human beliefs during the planning process, and to tackle estimated false human beliefs that may be detrimental to the task resolution.

Some works already consider ToM during execution, but this isn't always enough. In some situations it is mandatory to consider ToM beforehand while planning in order for the robot to be proactive during execution, and not only reactive. 

We would want the robot to be able to reason and maintain correctly the distinct human beliefs. Despite modeling distinct beliefs, HATP/EHDA doesn't maintain in a principled way, only in a scripted way (domain specific). Here we propose some models and algorithms to integrate some concept of Theory of Mind in the planning process of HATP/EHDA. This way, the robot can estimate more accurately the human's belief to predict their behavior. Moreover, we propose solutions for the robot when to tackle estimated false human beliefs that may impact the task resolution.




%%% SECTION %%%%%%%%%%%%%%%%%%%%%%%%%%%%%%%%%%%%%%%%%%%%%%%%%%%%
\section{Related works}

    This chapter's contribution is related to several topics not mentioned yet. Hence, to better capture the contribution, this section introduces the new topics and relevant related works.  

    %%% SUB-SECTION %%%%%%%%%%%%%%%%%%%%%%%%%%%%%%%%%%%%%%%%%%%%
    \subsection{Epistemic planning}
    Epistemic planning plays a crucial role in human-robot collaboration. Thomas Bolander is one of the main contributors of this field and describe it as follows in \cite{bolander_gentle_2017}. Epistemic Planning is the enrichment of planning with epistemic notions, that is, knowledge and beliefs. The human or robot might have to reason about epistemic aspects such as: Do I know at which post office the parcel is? If not, who would be relevant to ask? Maybe the parcel is a birthday present for my daughter, and I want to ensure that she doesn't get to know, and have to plan my actions accordingly (make sure she doesn't see me with the parcel). The epistemic notions are usually formalized using epistemic logic. Epistemic planning can naturally be en as the combination of automated planning with epistemic logic, relying on ideas, concepts and solutions from both areas. 

    It enables robots to make plans to achieve the required knowledge and to reason about the knowledge and capabilities of other agents, ensuring effective collaboration and coordination in human-robot interactions \cite{belle_epistemic_2023}.

    Bolander et al. proposed the Dynamic Epistemic Logic (DEL) approach \cite{bolander_gentle_2017} and even implemented some Theory of Mind concepts in it. However, despite being more expressive than HATP/EHDA, like the latter, this approach still doesn't maintain the human beliefs in a principled way. They tend to use domain-specific rules given in the probleme specification, usually with conditional action effects. 

    Muise et al. worked on multi-agent epistemic planning using a classical planning approach. Since involving nested beliefs is computationally demanding, their work proposes to convert and encode such problems into classical planning problems. Hence, state-of-the-art classical planning techniques can be used to tackle nested beliefs of multiple agents problems. 

    \subsection{Theory of Mind in HRC}
    Epistemic planning helps to plan the correct sequence of actions to perform to reach a desired knowledge, including a desired world state. However, estimating the current knowledge and beliefs of the different agents is challenging. To do so, we have to consider Theory of Mind concepts, especially perspective shift and the notion of co-presence. Some epistemic planning works already integrate of ToM notions, but the subject is worth being discussed a bit more in details. Indeed, in the HRI field, ToM is used in various domains like navigation, dialogue and like here in task planning. 

    Theory of mind (ToM) refers to the ability to attribute mental states to oneself and others, such as beliefs, desires, and intentions. 
    
    % (v1) Robots endowed with ToM abilities can anticipate and understand the mental states of their human partners, allowing for more effective interaction and decision-making. By inferring their partner's trust and strategy, robots can adjust their own decision models and policies to optimize team performance [1]. Mimicking ToM in robots influences human decision-making behavior and trust, making it more appropriate and conducive to collaboration [2]. Robots implementing ToM are perceived as more socially intelligent and helpful, enhancing the quality of human-robot interactions [3]. Computational theory of mind, based on abstractions of beliefs into higher-level concepts, enables agents to reason and collaborate with humans efficiently, improving decision-making outcomes [4]. However, the field lacks a unified construct and consistent benchmarking, hindering progress in endowing robots with ToM capabilities [5].

    \textbf{TODO: rephrase, less listing of references... Use comment above} 
    Robots endowed with ToM abilities are more effective in proactive robotic assistance and are perceived as more socially intelligent by humans \cite{shvo_proactive_2022}. ToM enables robots to infer human desires, beliefs, and intentions, allowing for natural interaction between robots and humans \cite{yu_robot_2023}. Robots with ToM can anticipate human strategies and incorporate them into their decision models, leading to better team performance \cite{romeo_exploring_2022}. The presence of ToM in robots influences human decision-making behavior and trust, making it more appropriate for human-robot collaboration \cite{schlobach_abstracting_2022}. Computational theory of mind, based on abstractions of beliefs into higher-level concepts, facilitates collaboration on decisions and improves the quality of human decisions \cite{gurney_robots_2022}. However, the lack of a unified construct and consistent benchmarking hinders progress in endowing robots with ToM capabilities.


    
    \subsection{Communication in HRC}

    Communication enables effective interaction between humans and robots, promoting inclusivity and reducing obstacles in human-robot interaction. Communication allows robots to share information about their actions and intentions, enhancing transparency and explainability \cite{mcmillan_human-robot_2023}. It helps in establishing trust and understanding between humans and robots, leading to improved teamwork and performance \cite{verhagen_influence_2022}. Non-verbal gestures and behavior of robots during collaboration can impact the perception of the robot and influence the willingness of humans to cooperate \cite{arntz_collaborating_2022}. Communication also allows robots to assess their own skills and limitations, propose alternatives, and adapt the execution of tasks to the capabilities of the collaborators \cite{ferrari_bidirectional_2022}. Overall, effective communication facilitates mutual knowledge, enables the exchange of information, and allows humans and robots to work together efficiently and successfully.


%%% SECTION %%%%%%%%%%%%%%%%%%%%%%%%%%%%%%%%%%%%%%%%%%%%%%%%%%%%
\section{Maintaining the human beliefs}

    %%% SUB-SECTION %%%%%%%%%%%%%%%%%%%%%%%%%%%%%%%%%%%%%%%%%%%%
    \subsection{Enhanced problem specification}

\begin{figure}
    \centering
    \includegraphics[width=0.5\linewidth]{Chapter3/cooking_task_draw.png}
    \caption{
    Let us consider cooking pasta as a human-robot shared task. 
    The robot has to turn on the stove (\textit{stoveOn}) and clean the counter (\textit{counterClean}), but the latter is not a part of the shared task. The human takes care of fetching the pasta while both agents can add salt into the water (\textit{saltIn}). Before pouring the pasta into the pot the human must know the facts, \textit{stoveOn} and \textit{saltIn}. 
    Unlike \textit{stoveOn}, the facts \textit{saltIn} and \textit{counterClean} are not directly observable. 
    Hence, by acting while the human is away to fetch the pasta, the robot may induce false beliefs which may be detrimental to the shared task (e.g., human adding salt again).
    }
    \label{fig:new_scene}
\end{figure}

We start from the HATP/EHDA probleme specification which is the one described in Chapter~\ref{chap:2}. For our collaborative cooking example, the sets $B$ and $X$ for the collaborative cooking example are the following:
{\small
\begin{align*}
&B           = Entities \cup Places \cup Booleans \cup \{\textsf{nil}\} \\
&\quad Entities    = Agents \cup Objects\\
&\quad Agents      = \{ \textsf{R}, \textsf{H} \} ~~ \backslash\backslash~\textsf{R}:robot,~\textsf{H}:human\\
&\quad Objects     = \{ \textsf{salt}, \textsf{pasta}, \textsf{counter} \}\\
&\quad Places      = \{ \textsf{kitchen}, \textsf{room} \}\\
&\quad Booleans    = \{ \textsf{true},\textsf{false} \}\\
&\\
&X = \{ at(e)~|~ e \in Entities, ~saltIn, ~stoveOn, ~counterClean  \}\\
&\quad \textit{Range}(saltIn ~|~ stoveOn ~|~ counterClean)=Booleans\\
&\quad \textit{Range}(at(\textsf{R} ~|~ \textsf{H} ~|~ \textsf{pasta})) = Places\\
&\quad \textit{Range}(at(\textsf{salt} ~|~ \textsf{counter})) = \{ \textsf{kitchen} \}
\end{align*}
}

Our first contribution starts from here where we augmented the specification by associating each state variable to a location and an observability type. Thus, in addition to the \textit{variable value assignment} function we define two other functions.

First, a \textit{variable observability assignment} function over $X$ is a function $obs$ that maps each $x_i \in X$ into an observability type $t_i \in \{ \texttt{OBS},  \texttt{INF} \}$: $\obsf = \{ (x_1,t_1), \ldots , (x_n,t_n) \}$. Respectively, when $\obsf(x_i) = \texttt{OBS} | \texttt{INF}$ then $x_i$ is said to be \textit{observable} $|$ \textit{inferable} in the state $s_i$.

Then, a \textit{variable location assignment} function over $X$ is a function $\locf$ that maps each $x_i \in X$ into a $l_i \in Places \cup \{ \texttt{nil} \}$: $\locf = \{ (x_1,l_1), ..., (x_n,l_n) \}$. 
$Places \subseteq B$ captures a group of constant symbols such that each member is a predefined area in the environment. 
Agents are always either ``situated'' in a place or moving between two places. 
We consider $x_i$ to be located in every $place \in Places$ if $\locf(x_i)=\texttt{nil}$. 
More details about how the environment should be divided into places will be given shortly when discussing the beliefs updates.

We update the definition of a \textit{state}, or \textit{planning state}, to consider the two functions introduced above s.t. $s_i = \{ \agentstate[R][i], \agentstate[H][i], \obsf[i], \locf[i] \}$. This definition still consider the two agent states s.t. $\agentstate[\varphi][i] = \{ \valf[\varphi][i], \agenda[\varphi][i], \partialplan[\varphi][i] \}$. Now, we keep track in each state of the observability type and location of each state variable, and we can reason on them to update the human beliefs ($\valf[H]$) accordingly. 

We remind that a state $s_i \in S$ contains \textit{false beliefs}, or \textit{belief divergences}, if $\exists x_j \in X, \valf[H][i](x_j) \neq \valf[R][i](x_j)$. 

The initial state of our cooking example can be described as follows. There is no initial belief diverge, the initial partial plans are empty, and both agents but perform the shared cooking task named $CookPasta$. In addition, the robot must clean the kitchen counter once the cooking is done, hence, the task $CleanCounter$ is added in the initial robot agenda after the shared task. The only \textit{non-obversable} facts are the presence of salt in the pot and if the counter is clean. All entities, including the counter, the stove and the salt, are located in the kitchen, only the pasta are in the adjacent room. More precisely, the initial state $s_0$ can be written like this: 

% \vspace{-0.5cm}

{\small
\noindent
\begin{align*}
&s_0 = \{\agentstate[R][0], \agentstate[H][0],~\obsf[0],~\locf[0]\} \\
&\quad \partialplan[R][0] = \partialplan[H][0] = ()\\
&\quad \agenda[R][0] = (CookPasta, CleanCounter)\\
&\quad \agenda[H][0] = (CookPasta)\\
&\quad \valf[R][0] = \valf[H][0] = \{at(\textsf{R}) = at(\textsf{H}) = \textsf{kitchen},~at(\textsf{pasta})=\textsf{room},\\
&\qquad \qquad \qquad \qquad \qquad \qquad \qquad \quad saltIn = \textsf{false},~stoveOn = \textsf{false},~counterClean = \textsf{false}\}\\
% &\quad \obsf[0] = \{ (saltIn,\inferable),(counterClean,\inferable),(x,\observable) ~|~ x \in X \backslash \{saltIn, counterClean\} \}\\
&\quad \obsf[0] = \{ (saltIn,\inferable),~(counterClean,\inferable),\\
&\qquad \qquad \qquad \qquad \qquad \qquad \qquad \qquad \qquad \quad \ (stoveOn,\observable),~(at(e),\observable) ~|~ e \in Entities)\}\\
% &\quad \obsf[0] = \{ (saltIn,\inferable),(counterClean,\inferable),(stoveOn,\observable),(at(e),\observable) ~|~ e \in Entities)\}\\
&\quad \locf[0] = \{ (saltIn,\textsf{kitchen}),~(counterClean,\textsf{kitchen}),~(stoveOn,\textsf{kitchen}), \}\\
&\qquad \qquad \qquad \qquad \qquad \qquad \qquad \qquad \qquad \qquad \qquad \qquad \ (at(e),\valf[0](at(e))) ~|~ e \in Entities \}
\end{align*}

    %%% SUB-SECTION %%%%%%%%%%%%%%%%%%%%%%%%%%%%%%%%%%%%%%%%%%%%
    \subsection{State Transitions and Beliefs Updates}

We now describe how the agents' beliefs are updated when executing a planned action, and thus, how the transition occurs from one state to another. The key principle is that the human agent learns from observing either an action execution or their environment. We first give the 3 assumptions we made in this approach.

\underline{Assumption 1}: We do not consider uncertainties. Thus, agents are either wrong or right about the state of the world but never uncertain. This would be an interesting future work. 

\underline{Assumption 2}: We do not consider cases where the robot's beliefs can diverge. Since the planner is part of the robot, the actual ground truth is unknown. We can only assume that the estimation of the state of the world by the robot is correct and reason on it.

\underline{Assumption 3}: Coming from the two previous assumptions, we assume that the human only makes deterministic moves when not being observed. Hence, regardless of being co-present, the robot's beliefs are always updated with the action's effects.

In our cooking example, to estimate which actions the human is likely to perform first we have to refine the human agenda. We obtain the following two possible actions, associated to two updated agendas:\\
{\small
\begin{equation*}
    \textit{ref}(\agenda[H][0], \valf[H][0]) = \{ a_1,\agenda[][1], a_2,\agenda[][2] \}= \{ (add\_salt(),\agenda[][1]), (move\_to(\textsf{kitchen}),\agenda[][2]) \}
\end{equation*}
}

Thus, Assumption 3 indicates that $\forall x \in X$ we always have,

\begin{equation}
    val_{i+1}(x) = \left\{ 
    \begin{array}{ll}
        w, & \mbox{if} ~ x \leftarrow w \in \textit{eff}(a)   \\ 
        val_i(x), & \mbox{otherwise}
    \end{array}\right.
\end{equation}

The place associated with a state variable can be modified by the action's effect like an agent moving to another room while holding an object. 
Currently, we assume in this work that the observability type of each fact is constant during the task. 
However, the overall approach could be adapted to make the observability types dynamic and this will be discussed later in this Chapter.
So, $\forall x \in X$,

\begin{align}
    &obs_{i+1}(x) = obs_i(x)\\
    &loc_{i+1}(x) = \left\{ 
    \begin{array}{ll}
        l, & \mbox{if} ~ x \leftarrow l \in \textit{eff}(a)\\
        loc_i(x), & \mbox{otherwise}
    \end{array}\right.
\end{align}

The new agenda of each agent ($\agenda[R][i+1], \agenda[H][i+1]$) are created by the HTN refinement algorithm, and thus, they are directly retrieved from the obtained refinement. 
This refinement decomposes abstract tasks in the agenda until the first task is a primitive action. To do so, every applicable method is applied leading to a set of possible actions (and refined task networks).

The new estimated human belief $\valf[H][i+1]$ is the two-step result of our Situation Assessment processes that models the human's real-time sensing and reasoning capabilities about their surroundings.

First, let us define the notions of \textit{co-presence} and \textit{co-location} which will be key to maintaining the evolution of agents' beliefs as planning progresses.

\begin{definition} \label{def:co-pre-loc}
    \textbf{(Co-presence \& Co-location.)} In a state $s_i \in S$, two agents, $\varphi_1$ and $\varphi_2$, are considered to be \textit{co-present} if $val_i(at(\varphi_1)) = val_i(at(\varphi_2))$. This relation is noted $\varphi_1 \curlywedge_i \varphi_2$ in the rest of the paper. Similarly, we say that an agent $\varphi_1$ is \textit{co-located} with a state variable $x \in X$ if $val_i(at(\varphi_1)) = loc_i(x)$, noted $\varphi_1 \curlywedge_i x$.
\end{definition}

Now we can define two SA processes that will maintain the estimated human beliefs.

\begin{definition} \label{def:new_inf}
    \textbf{(Inference Process.)} An agent observes the execution of an action by being either co-present with the acting agent, 
    or by being the acting agent. If so, the agent infers the new values of every state variable present in the action's effects.
\end{definition}

Based on the above definition, the human's beliefs are updated as follows when action $a$ is executed in state $s_i$, 

\begin{equation}
val'^H_{i+1}(x) = \left\{ 
\begin{array}{ll}
    w, & \mbox{if} ~ x \leftarrow w \in \textit{eff}(a) ~ \mbox{and}  \\ 
    & (H = \textit{agt}(a) ~\mbox{or}~ H \curlywedge_i \textit{agt}(a)\\
    & ~\mbox{or}~ H \curlywedge_{i+1} \textit{agt}(a))\\
    val^H_i(x), & \mbox{otherwise}
\end{array}\right.
\end{equation}

To change its \textit{place} in the environment, agents would use a dedicated \textit{``move''} action, such that its effect only updates the agent's location. 

\begin{definition} 
\label{def:new_obs}
    \textbf{(Observation Process.)} An agent observes its surroundings and assesses the exact value of each state variable located in the same place (i.e., each state variable the agent is co-located with).
\end{definition}

After applying the effects of an action to obtain $val_{i+1}$ and the human beliefs $val'^H_{i+1}$ (using the inference process), the observation process is executed. It updates again the estimated human beliefs with the facts currently observable by the human and provides fully updated human beliefs to store in the state $s_{i+1}$, $\forall x \in X$:

\begin{equation}
val^H_{i+1}(x) = \left\{ 
\begin{array}{ll}
val_{i+1}(x), & \mbox{if}~ H \curlywedge_{i+1} x ~\mbox{and}~ \\
    & obs_{i+1}(x) = \texttt{OBS}\\
val'^H_{i+1}(x), & \mbox{otherwise}
\end{array}\right.
\end{equation}

Note that before starting the planning process, the observation process is executed once on the initial state $s_0$. This allows us to potentially correct the estimated human beliefs with the facts the human should initially be able to observe. 

The definition of the set $Places$, i.e. how the environment is divided into different \textit{places}, is guided by the shape of our state transition function. Hence, a $place \in Places$ is an area in the environment such that, when situated in it, agents are aware of each other's activity and they can assess every observable fact located in it. 

Note that unlike in DEL~\cite{KR2021-12}, our knowledge representation is simple and prevents us from expressing agents being \textit{uncertain} about a fact. 
In line with the classical closed-world assumptions, agents either know the truth or have a false belief w.r.t. the ground truth. 
We consider a straightforward scenario in which the human is \textit{``unaware''} of non-observed changes in the environment. 
This results in estimated false human beliefs, helping to detect whether a non-observed robot action can disrupt a seamless collaboration. 

%%% SECTION %%%%%%%%%%%%%%%%%%%%%%%%%%%%%%%%%%%%%%%%%%%%%%%%%%%%
\section{Relevant False human beliefs}

In this section, we explain our procedure to detect \textit{when} a false human belief should be corrected and \textit{how}.


    %%% SUB-SECTION %%%%%%%%%%%%%%%%%%%%%%%%%%%%%%%%%%%%%%%%%%%%
    \subsection{Detection}

The human and the robot carry individual distinct beliefs, while the two can be aligned, or diverging when the human has a false belief. To produce a legal solution plan the robot is fine with such false human beliefs unless they are qualified as \textit{relevant} (Definition~\ref{def:relevant_false_belief}). In such cases, the relevant false belief needs to be tackled.

\begin{definition} \label{def:relevant_false_belief}

A \textbf{relevant false belief} is a false belief that influences the next action(s) the human is likely to perform, either in terms of number, name, parameters, or effects. This can be written as follows:
A state $s_i$ contains a relevant false belief if either (\ref{eq:rel_div_cond_1}) or (\ref{eq:rel_div_cond_2}) is true:

\begin{equation} \label{eq:rel_div_cond_1}
ref(tn^H_i, val^H_i) \neq ref(tn^H_i, val^R_i)
\end{equation}
\begin{equation} \label{eq:rel_div_cond_2}
\{ \gamma(s_i,a) ~|~ \forall a \in ref( tn^H_i, val^H_i ) \} \neq \{ \gamma(s_i,a) ~|~ \forall a \in ref( tn^H_i, val^R_i ) \}
\end{equation}
\end{definition}

We consider that as soon as a false belief has an effect on human actions it should be tackled. An interesting future work could  be to check in a principled way the overall positive and detrimental impacts of this false belief on collaboration. But it is out of the scope of this work.

    %%% SUB-SECTION %%%%%%%%%%%%%%%%%%%%%%%%%%%%%%%%%%%%%%%%%%%%
    \subsection{Resolution with minimal communication}

A state containing a false human belief marked as \textit{relevant} must be handled. 
The first way to do it is by planning communication actions such that the robot communicates only the required facts to the human. This allows to correct false human beliefs that are relevant, but false beliefs that are \textit{``non-relevant''} will remain. 

\subsubsection{Modeling Communication Actions} 
We propose a generic communication action schema ($ca$) in this context. 
An agent $\varphi_i$ can \textit{communicate} an assertion $x=z$ (with $x \in X$ and $z \in$ Range($x$)) \textit{via} the action $ca_{\varphi_i, \varphi_j}(x,z)$ if $val^{\varphi_i}(x) = z$ and $val^{\varphi_j}(x) \neq z$.
The effect of $ca_{\varphi_i, \varphi_j}(x,z)$ corresponds to $val^{\varphi_j}(x) \leftarrow z$. Such actions are considered equally costly and instantaneous.

\subsubsection{Communicate Only the Required Facts}
Definition~\ref{def:relevant_false_belief} indicates if there is at least one diverging state variable in the human beliefs causing adverse effects, but without identifying which one(s).
Hence, we explain a subroutine below with the three steps, describing how we first identify the pertinent state variables to align, and then how the corresponding communication actions are created and inserted into the robot's plan.

\begin{enumerate}
    \item 
    \textit{Store} each state variable whose value differs in the human beliefs from the robot beliefs: $X_{diff} = \{ x ~|~ x\in X, val^H_i(x) \neq val^R_i(x) \}$.

    \item
    \textit{Build}, for each stored state variable $x \in X_{diff}$, a communication action $ca_{R, H}(x,val^R_i(x))$, all stored in a set $\mathit{CA}_{diff}$.

    \item 
    \textit{(Breadth-First Search.)} 
    The \textit{source} is $s_i$. Applying each $ca \in \mathit{CA}_{diff}$ generates a new state by aligning \textit{exactly} one state variable in the human beliefs s.t. $s'_i = \gamma(s_i, ca )$. 
    The search continues until the first state $s'_i$ selected to expand doesn't contain a relevant false belief. The communication actions used from the root until this selected state are \textit{retrieved} in a set $\mathit{CA}$.
\end{enumerate}

Once the above subroutine finishes, the retrieved communication actions in the set $\mathit{CA} = \{ ca_{R, H}(x_1,val^R_i(x_1)),..., ca_{R, H}(x_j,val^R_i(x_j)) \}$ must be inserted in the plan for belief alignment. Thus, Definition~\ref{def:joint-sol-plan} \textbf{TODO: from Chapter, probably best to recall the definition? or not?} is redefined to be sound w.r.t. our approach. An edge can now either be a human action $o^h$ or a robot action $o^r$ with a set of communication action $CA$.
At each step, humans perform \textit{Observation}, while the robot executes each communication action $ca \in \mathit{CA}$, making the human's belief to \textit{update instantaneously}.

The set $\mathit{CA}$ is inserted before the diverging human actions and after the closest state where agents are co-present. 
But it could be interesting to reason with a better plan evaluation system to find the best place to insert this set.

    %%% SUB-SECTION %%%%%%%%%%%%%%%%%%%%%%%%%%%%%%%%%%%%%%%%%%%%
    \subsection{Resolution by delaying non-observed robot action}

So far we relied on communication, but depending on the environment (e.g. noisy), communication can be cognitively demanding. 
Thus, when the relevant false belief is due to a non-observed robot action, we propose to also consider implicit communication by postponing the pertinent robot action until the human is estimated to be observing its execution. 
This prevents false beliefs from even occurring.

First, a branch using communication is explored and the state variables concerned by the relevant false beliefs are retrieved (through all $ca \in CA$).
Then we check if the divergence is produced by a non-observed action. For now, it is done by checking if the relevant divergence concerns only one inferable state variable and if it was not present in the initial state.   
After, we identify which action creates the divergence by sequential regressing the current branch/trace. Hence, we can identify when the relevant divergence appears and which action should be delayed.
Once identified, we create another branch in the plan just before the identified action. In this new branch, {\sc delay} actions are inserted in the robot's plan until the human is co-present. When the human is co-present again, the identified action is inserted and observed by the human. Then the nominal planning process is resumed.  

%%% SECTION %%%%%%%%%%%%%%%%%%%%%%%%%%%%%%%%%%%%%%%%%%%%%%%%%%%%
\section{Result}

Referring to the related work section, we are not aware of an implemented planning system that can be used as a baseline. Hence, we use the HATP/EHDA solver to help present our approach's results on three \textit{novel} planning domains.

\subsubsection{Cooking Pasta Domain}
The running example corresponds to a specific problem in this domain. In fact, agents and pasta can initially either be in the kitchen or in the adjacent room, the stove might be on or off and there might be salt or not in the water.  
In the results, we will focus on the following three state variables from $X$. Both $stoveOn$ ($\texttt{OBS}$) and $saltIn$ ($\texttt{INF}$) are relevant to the human, unlike $Clean$ ($\texttt{INF}$) which only concerns the robot. 

\subsubsection{Preparing Box Domain}
A box with a sticker on it and filled with a fixed number of balls is considered prepared and needs to be sent. Both agents can \textit{fill} the box with balls from a bucket, while only the robot can \textit{paste} a sticker and only the human can \textit{send} the box. The bucket can run out of balls, so when one ball is left, the human \textit{moves} to another room to \textit{grab} more balls and \textit{refill} it. 
The number of balls in the box is \textit{inferable}, while all other variables are {\em observable}. 
In the following, three boxes have been considered.

\subsubsection{Car Maintenance Domain}
The washer fluid ($\texttt{OBS}$) and engine oil ($\texttt{INF}$) levels have to be \textit{full} before \textit{storing} the oil gallon in the cabinet ($\texttt{INF}$). 
Only the robot can \textit{refill} both the tanks and store the gallon while situated at \textit{Front} of the car. 
\textit{Front-left} and \textit{Front-right} headlights have to be \textit{checked} and a light-bulb has to be \textit{replaced} at \textit{Rear}. 
Only the human can check and replace lights, and they can start with either of these two tasks.
Both agents start at \textit{Front}.
The car's hood needs to be \textit{closed} by the human at last.

\subsection{Qualitative Analysis}

\begin{figure}[t!]
    \centering
    \includegraphics[width=0.65\linewidth]{Chapter3/plans.pdf}
    \caption{
    Plan obtained for the cooking scenario. 3 branches. Left: The human starts by adding salt. The only false belief is about \textit{``counterClean''} which is not relevant for the human agent, hence no comm is added. Middle: While the human is away the robot turns on the stove and adds salt, creating 2 false beliefs. 
    Once back, we estimate that the human agent
    will be able to assess the observable fact \textit{``stoveOn''} but not \textit{``saltIn''}. Since the human agent might add salt again due to this false belief, it is relevant and fixed with a communication action. Right: The relevant false belief about \textit{``saltIn''} is avoided by delaying the robot's action until the human is co-present.
    }
    \label{fig:cooking_plan}
\end{figure}

Considering the cooking domain, we discuss in detail the plans obtained with our approach to a problem corresponding to the description given in the introduction. 
I.e., there is no initial human false belief, agents both start in the kitchen, the pasta is in the adjacent room, the stove is off, and there is no salt in the water. The resulting plans are shown in Fig.~\ref{fig:cooking_plan} and their detailed presentation explains how the approach works in practice. 
Since human is uncontrollable and has different possible actions, the plan branches and the robot's actions are different in each case. 

In (\textit{left}) the human first adds salt and then the robot turns on the stove. In both cases, thanks to the inference process, we estimate that the human will be aware of both facts about the salt (\textit{acting}) and the stove (\textit{co-present}). Then while the human is away to fetch the pasta, the robot cleans the counter and since the human isn't co-present their beliefs aren't updated, containing now a false belief. Once back, since \textit{counterClean} is not \textit{observable} the observation process does nothing and the false belief remains. However, this false belief doesn't affect human actions (non-relevant), hence, there is no need to align human beliefs.

In (\textit{middle} and \textit{right}) the human first fetches the pasta by leaving the kitchen. Let's first focus on the (\textit{middle}) trace. The robot turns on the stove and adds salt while the human is away, creating two false beliefs. When returning to the kitchen, the observation process updates the human beliefs with the observable facts located in the kitchen. This fixes the false belief about \textit{stoveOn}. The robot then cleans the counter, observed by the human. 
However, without communication, the human's next action will be either ``add salt'' or ``ask the robot'', but considering the ground truth the human could directly pour the pasta. Hence, the false belief on \textit{saltIn} is relevant and has to be corrected. To do so a communication is inserted in the robot's plan and a ``delay'' branch is created (\textit{right}). 
In this delaying branch, the robot delays the add salt action until the human is co-present in order to make it observed (inference process) by the agent. 
In addition to this implicit communication, like in (\textit{middle}), the human assesses that the stove is on and hence can directly pour the pasta. 
\begin{table}[t]
    \centering
    \vspace{0.1cm}
    \caption
    {
    Success and communication ratio of different approaches. 
    }
    \label{tab:q_results}
    \begin{tabular}{@{}c|c c|| c| c@{}}
    % \begin{tabular}{c|c c|| c}
        \multirow{2}{*}{\textbf{Domain}} & \multicolumn{2}{c||}{\textbf{HATP/EHDA}} & \multicolumn{1}{c|}{\textbf{Only Comm}} & \multicolumn{1}{c}{\textbf{With Delay}}
        \\
        & \multicolumn{1}{c}{\textit{S}} & \multicolumn{1}{c||}{\textit{S I.Div.B.}} & \multicolumn{1}{c|}{\textit{Comm}} & \multicolumn{1}{c}{\textit{Comm}} 
        \\ \cline{1-5}
        \textit{Cooking}    &   18.6\%  &  6.9\%    & 69.5\% & 65.2\%\\
        \textit{Box}        &   25.0\%  & 14.3\%    & 79.7\% & 75.0\%\\
        \textit{Car}        &   12.5\%  & 0.0\%     & 68.8\% & 64.1\%\\
        \hline
        \textbf{Average}    &   18.7\%  & 7.1\%     & 72.6\% & 68.1\%\\
    \end{tabular}
\end{table}

\subsection{Experimental Results and Analysis}

In each domain, the actions and tasks remain the same. So here, a problem is defined by a starting agent ($R$ or $H$) and a pair of initial beliefs ($val^R_0, val^H_ 0$).
Initial ground truth ($val_0 \Leftrightarrow val^R_0$) is defined by setting each state variable to an initial value. But, 5 selected state variables can be set to 2 possible values instead of 1. Among these selected ones, 3 can diverge in human belief. This generates 256 pairs of initial beliefs where 12.5\% of them include initially aligned beliefs. Then, considering the starting agent, we obtain 512 problems for each domain. 
Each of the 1536 generated problems has been solved by HATP/EHDA, by \textit{our approach} using first \textit{only communication} and then using also \textit{delay}.
The obtained quantitative results appear in TABLE~\ref{tab:q_results}.
 
The overall success rate ($S$) and the one for initially diverging beliefs ($S I.Div.B.$) are shown for the HATP/EHDA solver. As expected, this solver always finds legal plans when dealing with initially aligned beliefs, but the low value of $S I.Div.B.$ reflects how poorly it handles belief divergences without specifically designed action models.
Our approach always finds legal plans so we omitted its success rates in the last two columns, and we can say that it solves a broader class of problems.

Furthermore, considering the initially diverging beliefs and the divergences created along the planning process, more than $87.5\%$ of all problems involve belief divergences. 
However, when using only verbal communication, only $72.6\%$ of the generated plans include communication actions.
This means that \textit{our approach} communicates only when necessary, and not systematically. 
The amount of communication is even reduced to $68.1\%$ when delaying actions. In the latter case, only delayed branches that do not imply the human to wait are kept. 

%%% SECTION %%%%%%%%%%%%%%%%%%%%%%%%%%%%%%%%%%%%%%%%%%%%%%%%%%%%
\section{Discussion and Limitations}


obs type constant: For instance, when an agent places an object in an opaque box, the observable object would no longer be observable despite being in the same place as the agent. Places can be symbolic, thus, with the right modeling one can model the disappearance of an object when being placed in a drawer. Making the obs type dynamic requires more thinking on how their are updated, yet, this would only [ref def] has to be modified and the whole scheme would continue to work.

The underlying scheme allows just a single agent to execute a \textit{``real''} action at a time. 
However, a post-process can allow the execution of actions concurrently~\cite{CrosbyJR14}, however, note that the domain modeler has modeled $\mathcal{P}_{rh}$ as a sequential joint task. 
Parallelism is not considered in the current modeling and planning process, which limits the potential for concurrent executions. However, we are working on extending the framework to enable systematic planning with concurrent actions, aligning with~\cite{ShekharB20}.

We believe our modeling-level SA proposals could fit in any other planning approach framing multi-party systems having one controllable agent while can only hypothesize remaining agents' behaviors (e.g., human-centered AI).

Agents' SA models cannot simply refute a false belief, they can only assess new true facts to correct them.
E.g., assume the human \textit{wrongly} believes that the pasta is in \textsf{kitchen}. The SA does not help refute this when the agent is in \textsf{kitchen}
because appropriate knowledge reasoning w.r.t. \textit{NotAt(Pasta)} in \textsf{kitchen} is not taken into account.  
However, such issues do not affect the completeness and, if necessary, our approach \textit{tackles} such cases as relevant false beliefs.

We have planned a user study for the future to conform our framework with reality and validate the approach.

We discussed earlier that DEL knowledge representation is more expressive and flexible, and can handle uncertainty. However, it requires an augmented action schema to accurately maintain each agent's beliefs.
Think of a specification for \textit{``move''} action manually listing all the environmental facts to be observed by an agent for managing their beliefs. In our case, it is implicitly maintained within a state.

We can consider running a set of rules (e.g., \textit{graph-based ontology}) to bring new interesting facts in the state based on a set of known facts. We believe that this aspect opens up new possibilities in the future to integrate human-aware collaborative planning and ontology.

%%% SECTION %%%%%%%%%%%%%%%%%%%%%%%%%%%%%%%%%%%%%%%%%%%%%%%%%%%%
\section{Epistemic Extension}

The planning process proposed in this contribution is part of the epistemic planning field. However, the uncertainties are limited and focused on the human decisions, in a simplistic manner compared to ``classical'' epistemic planner. Indeed, state-of-the-art epistemic planners consider epistemic states rather than ``simple'' beliefs and thus are able to keep track of several possible worlds. More precisely, [cite bolander] this work using the DEL approach are able to keep track and reason on several possible worlds for each agent, and can even manage distinguishable worlds from others. Hence, the planner is able to anticipate that an agent can believe in three different possible worlds and tell that at execution time the agent will be able to distinguish if they are in world 1 or not, but they will not be able to tell distinguish between world 2 and 3 (based on observable facts).
Such approaches currently heavily rely on conditional action effects and scripting. That is, when an agent enters a room with an action \textit{move}, the \textit{move} conditional effects will update accordingly the agent's possible worlds (the agent sees a box, notice that someone isn't in the room anymore, and so on...). These situation assessment rules must be encoded into the action model used, and more precisely, in the action's effect which isn't convenient and is domain specific.
Our contribution proposes generalized situation assessment rules to maintain the beliefs but at the price to have simplified epistemic states.

In some preliminary work we explored an extension of our approach to handle such classical epistemic representation. 

* Describe first findings and models * 

* Some results ? *

* Discuss limitations: the current design is computationally very expensive and doesn't scale. Work on this aspect are needed and in progress to continue investigating this extension * 


%%% SECTION %%%%%%%%%%%%%%%%%%%%%%%%%%%%%%%%%%%%%%%%%%%%%%%%%%%%
\section{Conclusion}

We propose an extension to a Human-Aware Task Planner called HATP/EHDA. 
The planner plans and implicitly coordinates the robot's actions with all estimated possible human (uncontrollable) behaviors that are then emulated to generate a new state.
Our extension and contribution are, first, to integrate a \textit{Situation Assessment} based reasoning system in the planner. This allows for maintaining distinct agents' beliefs based on what they can/should observe.
Compared to existing epistemic planners, this simplifies the action descriptions by focusing on their effects on the world, and not how they influence each agent's beliefs.
In addition, we propose to detect false human beliefs and tackle only the necessary ones in a principled way. First, we propose minimal and proactive explicit communication. Second, when pertinent, 
we propose an implicit communication by postponing the non-observed robot action until the human is co-present to observe it.  

The relevance of false belief, when to optimally communicate and parallelization are interesting future works, and we aim at conducting a user study to validate the benefits of the proactive robot behavior that our approach permits. 
\ifdefined\included
\else
\setcounter{chapter}{3} %% Numéro du chapitre précédent ;)
\dominitoc
\faketableofcontents
\fi

\chapter{A Task planner making a robot compliant to human online decisions and preferences}
\chaptermark{A Task planner making a robot compliant to human online decisions and preferences}
\label{chap:4}
\minitoc


\section{Introduction}

\section{Related works}
paper Sonia UHTP



\section{Model of Execution}

\subsection{Based on joint action literature}

\subsection{Model description}

\subsection{Model utility}

\subsection{From Article}

\begin{figure}
    \centering
    \includegraphics[width=\linewidth]{images/Chapter4/Execution_Automaton.drawio.pdf}
    \caption{
    The Model of Execution, in the form of an automaton and here simplified, captures the latitude of uncontrollable humans in their actions and guides our task planning approach.
    In this paradigm, the two agents can act concurrently but one is always compliant with the other's decision to act.
    Here, the human is always free to decide whether to start acting first, or after the robot, or not to act at all.
    To be compliant, the robot attempts to identify human decisions using perception and situation assessment as well as possible collaborative human signaling acts (e.g., gestures or speech).
    }
    \label{fig:model_of_execution}
\end{figure}

Our task planning approach uses a model of execution to improve the fluency and amenability of HRC. 
This model is in the form of an execution controller as shown in Figure~\ref{fig:model_of_execution}, and is based on several key notions and mechanisms borrowed from studies on joint actions~\cite{Sebanz-2016,kourtis2014attention}, and adapted to Human-Robot Joint Action~\cite{clodic-2017,curioni-2019}.
The key idea is that co-acting agents co-represent the shared task context and integrate task components of their co-actors into their own task representation~\cite{Schmitz-2017, Yamaguchi-19}. Also, coordination and role distribution rely strongly on reciprocal information flow, e.g., social signals~\cite{curioni-2019}, prediction of other's next action~\cite{luke-2018}.

Our proposed execution model is implemented on a robot that co-acts with a human, integrating explicit representation and exploration of the task representations for the robot and for the human. 
It also identifies precisely how reciprocal information flow is used in task execution (detecting and interpreting human actions, signals produced by the robot while acting, and also when the robot waits for human actions or their signals).

Another essential question is the criteria for choosing the next action, or more globally, how to share the load between the two co-actors. The choice depends on the context and actors' preferences~\cite{Gombolay-2015, Strachan-2020,Curioni-2022}. 
Concerning the case when one actor is a robot, we think it is important to provide a standard default behavior of the robot where the robot does its best to reduce human load but still leaves full latitude to act whenever humans want. 
Our scheme provides this ability and also allows humans to inform about their preferences at any moment.

Consider an example to clarify the execution automaton. 
Assume a human and a robot have to pick up two blocks, \textit{A} and \textit{B}, that both can reach. 
They can pick it up both at the same time unless they try to pick up the same block, which causes conflicts between their actions. 
As a result, despite being executable in parallel, the actions are interdependent, and in order to avoid conflicts, one agent must be compliant with the other. 
However, if we consider a third block \textit{C} that only the robot can reach, it can always pick up this block without any risk of conflicts with the human's choice. 

In a state, a human decision can result in one of three outcomes.
First, the human can choose to act first (\textit{left~subtree}).
If the robot's best action is not in conflict with the human action (e.g., \textit{pick~C}), the robot can safely perform this action concurrently with the human operator (\textit{branch~3}).
However, if the robot's best action is either \textit{pick~A} or \textit{pick~B}, the human action must be identified first with a subroutine in order to be compliant with it.
If this subroutine is successful the robot can perform any action which is congruent with the identified human action (\textit{branch~1}). 
This includes the robot's choice to be \textit{passive} and let the human act alone. 
However, if the robot is unable to identify the human action, it must remain passive in order to avoid potential conflicts (\textit{branch~2}). 
Then, the human can either decide to be \textit{passive} or to act after the robot (\textit{right~subtree}). 
In both cases, the human is \textit{passive} at the beginning, making the robot to start performing alone a feasible action. 
While the robot is acting, the human is free to remain \textit{passive} until the next step (\textit{branch~5}), or to choose a congruent action to act concurrently (\textit{branch~4}). 
As a result, the human can always choose to 1) act first, 2) act after the robot, or 3) not act at all. 
The robot will always be compliant with these online human decisions.

When both agents finish their actions, the step is considered as \textit{``over''}. 
Then, another subroutine assesses the new world state ($s_{i+1}$), which is the result of the concurrent actions being executed in the state $s_i$, before repeating the whole process until the task is solved.

Note that if both agents are passive (the human decides to be passive when the robot cannot act) then the step is repeated. 

\section{Problem specification + solution}

\subsection{Problem}

Belief divergences are out of the scope of this particular work. Hence, for simplicity reasons, we consider the two beliefs (robot and estimated human ones) as always aligned, and they are represented as a unique world state. However, we are convinced that this work could be adapted easily to consider the two distinct beliefs.

The problem is specified as follows. One initial world state, described using state variables. Distinct human and robot action models described with HTNs. Distinct human and robot initial agendas. 

\subsection{Solution}

A Planning state, referred as p-state, corresponds to a state in which the planning problem is while progressing toward a solution. A p-state contains the current world state (aligned beliefs of the agents) and the human and robot agendas. Thus, keep in mind that p-states are very different from world states.
The initial p-state is formed using the initial human-robot agendas and the initial world state given in the problem specification. P-states are connected with each other through concurrent pairs of human-robot actions. A goal p-state as the characterized by a world state satisfying given goal conditions and by empty agendas.
The exploration produces a directed graph from the initial p-state to several goal p-states through sequences of concurrent action pairs. Thus, any path from the root to a leaf is a possible plan. Once the exploration done, directed solution graph computed, another process extract the optimal robot policy from the graph. 

It's design choice to do consider explicit action costs and perform an exhaustive offline search to produce this solution graph to solve a problem. Since the policy generation is very quick it allows generating and update the robot policy online according to human feedbacks. 

We keep track of the p-state to explore, this set being initialized with the initial p-state. Then, until the set is empty we select one and explore it.
First, from this selected p-state, every possible concurrent human-robot action pairs are computed considering both agendas, the world state and reasoning on the compability of the actions in terms of preconditions and effects. This process requires several sub-steps and is detailed later. Thus, we obtain several action pairs leading to the same amount of new p-states (with updated world state and agendas).
Second, we check if any of the newly created p-state are similar to any existing p-state. If so, we can "merge" them to avoid redundant computations. To do so, we basically keep track of the unique p-state already checked and for each new p-state we check if it is similar to one of the already checked one. 
If not, the new p-state is added to the set of already checked p-states. 
If a similar one is found, the new p-state is deleted and the action pair leading to it is connected to the existing p-state instead.
Eventually, the remaining new p-states are added to the set of p-states to explore. 
When the set of p-states to explore is empty then the exploration is over and the graph obtained corresponds to the "solution graph" where any path from the initial p-state to a leaf corresponds to a possible execution trace / plan.

Now, the robot's policy must be determined. We want to preserve the latitude of choice the human have at execution. Hence, the policy must indicate in every p-state which action the robot should concurrently perform according to each action the human is likely to perform. These best concurrent robot actions are determined by aiming to optimally satisfy an estimation of the human preferences regarding the task. Eventually, at execution, the human is free to perform any of the explored action and the robot will accordingly perform an optimal concurrent action to both solve the task and satisfy their estimated preferences.  

\section{Exploration}

\subsection{Compute next agent actions}

\subsection{compliant pairs}

\subsection{graph, merge state planning state}


\section{Policy generation}
Light, 

\subsection{Human preferences}
estimations, format, Discussion(often inaccurate, hence our Approach)

\subsection{process}
propagation + merge + policy format 

\section{Results}
simulation of execution, without durative action

\subsection{concept of aligned-adversarial pairs of prefs/estimations}

\subsection{results}

\section{Discussion and Limitations}
\section{Conclusion}





% \ifdefined\included
\else
\setcounter{chapter}{4} 
\dominitoc
\faketableofcontents
\fi

\chapter{User Study to evaluate an integrated plan and execution scheme in simulation}
\chaptermark{Integration and Evaluation of a collaborative scheme}
\label{chap:5}
\minitoc

\section{Introduction}

why simulation: HRI rebuttal: we rely on a reactive execution, real life robot are slow and not very reactive thus may bias our results

Thus, we developed an interactive simulator running robot policies generated as explained the Chapter~\ref{chap:4}. Then, we conducted a user study using this simulator to evaluate our approach.
First, the simulator is described (planning and exec part). Then, the Procedure of the Study is presented. Eventually, the obtained results are presented and then discussed.

The goal is to validate our approach presented in previous chapter and justify the following. We believe that the robot should plan their actions with a compromise of optimizing the task efficiency, maximizing social criteria and satisfying human preferences while collaborating with humans. Despite being efficient we wouldn't like a robot handing us a sandwich while mopping the floor. Numerous planning works solve very well problems with maximal efficiency guarantees. However, satisfying those social criteria is much harden since they cannot be accurately quantified. Especially for human preferences which can be fluctuating a lot depending on their mood and context. That is why we believe, in addition to planning the best course of action using all these criteria, it is relevant or even mandatory for the robot to adapt and be compliant to human online decisions and actions. 

\textbf{TODO: This can almost be put in the previous chapter as motivation}

\section{Related work}
of plan+execution, simulator

\section{Interactive Simulator}
planning is same

execution is based on model of execution and mock components

human collaborates in real time with robot in blockwords task

describe simulator, mouse, robot and human capabilities, goal shown, prompt

Synchronization, internal communication through simulated visual signal, explicit reaction time, ID phase currently always successful a success rate can be given. 


***********

In this section we provide a few functional and technical details about the developed interactive simulator used for the User Study, its overall structure is shown on fig~\ref{fig:iteractive_simulator}.

\begin{figure}
    \includegraphics[width=\textwidth]{images/Chapter5/simulator.png}
    \caption{Interactive Simulator Overview}
    \label{fig:iteractive_simulator}
\end{figure}

First, it is using Gazebo to simulate the scene (table, cubes, agents). 
The robot is a Tiago robot from PAL Robotics. We used a MoveIt controller to plan and execute arm motions. We used a publicly available component to control the gaze of the robot. The head behavior is ad hoc and designed by us.
The human agent is simulated only with its hand which is moved with a handmade dedicated controller. 
For simplicity reasons physics of objects isn't simulated, thus, each agent interact with the scene by statically attaching/detaching object respectively to the robot's gripper or the human hand.
A higher level component called the simulator controller is in charge of translating primitive actions to low level commands and calling the previously presented controllers to make the agent perform actions. Since it manages agent's action, it is also in charge of sending simulated visual signals and indicating when a step is over. Indeed, the agents synchronize themselves using simulated visual signals, and not internal signals. That is, when the human starts performing an action, a brief motion planning phase occur before effectively moving the hand, the motion visual signal is sent to the robot only at this moment. Moreover, to simulate some real perception aspect, a voluntary reaction time is introduced in the robot. Thus, the visual signal can be treated only after this reaction time is passed. A step starts when one agent starts acting (the human or the robot) and it is over when both agents are inactive. Hence, once one agent started to act the other is free to perform an action concurrently and the step will be over only when both are done. If the other agent doesn't act it will be considered as passive/inactive and the step will be over as soon as the first agent finishes.

Then we can find the respective agents' components. The robot component is an implemented version of the model of execution. Its purpose is to supervise the policy execution of the robot while running the automaton described in the model of execution. In practice, the robot starts by indicating when the step started with a sound signal. Then, it waits to receive the visual human signals for a defined amount of time. Without any visual signal the human is considered as passive. The human can either start acting (how will be described just after), which will eventually send a visual signal to the robot, or the human can make an explicit hand gesture to indicate to the robot that they will be passive (this doesn't force the human to be passive for the whole step). Note that even if the robot received the human visual signal of the human starting to act, the actual action being performed isn't yet identified, it only means that the human started to act. The robot must perform a dedicated identification process (ID Phase/Process) to identify the human action. We assume that the hand gesture that the human can make is always successfully identified instantly (without ID Process). Once either the signal received or the time-out reached, if the best robot action doesn't depend on the human decision then the robot can directly start executing this action. The robot action choice is sent to the simulator controller that will start executing the action. If the best robot action depends on the human decision then the human decision must be identified and the ID Process is run. Only after the robot can perform the best action indicated by the policy which is compliant with the identified human action. If the ID Process fails then the robot remains passive, this case is already part of the policy.
Eventually the robot waits for the step over signal from the simulator controller and run the Assessment Process. The latter identify what happened during the step (which action the human performed, even if ID wasn't necessary) and in which state we are to be ready to execute the next step and progress in the policy and toward the goal. 

The human action execution consists in the following. First, the gazebo simulator is made as the main Human-Machine Interface (HMI). Through a handmade plugin one can mouse click in the gazebo simulator window. The click position is sent on a ROS Topic and treated in a dedicated node to potentially trigger a human action. For instance, clicking on a cube starts a "picking" action with the corresponding cube as parameter, clicking on the stack area starts a "place" action, clicking on the table while holding a cube starts a "hold" action, and clicking on the hand make an explicit hand gesture to signal the robot the human desire to be passive. After clicking the hand once, the participant can click another time to indicate the robot full passivity until further notice. The robot will consider the human as unavailable or even not present and will not wait for them. 
For this to be possible we did the following. First, another gazebo plugin has been written to lock the GUI camera as a First Person View and avoid the participant to accidentally move the camera.
Then, at every step, the robot component sends to the component receiving the coordinates of the click the currently valid human actions that the participant can perform. Thus, the participant cannot try to place a cube without holding it. Eventually, each human action is mapped to a clickable zone on the gazebo window. Hence, when the click coordinates are received, we check if it is inside one of the designed zone and is the action associated to the zone is currently valid. If so, then the action decision is sent to the simulator controller to be executed. 
For clarification purpose, when clicking on a cube the robot doesn't immediately know that the human is picking a cube. First, it will know that the human started to act only after receiving the visual signal (after the defined reaction time). And the robot must perform an additional perception phase (ID Process) to identify which action the human performed. 

Additionally, the stacking goal pattern is shown inside the simulator and a small prompt overlay is present to simulate a potential screen with which the robot can communicate its intention to the human during the collaboration. 

Thanks to all these processes, the participant is able to intuitively perform actions using mouse control and can collaborate with the robot to solve the task.

An integrated tutorial has been made to familiarize the participant with the mouse control and the relevant concepts used in the collaboration (notion of step, able to be passive). 

\section{Detailed Model of Execution}

Explicit the model of execution and the signals exchanged.

\begin{figure}
    \centering
    \includegraphics[width=\linewidth]{images/Chapter5/detailed_automaton.png}
\end{figure}

\section{Study protocol}

objective, participants, material, experiment design, procedure, measures

This section describes the objectives and protocol of this study.  

\begin{figure}
    \centering
    \includegraphics[width=0.8\textwidth]{images/Chapter5/UserStudyProcedure.png}
    \caption{A scenario of the User Study Protocol. Each participant goes through 6 scenarios and answer 6 questionnaires to evaluate each different robot behavior}
    \label{fig:user_study_protocol}
\end{figure}




Through this study we want to demonstrate the benefits of using the model of execution described in the previous chapter~\ref{chap:4} in a collaborative context. We believe this model of execution is pertinent to be taken into account when executing and supervising a robot's plan. For the same reasons we based the policy generation of this model and aim to justify our choice and validate our approach.

In this study, each participant is made to collaborate six times with a simulated robot to achieve a shared task, each time is referred to as a scenario. The robot exhibits a different behavior in each scenario. After each scenario, the participant evaluate the robot's behavior through the PeRDITA questionnaire \cite{devin_evaluating_2018}.

Beforehand, every participant answers a few general/demographic questions and is familiarized with the simulator functionalities through an integrated tutorial. Only then they start the six consecutive collaborative scenarios, answering every time a questionnaire to describe the interaction. Eventually, every participant is asked to share their general feelings and impressions about the overall interaction with the simulated robot, and they are asked to tell which scenario they preferred the most and the least.

\begin{figure}
    \centering
    \includegraphics[width=0.8\linewidth]{images/Chapter5/task_description_study.png}
    \caption{Description of the shared task to achieve in the study.}
    \label{fig:task_description_study}
\end{figure}

We now provide details about the task, the scenarios and how the different robot behavior are generated. 
The shared goal, which is stacking the cubes to match the given pattern, remains the same in all scenarios. The cube disposition on the table also doesn't change either. The task description is depicted in fig~\ref{fig:task_description_study}. 

To progress in the task, the agents can perform three different primitive actions which are the following: \textit{pick} a cube, \textit{place} a cube in the stack, or, \textit{drop} a cube back on the table.
These actions have a few preconditions, more or less intuitive, that are communicated and experienced by the participant during the integrated tutorial.
First, one can \textit{place} a cube if they hold the cube, and if the targeted location is free and supported. That is, the cubes directly below the targeted location must be placed before being able to place a cube in the targeted location.
Secondly, one can only \textit{pick} a cube from their respective reachable zones of the table, i.e., Human and Center zones for the human, and Robot and Center zones for the robot. Also, one can only pick a cube if it can be placed immediately. Thus, one cannot pick a cube ``in advance'' and has to wait to its placement condition to be true before picking it up. For instance, both pick bars can only be picked up after the yellow and red cubes have been placed. This rule helps to create interaction conflicts serving the purpose of this study. Moreover, although the participants find this not intuitive they get used to it really fast and this feeling seems to be significantly reduced along the experiment.
Third, one can \textit{drop} a cube back on the table only if they hold it and if it cannot be placed.

For each scenario the participant is given instructions regarding how to solve the task. The participants are asked to consider these instructions as their own choice and preferences regarding the task resolution, and thus, to act accordingly while collaborating. The instructions at each scenario are one of the two following.
On the first hand, the participant shall act in a way to finish the task as soon as possible. Here, it consists in trying to perform as many actions in parallel as possible to progress faster. These preferences are latter referred to as Task End Early (TEE).
On the other hand, the participant shall act in a way to be freed as soon as possible. That is, they should finish their mandatory part of the task as soon as possible, so they could leave and let the robot finish alone. Here it consists in placing the pink bar from the Human zone as soon as possible. These preferences are latter referred to as Human Free Early (HFE).
On its side, the robot doesn't directly have access to these instructions/preferences, they are only estimated. Hence, for each scenario, the robot is given a more or less accurate estimation of the human preferences that are communicated to the participant. Note that the participants aren't aware that the robot has an estimation of their preferences, neither that this estimation can be inaccurate.
This way, we created three scenarios with different pairs of human preferences and associated estimation. In the first pair, the human shall finish the task early and the robot has a correct estimation, i.e., the robot's policy is helping the human to finish the collaborative task early. In the second pair, the human preferences remain the same, but the robot estimation is incorrect. The robot is trying, mistakenly, to minimize the human effort. As a consequence, the robot tends to pick cubes that the human could pick, preventing the human from acting and making the task completion longer. In the third pair, the human shall free themselves early, but the robot estimation is again erroneous. The robot will try to finish the task early while its priority is to place the first pink bar, which is conflicting with the given human preferences.

Additionally, in each scenario, the robot follow one of the two following execution regimes:
\begin{itemize}
    \item \textbf{Robot-First (RF)}: the robot always initiates actions first, and the participant take action afterwards.
    \item \textbf{Human-First (HF)}: the robot always lets the participant take the initiative, and then acts.
\end{itemize}
The \textit{Human-First} execution regime corresponds to the Model of Execution described in the previous chapter. At each step, the robot waits for the human's decision and will execute the best action that complies with it. The human always start acting first and the robot follows. On the other hand, the \textit{Robot-First} regime corresponds to a naive and straightforward policy execution where, at each step, the robot directly starts executing the overall best robot action given by the policy. The robot always starts acting forcing the human to comply. The \textit{Robot-First} regime serves as a baseline to evaluate the proposed \textit{Human-First} regime, described by our Model of Execution and used in the policy generation.
Eventually, we associate each of the three previous pairs of preferences and estimation with one of the two different execution regime. As a result, we obtain six different scenarios with six different robot behaviors named in table~\ref{tab:scenario_names}.

\begin{table}
    \caption{Name of the six scenarios. 
    Columns represent the preferences/estimation pairs and the rows correspond to the execution regimes.}
    % \vspace{-15pt}
    \begin{center}
    \begin{tabular}{c|c|c|c|}
        \cline{2-4}
                                                & Pair A        & Pair B            & Pair C\\
                                                & TEE: correct  & TEE: incorrect    & HFE: incorrect\\
        \hline
        \multicolumn{1}{|c|}{Human-First}       & S1            & S3                & S5\\
        \hline
        \multicolumn{1}{|c|}{Robot-First}       & S2            & S4                & S6\\
        \hline
    \end{tabular}
    \end{center}
    \label{tab:scenario_names}
\end{table}

Note that our goal is to evaluate and compare the different robot behaviors. However, at the beginning, the participants don't have any references to compare with which can influence their answers in the very first scenarios. One solution is to ask the participants to answer all six questionnaires at the end, after being familiar with the six scenarios. We consider that this option demands a too heavy mental workload to recall accurately each specific scenario, and may bias the answers. As a consequence, we decided to ask the participants to answer the questionnaire after each scenario as a draft. Along the experiment, they can rectify their answers to match more accurately their feelings. At the end, using the drafts, they share their final answers for each scenario. We believe this process allow to more accurately gather the feelings of the participants. Moreover, the ordering in which the participants encounter the scenarios is uniformly randomized to prevent any order effect. 

\textbf{TODO: Give more details about the PeRDITA questionnaire (questions, goal, etc..)}

On top of the answered questionnaires, for each scenario, the interactive simulator produces logs from which we extract several metrics and an overall timeline of the execution. The timeline depicts the activities and actions of each agent along the task progression. The subjective measures done through the questionnaire are complemented with the objective metrics extracted such as the duration to complete the task, the number of human action, the total duration of human inactivity, and more. 



\section{Study results}

% ns  P > 0.05
% *   P ≤ 0.05
% **  P ≤ 0.01
% *** P ≤ 0.001

In this section, we share the main results obtained through the answered questionnaires and the metrics extracted from the logs.

\subsection{Assumptions}

For analysis we rely on ANOVA tests that requires data to follow a normal distribution. 

Here the data, most of the metrics, are close to following a normal distribution (checked using Kolmogorov-Smirnov, Shapiro-Wilk and Anderson-Darling tests). Thus, parametrics tests can be applied, and we used either paired t-tests or ANOVA with repeated measures.

In the last case, Bonferroni Post-hoc-Tests are performed to identify exactly which groups are significantly different from others.



\subsection{From timeline}
\textbf{TODO: For now with only 9 participants, it is hard to extract any relevant result from the logs. The timelines vary considerably. So far, we can say that S2 seems faster than S1. And aberrant result can be found in certain scenarios. However, we also extract the ratio of optimal human action, depicting if the participant followed optimally or not the given instructions. This metric helps to justify this aberrant results since the participant behaved significantly differently than the others.}

\subsubsection*{Preferences satisfaction (task completion time + time to be freed)}
We should be able to show that in scenario with similar execution, i.e. S1 and S2, RF allows to solve the task faster, thus, human preferences are better satisfied with RF (\textbf{Currently not significant...}).

The S1 1 (S1 task time completion) group had higher values (M = 60,49, SD = 6,74) than the S2 1 (S2 task time completion) group (M = 56,04, SD = 4,42). A t-test for paired samples showed that this difference was statistically significant, t(15) = 2,32, p = ,035. (Medium effect)

In other pairs of scenarios, the erroneous estimation leads to significant execution differences between HF and RF. Thus, when there is an erroneous estimation using the RF regime the human fails to satisfy optimally their preferences. Let's look at the other two pairs where there is an erroneous human preferences estimation. 

With Pair B, HF allows to solve the task significantly faster than using RF. Thus, the human preferences are significantly more satisfied using HF.   
Indeed, the S3 1 group (S3 time task completion) had lower values (M = 68,79, SD = 10,68) than the S4 1 group (S4 time task completion) (M = 83,08, SD = 4,56). A t-test for paired samples showed that this difference was statistically significant, t(15) = -4,68, p = <.001. Large effect.

With Pair C, the human prefers to be freed as soon as possible. We mesured the time after which the human is freed from the task, that is, the time the human places their pink bar which is the only cube not reachable by the robot. The results that HF allowed the human to be freed significantly earlier than with RF. Thus, with erroneous estimation, HF satisfies significantly better the human preferences than RF.
Indeed, the S5 19 group had lower values (M = 21,92, SD = 2,1) than the S6 19 group (M = 63,9, SD = 8,44). A t-test for paired samples showed that this difference was statistically significant, t(15) = -19,47, p = <.001. Large effect. 

\subsubsection*{Ratio human optimally}
The participants were given objective/preferences to satisfy and that guide their behavior. However, in practice, the explicit actions to conduct were not given, and the participants were free to act as they will. Naturally, not all participants behaved in the same way. There were differences on the reaction time of each, but also in the action decisions, leading to different execution traces. Since different execution traces influence significantly influencing the metrics of the timeline, it is worth discussing how the participants behaved.
In the same manner as for the robot, an optimal human policy is generated for each scenario (considering the actual preferences given to the participant). Hence, it is possible to check at each step if the participant performed the optimal action or not, and thus, compute an optimal ratio which is the number of optimal human action performed divided by the total number of human action performed.  
Though there are no significant difference between the different scenarios, some scenarios still have lower average optimal ratio and high standard deviation, meaning that participants tend to have more various behaviors on these specific scenarios. 
The average number of human action per scenario is about 7, from x to x. 

\begin{table}
    \caption{Optimal human action ratio per scenario}
    % \vspace{-15pt}
    \begin{center}
    \begin{tabular}{c|c|c|c|c|c|c|}
        \cline{2-7}
                                                & S1    & S2    & S3    & S4    & S5    & S6\\
        \hline
        \multicolumn{1}{|c|}{Mean}              & 94.14 & 97.42 & 88.52 & 98.18 & 95    & 90.09 \\
        \hline
        \multicolumn{1}{|c|}{Std. Deviation}    & 9.27  & 5.72  & 8.37  & 3.96  & 10.54 & 14.31 \\
        \hline
    \end{tabular}
    \end{center}
    \label{tab:optimal_human_ratio}
\end{table}

As depicted in the table~\ref{tab:optimal_human_ratio}, S3 has the lowest optimal ratio meaning that participants tend to follow less the optimal course of action in this scenario. It is interesting to also comment on the std. Deviation of certain scenarios. S6 has the highest sd, which can be expected because the robot, by placing its pink bar even though the human holds their own, prevents the participants from being free early. This is a frustrating and confusing behavior to which participants react in various ways. The optimal policy indicates that human should drop their pink bar back on the table and help the robot stack the cubes in order to place the dropped pink bar faster. In practice, a number of participants act as such. However, a significant amount of participants get confused and are passive during several steps. Some even remain passive almost for the whole task, waiting for the robot to stack the cubes alone until the human pink bar has to be placed. This diversity in the participants' reaction is reflected in the high sd of S6.
The low sd of S3 is also noticeable. Indeed, here the robot tends to steal the cube from the reach of the human. This behavior prevents the human from acting, and thus, from making decisions. As a result, fewer decisions are taken by the human in this scenario which results in a high average optimal ratio and a low associated std deviation. 

\subsubsection*{RF execution is faster}
Since RF doesn't wait for the human decision, and generally the robot action are slightly slower than the human one, the RF regime should result in faster execution.
The S1 1 group had higher values (M = 60,49, SD = 6,74) than the S2 1 group (M = 56,04, SD = 4,42). A t-test for paired samples showed that this difference was statistically significant, t(15) = 2,32, p = ,035.
In addition, since agents have to synchronize together at each step, we are able to measure the amount of time the human has to wait for the robot. This amount should be significantly lower when using RF than HF. (Wait ns?)

\subsubsection*{Reaction time}
Participants reaction time fluctuates a lot. Especially with HF. Indeed, at every step, the HF robot waits for a defined amount of time to observe the human decision, and acts accordingly. Any human visual signal received interrupts this timer. This amount of time will be referred to as the HF Timeout because after it is reached the robot considers the human as passive.

By the way, this timeout was initially set to 3$s$ with the hypothesis that it should be quite small to allow fluent interaction. With a precise action in mind, the human is able to act first, otherwise, the robot fluently takes the lead and acts first. However, during the preliminary tests, the participants felt in a rush and oppressed by this relatively low timeout. Indeed, when they didn't have a precise action to perform when the step started, they didn't have the time to think properly and tend to be rushed by the timer progressing towards the timeout. Hence, we decided to increase the timeout from 3$s$ to 4$s$ which was felt way more comfortable. 

An interesting comment is that the participants tend to react in two different ways regarding the timeout. Either the participants were acting quite fast or sometimes they were taking as much time as the timeout was allowing them. That is, seeing that they had time to think about their action the participants tend to take that time systematically instead of acting fast. \textbf{Not sure to be able to prove it, not even that systematic so maybe just remove the comment}

\begin{figure}
    \centering
    \includegraphics[width=\linewidth]{images/Chapter5/av_reaction_times.png}
    \caption{Average reaction times. For all 6 scenarios. HF has odd numbers, and RF has even numbers. The horizontal dotted line represents the mean value and the dotted triangles the standard deviation around the average.}
    \label{fig:h_reaction_time}
\end{figure}

The average reaction times are measured as follows. After one participant finished one scenario, we measured their reaction time on each step. To do so, we first consider for each step the time when the step begins, which is signaled with text, a gaze and a sound from the robot. Then we consider the time when the human sends a signal by either starting an action or by waving their hand. The duration between these two times is considered as the reaction time of the human. Note that if the human remains passive (no signal) for a step, no reaction time is computed for this specific step. Then, we extract the average reaction time of the participant on the scenario from all the computed ones, compute the standard deviation \textit{and get the maximum and minimum values.}

Average reaction over all scenarios is about $0.76s$. Max values : ?

The ANOVA indicates that the only significant differences are between S2-S5 and S5-S6. 
Indeed, S2 is the RF robot with a correct estimation of human preferences. Since the robot takes the lead people tend to follow the robot and pay attention to what it is doing, not only to the scene. Hence, it is longer to process both the scene and the robot action/intentions and the reaction time tend 

\textbf{TODO: extract max and min reaction/decision time?}


\subsection*{H waiting for the robot (passive, tps mort)}

\subsection*{Tps execution actions}

\subsection*{Combien de fois dépasse TO?}
\textbf{TODO: hard to measure actually... Would be easy to just count number to TO signals. However, with AUTO-PASS TO are triggered anyway, and which shouldn't be counted as TO.}

\subsection{From questionnaires}

Some questions have large standard deviations, such as about the perceived ``Intelligence'', but often only on specific scenarios.

The ``Reactive'' answers are overall the same, whatever execution regime and scenario. 
Overall, based on an ANalysis Of the VAriance with repeated measures (ANOVA), all other answers than ``Reactive'' vary significantly. Post-tests show that the main differences come from scenarios S4 and S6. Those two scenarios correspond to ones where the robot has an incorrect estimation of the human preferences and is following the \textit{Robot-First} execution regime. This indicates that all HF behaviors are perceived similarly despite the erroneous estimation of the robot. It also indicates that the RF scenario with correct estimation is positively perceived, but an erroneous estimation has a significant impact on the answers when using RF.  

Except for the ``Reactive'' aspect, answers about RF have a larger standard deviation. Thus, participants tend to be more indecisive regarding RF than HF.

We notice very few significant differences on HF only.

\begin{figure}
    \includegraphics[width=\linewidth]{images/Chapter5/Mean per scenario.pdf}
    \caption{Mean answers per scenario}
    \label{fig:mean_per_scenario}
\end{figure}

\begin{figure}
    \includegraphics[width=\linewidth]{images/Chapter5/SD per execution regime.pdf}
    \caption{SD per Execution Regime}
    \label{fig:sd_per_execution_regime}
\end{figure}


\begin{figure}
    \centering
    \includegraphics[width=\linewidth]{images/Chapter5/positive_scale.png}
    \caption{Positive Interaction Scale with ANOVA + Post-hoc-Tests}
    \label{fig:positive_scale_anova}
\end{figure}

\begin{figure}
    \centering
    \includegraphics[width=\linewidth]{images/Chapter5/adaptivity_scale.png}
    \caption{Adaptivity Scale with ANOVA + Post-hoc-Tests}
    \label{fig:adaptivity_scale_anova}
\end{figure}


% ns  P > 0.05
% *   P ≤ 0.05
% **  P ≤ 0.01
% *** P ≤ 0.001

\begin{table}[]
    \footnotesize
    \begin{tabular}{c|c|c|c|c|c|c|c|c|c|c|c}
    \cline{2-11}
                                                 & \begin{tabular}[c]{@{}c@{}}p\\ value\end{tabular} & \begin{tabular}[c]{@{}c@{}}Effect\\ size\end{tabular} & 4-1          & 4-2          & 4-3          & 4-5          & 6-1          & 6-2          & 6-3          & 6-5          &           \\ \cline{1-11}
    \multicolumn{1}{|c|}{Responsive}             & \textit{ns}                                       & \textit{0.14}                                         & \textit{ns}  & \textit{ns}  & \textit{ns}  & \textit{ns}  & \textit{ns}  & \textit{ns}  & \textit{ns}  & \textit{ns}  & \textit{} \\ \cline{1-11}
    \multicolumn{1}{|c|}{Competent}              & \textit{***}                                      & \textit{0.05}                                         & \textit{*}   & \textit{**}  & \textit{**}  & \textit{**}  & \textit{**}  & \textit{***} & \textit{***} & \textit{**}  & \textit{} \\ \cline{1-11}
    \multicolumn{1}{|c|}{Intelligent}            & \textit{***}                                      & \textit{0.25}                                         & \textit{ns}  & \textit{ns}  & \textit{ns}  & \textit{ns}  & \textit{ns}  & \textit{ns}  & \textit{**}  & \textit{ns}  & \textit{} \\ \cline{1-11}
    \multicolumn{1}{|c|}{\textbf{Positive}}      & \textit{***}                                      & \textit{\textbf{0.65}}                                & \textit{***} & \textit{***} & \textit{***} & \textit{**}  & \textit{***} & \textit{***} & \textit{***} & \textit{***} & \textit{} \\ \cline{1-11}
    \multicolumn{1}{|c|}{Simple}                 & \textit{**}                                       & \textit{0.22}                                         & \textit{ns}  & \textit{ns}  & \textit{ns}  & \textit{ns}  & \textit{ns}  & \textit{ns}  & \textit{ns}  & \textit{ns}  & \textit{} \\ \cline{1-11}
    \multicolumn{1}{|c|}{Clear}                  & \textit{***}                                      & \textit{0.36}                                         & \textit{*}   & \textit{ns}  & \textit{ns}  & \textit{ns}  & \textit{**}  & \textit{ns}  & \textit{*}   & \textit{*}   & \textit{} \\ \cline{1-11}
    \multicolumn{1}{|c|}{\textbf{Adaptive}}      & \textit{***}                                      & \textit{\textbf{0.67}}                                & \textit{***} & \textit{***} & \textit{***} & \textit{***} & \textit{***} & \textit{***} & \textit{***} & \textit{***} & \textit{} \\ \cline{1-11}
    \multicolumn{1}{|c|}{Useful}                 & \textit{***}                                      & \textit{0.42}                                         & \textit{ns}  & \textit{**}  & \textit{ns}  & \textit{*}   & \textit{*}   & \textit{***} & \textit{ns}  & \textit{***} & \textit{} \\ \cline{1-11}
    \multicolumn{1}{|c|}{\textbf{Efficient}}     & \textit{***}                                      & \textit{\textbf{0.69}}                                & \textit{***} & \textit{***} & \textit{***} & \textit{**}  & \textit{***} & \textit{***} & \textit{***} & \textit{***} & \textit{} \\ \cline{1-11}
    \multicolumn{1}{|c|}{\textbf{Appropriate}}   & \textit{***}                                      & \textit{\textbf{0.65}}                                & \textit{***} & \textit{**}  & \textit{**}  & \textit{ns}  & \textit{***} & \textit{***} & \textit{***} & \textit{***} & \textit{} \\ \cline{1-11}
    \multicolumn{1}{|c|}{\textbf{Accommodating}} & \textit{***}                                      & \textit{\textbf{0.67}}                                & \textit{***} & \textit{*}   & \textit{**}  & \textit{**}  & \textit{***} & \textit{***} & \textit{***} & \textit{***} & \textit{} \\ \cline{1-11}
    \multicolumn{1}{|c|}{Predictable}            & \textit{***}                                      & \textit{0.41}                                         & \textit{*}   & \textit{ns}  & \textit{ns}  & \textit{ns}  & \textit{***} & \textit{ns}  & \textit{***} & \textit{**}  & \textit{} \\ \cline{1-11}
    \end{tabular}
    \caption{Significant differences in the questionnaire answers between the different scenarios. For each item of the questionnaire, }
    \label{tab:questionnaire_answers}
\end{table}


\begin{table}[]
    \footnotesize
    \begin{tabular}{ccc}
    \hline
    \textbf{Dimension}                         & \textbf{Question}                                                                                                                     & \textbf{Item}             \\ \hline
    \multirow{3}{*}{\textbf{Robot perception}} & \multirow{3}{*}{\begin{tabular}[c]{@{}c@{}}In your opinion,\\ the robot is rather:\end{tabular}}                                      & Apathetic/Responsive      \\
                                               &                                                                                                                                       & Incompetent/Competent     \\
                                               &                                                                                                                                       & Unintelligent/Intelligent \\ \hline
    \multirow{3}{*}{\textbf{Interaction}}      & \multirow{3}{*}{\begin{tabular}[c]{@{}c@{}}In your opinion, the interaction\\ with the robot was:\end{tabular}}                       & Negative/Positive         \\
                                               &                                                                                                                                       & Complicated/Simple        \\
                                               &                                                                                                                                       & Ambiguous/Clear           \\ \hline
    \multirow{3}{*}{\textbf{Collaboration}}    & \multirow{3}{*}{\begin{tabular}[c]{@{}c@{}}In your opinion, the collaboration\\ with the robot to perform the task was:\end{tabular}} & Restrictive/Adaptive      \\
                                               &                                                                                                                                       & Useless/Useful            \\
                                               &                                                                                                                                       & Inefficient/Efficient     \\ \hline
    \multirow{3}{*}{\textbf{Acting}}           & \multirow{3}{*}{\begin{tabular}[c]{@{}c@{}}In your opinion, the robot\\ choices of action were:\end{tabular}}                         & Inappropriate/Appropriate \\
                                               &                                                                                                                                       & Annoying/Accommodating    \\
                                               &                                                                                                                                       & Unpredictable/Predictable \\ \hline
    \end{tabular}
    \caption{PeRDITA Questionnaire: Participants have to place themselves between the two antonym items in a scale of 7.}
    \label{tab:perdita_questionnaire}
\end{table}

\section{Discussion}

fazfa


\section{Conclusion}

fazfafza
% \include{Chapters/Chapter6_Inhus}
% \chapter{Conclusions}
% \addstarredchapter{Conclusions} 
\markboth{Conclusions}{}


blabla conclusion

% \appendix
% \chapter{Simulating Human Agents for Testing HAN}
One of the challenges during the development of a HAN system is to test the system before its final deployment and real-world runs. Robotic simulators are of great use during this period as we can test the system under various conditions and in several environments. Unlike the classical setting, testing HAN requires the simulation of humans, which is still research under development. Until recently, the HAN community used the crowd simulators like Pedsim or MengeROS~\cite{aroor2017mengeros} to simulate humans in semi-crowded or crowded scenarios. However, the motion generated by these simulators uses reactive schemes like \acrshort{sfm} or \acrshort{orca}, which are good for generating crowds but lack intelligence at the level of an individual human. Recent simulators like SEAN~\cite{tsoi2020sean, tsoi2022sean} and SocNavBench~\cite{biswas2022socnavbench} tried to generate somewhat intelligent behaviours using new approaches and real-world data. However, these human agents are either not reactive (as they replay real-world trajectories without considering the robot) or use schemes similar to \acrshort{orca}. Although they have better human agents and environments for testing HAN, they still lack intelligent agents that can be used to test intricate scenarios in spaces like offices, labs or homes. Hence, we have used different ways to control the human(s) while testing the HAN proposed in this thesis. These ways include manual control using a joystick, a simple human controller that follows the generated trajectories, and finally, an intelligent human with rational decision-making capabilities.      

\section{Planning and Control for Human Agents}
In this section, we talk about the simple modes and planners employed to control the human avatar in the robot simulator. A human is assumed to be a robot with special requirements.

\subsection{Manual Control}
One of the simplest ways to control a human is to move the human manually using a keyboard or joystick. This is efficient in testing some very complicated scenarios involving intelligent decisions. Since a real human is already controlling the human avatar in the simulator, all the decision-making process is handled by the human operator. To integrate such human avatar into our system, we used the \textbf{\textit{joy}}\footnote{\url{http://wiki.ros.org/joy}} \acrshort{ros} package and then mapped the inputs of the joystick to the avatar's velocity with a cap at \SI[per-mode=symbol]{2}{\metre\per\second}.

Manual control is good for testing some interesting and particular cases, but it becomes tiresome to run several scenarios to benchmark or quantify the results. Moreover, the simulation runs cannot be completely automated as the human operator is always involved in the loop. So, the next solution was to plan and control the human, like the robot. It is different from collision avoidance algorithms as the human has a global path to trace and a separate local planner to move the human. 

\subsection{Planning based Control}
To automate and replicate the tasks easily, we have developed a simple \textbf{\textit{humans navigation}}\footnote{\url{https://github.com/sphanit/humans_nav}} package using the \textbf{\textit{global planner}}\footnote{\url{http://wiki.ros.org/global_planner}} in \acrshort{ros} and a simple controller. The developed system has two modes of operation:
\begin{enumerate}
    \item \textit{Trajectory following}: In this mode, the human follows the trajectory that is provided externally through a \acrshort{ros} topic. We used this mode to test the ideal situations where the human follows the trajectory planned by \acrshort{cohan}.
    \item \textit{Goal-based Control}: This mode is more autonomous as we only provide a goal via a \acrshort{ros} topic, and the system plans and moves the human to the goal. The planning module updates the plan as the human moves, and the simple controller traces the path. This mode was used to run multiple tests to check how well the robot adapts to the human.
\end{enumerate}
Both these modes can control more than one human simultaneously. The \textit{Trajectory Following} mode used the trajectories planned by \acrshort{cohan} to move the humans. The trajectory provided the desired velocities, which were sent directly to the human controller. However, in the \textit{Goal-based Control} mode, the velocities were calculated based on the current position and planned paths of the humans. To accept multiple goals and plan for all the humans together, a \textbf{\textit{multi-goal planner}} was developed, and it is used internally by the \textbf{\textit{humans navigation}} package to get plans based on the provided goals.

Even though this kind of system solves the issues of automation and is less tiresome to the developer, the human agent is still not intelligent and simply follows the given trajectory. Although the trajectory provided by \acrshort{cohan} takes care of many human-robot social constraints, this ideal behaviour may not be expected from humans. In the second mode of control, the human agent might have better behaviour, but the agent is still not intelligent and somewhat reactive, like in collision avoidance schemes. 

\section{InHuS}
InHuS\footnote{\url{https://github.com/AnthonyFavier/InHuS_Social_Navigation}}~\cite{favier2022_hri} stands for \textbf{I}ntelligent \textbf{Hu}man \textbf{S}imulator, and it was developed to specifically simulate a human agent that is rational and persistent about its goal, unlike the reactive schemes. 
\begin{figure}[!hb]
    \centering
    \includegraphics[width=0.9\columnwidth]{images/appendix/architecture.pdf}
    \caption{InHuS Architecture. The system consists of the Boss and the InHuS Human Controller macro components. The human operator interacts with the system through the Boss which in turn interacts with the Human Controller and the HAN Planner. Both the human and the robot controllers are connected to the same simulator where they control different components but share the same environment.}
    \label{fig:inhus}
\end{figure}
\subsection{Architecture}
The architecture of the proposed system is shown in Fig.~\ref{fig:inhus}. InHuS is composed of two major components namely, 1) \textit{Boss} and 2) \textit{Human Controller}. These components are explained in detail below:
\begin{itemize}[label={}]
    \item \textbf{Boss}: The \textit{Boss} component is responsible for taking input from the user and sending the appropriate instructions to the human and the robot planners. Hence it is provided with a graphical user interface through which a user can give individual goals to each agent, run or re-run predefined scenarios or initiate an endless loop of the human and robot navigating continuously from one goal to another. The endless loop can be used to identify the limits of the HAN system under test. After taking the input from the user, this component communicates the navigation goals to the \textit{Robot Manager} and \textit{Human Manager} at the appropriate times. Once the goals are communicated to these components, the \textit{Boss} does not interfere with the execution unless the human operator selects a new goal or scenario.
    \item \textbf{Human Controller}: This is the main component of InHuS that controls human motion and decides what to do in case of a conflict with the robot. It has an internal `\textit{Human Behaviour Model}' that makes these decisions and sets some attitudes to humans. The other major sub-component is the `\textit{Supervisor}', which supervises the execution of the goal and the progress, and activates the respective components as needed. While the `\textit{Geometric Planner}' provides the geometric path and trajectory to the goal, the `\textit{Low Level Controller}' sub-component sends the command velocity to the human avatar. The `\textit{Task Planner}' was used to define different kind of navigation tasks like \textit{go to goal}, \textit{wait} etc. Finally, the \textit{Log Manager} of this component logs the data and sends it to a GUI for visualisation of the calculated metrics.
\end{itemize}

\subsection{Supervisor and Geometric Planner}
The navigation goal of the human avatar is sent to the \textit{Supervisor}, which in turn asks the \textit{Task Planner} for a plan to achieve this goal. The \textit{Supervisor} then supervises and coordinates the execution of all the actions in the plan while managing the conflicts. The navigation plan generally consists of `moving' and `waiting' actions. This kind of architecture allows us to define complex navigation goals with multiple steps. The \textit{Supervisor} queries the \textit{Human Behaviour Model} from time to time to detect any potential conflicts. It has the power to suspend the execution of a plan in case of a conflict and resume it whenever necessary. This is especially useful to execute other actions in case of conflict to show the navigation intention and goal persistence of the human agent, unlike the reactive or simple planning approaches.

When the \textit{Supervisor} has to execute a `moving' action, it sends the navigation goal to the \textit{Geometric Planner}, which generates the path and then calculates the velocity commands to make the human move towards the goal. Depending on the type of trajectory planner selected, human motion can be different. In the current version, the standard \acrshort{ros} Navigation Stack or \acrshort{cohan} can be selected as the \textit{Geometric Planner}. The velocity command given by this component is not directly sent to the human. The \textit{Low Level Controller} receives this velocity and, if necessary, modifies or perturbs this velocity before sending it to the human avatar. This is used to emulate some reactions while navigation called `\textit{Attitudes}', which is presented in the next subsection.

\subsection{Human Behaviour Model}
The \textit{Human Behaviour Model} is the most important component of the proposed architecture. It controls the human avatar and is responsible for the behaviours exhibited by the human agent. As mentioned previously, this component manages the navigation conflicts, and in this version, only the blockage of the path by the robot is handled. Whenever the \textit{Geometric Planner} is called for the first time, the shortest path to the goal without any moving agents is calculated and sent to the `\textit{Navigation Conflict Manager}' of this component. If any blockage of this path by the robot is detected by this component, it changes the human state, and then the \textit{Supervisor} suspends the goal. The human avatar then performs an `approach' action where it moves towards the goal until it reaches a limiting distance from the robot. After this, it stops and waits for the robot to clear the way. Note that any collision avoidance or simple planning strategies will fail to show such behaviour as they will completely change the path of the human agent instead of showing persistence towards the goal.

This component can also set the goals for the human agent apart from the \textit{Boss}. The internal goal selection mechanism is responsible for different \textit{Attitudes} of the human avatar. Three kinds of attitudes are provided in InHuS:
\begin{enumerate}
    \item \textbf{Stop and Look}: It emulates a curious behaviour, where the human avatar navigating to the goal stops and looks at the robot shortly if the robot is in close vicinity of the human avatar. After this action, the navigation to the original goal is resumed.
    \item \textbf{Harass}: This attitude emulates a behaviour where the human avatar continuously disturbs the robot by blocking the robot's path. The idea here is to generate a child-like behaviour for the human avatar.
    \item \textbf{Random Goal}: In this attitude, a new random goal is set to the human avatar while it is already moving towards a goal, emulating something like a change of mind.
\end{enumerate}
To make the system more realistic, this module builds the perception of the human avatar using the map of the environment and the location of the other agents. The information about the other agents is taken directly from the simulator rather than using simulated cameras or lasers. Therefore, the human avatar does not consider the robot that it cannot see, even if it is below the threshold distance geometrically.

\subsection{Logs and Metrics}
The \textit{Log Manager} logs the data of the human-robot navigation interactions and sends it to GUI based data visualiser. This visualiser shows the different states of the human, the paths of the human and the robot, and some metrics to evaluate the robot's navigation. The logged data can also be used to calculate new metrics or methodologies for evaluation. A screenshot of this visualisation is shown in Fig. \ref{fig:gui_inhus}.
\begin{figure}[!ht]
    \centering
    \includegraphics[width=0.9\columnwidth]{images/appendix/gui_bis_bis.png}
    \caption{The data visualisation in GUI. On the right, the paths taken by the agents are shown, while on the left, the human states and the calculated metrics are shown.}
    \label{fig:gui_inhus}
\end{figure}
The paths shown on the right in Fig. \ref{fig:gui_inhus} are coloured over time. It means that the same colour on the paths represents the same time instant, and using this, one can interpret the behaviour of the agents better. On the left, the plot on the top shows the human avatar's distance to the goal and its estimated state over time. If no conflict occurs, the human stays in a single state, and the distance to the goal decreases linearly. The plots below the first one show some of the calculated metrics and the agents' velocities over time. One can calculate and add more metrics as needed using the logged data.

\subsection{Generating Different Behaviours}
\begin{figure}[!h]
    \centering
    \includegraphics[width=0.9\columnwidth]{images/appendix/paths_coop_new.png}
    \includegraphics[width=0.9\columnwidth]{images/appendix/paths_stop_new.png}
    \caption{Traversed paths generated by InHuS during the Pillar corridor scenario. The top part is with cooperative settings and the bottom part with non-cooperative settings along with the \textit{Stop and Look} attitude.}
    \label{fig:paths_pillar_corridor_inhus}
\end{figure}
Depending on the \textit{Geometric Planner} and the \textit{Attitude}, different navigation behaviours can be emulated for the human avatar. For example, using the standard \acrshort{ros} Navigation stack and \textbf{Stop and Look} attitude, we can simulate a non-cooperative human who contributes nothing in a setting like a corridor. If \acrshort{cohan} is used, a cooperative yet curious human can be simulated. Moreover, \acrshort{cohan} can be tuned to set a degree of cooperative behaviour. The comparisons of different combinations and behaviours generated are presented in more detail in \cite{favier2021simulating}. Fig. \ref{fig:paths_pillar_corridor_inhus} shows the paths of the robot and human in two situations, one where the human is non-cooperative and curious and the other in which the human is completely cooperative.

% \section{ImHuS}

% \chapter{Effects of Social Constraints}
Each of the proposed social constraints in this thesis has some particular effect on the behaviour of the robot. Some of the predominant effects of these are briefly presented here. Each case presents a scenario without and with the social constraint activated. The figures of each scenario show the paths taken by the human and the robot (starting at blue and moving towards red) and their velocities (robot's velocity in red and human's velocity in blue) below. The velocity plot also includes the distance between the human and the robot during the execution of the scenario. 

\section{TTCplus Constraint}
\subsection{Approach}
\begin{figure}[!htb]
\centering
\begin{subfigure}{0.5\columnwidth}
  \includegraphics[width=\textwidth]{images/appendix/ttc/approach/approach_without.png}
\end{subfigure}
\vspace{0.5cm}
\begin{subfigure}{0.8\columnwidth}
  \includegraphics[width=\textwidth]{images/appendix/ttc/approach/without.png}
  \caption{without TTCplus}
\end{subfigure}

\begin{subfigure}{0.5\columnwidth}
  \includegraphics[width=\textwidth]{images/appendix/ttc/approach/approach_with.png}
\end{subfigure}
% \hspace{-0.75cm}
\begin{subfigure}{0.8\columnwidth}
  \includegraphics[width=\textwidth]{images/appendix/ttc/approach/with1.png}
  \caption{with TTCplus}
\end{subfigure}
\caption{The robot approaches a human head-on. The addition of the TTCplus constraint makes the robot deviate a little and slow down as it nears the human.}
\label{fig:approach_ttc}
\end{figure} 

In this scenario, the robot and human move towards each other and stop at a very close distance from each other. From Fig.~\ref{fig:approach_ttc} (a), it can be seen that the robot and human move at their full speeds towards each. However, with the addition of the TTCplus constraint, the robot has a decreasing velocity profile as the human-robot distance decreases, showing the robot's intention to stop (shown in Fig.~\ref{fig:approach_ttc} (b)).

\subsection{Open Space Crossing}
\begin{figure}[H]
\centering
\begin{subfigure}{0.5\columnwidth}
  \includegraphics[width=\textwidth]{images/appendix/ttc/wide/without.png}
\end{subfigure}
\vspace{0.5cm}
\begin{subfigure}{0.8\columnwidth}
  \includegraphics[width=\textwidth]{images/appendix/ttc/wide/wide_without1.png}
  \caption{without}
\end{subfigure}

\begin{subfigure}{0.5\columnwidth}
  \includegraphics[width=\textwidth]{images/appendix/ttc/wide/with.png}
\end{subfigure}
% \hspace{-0.75cm}
\begin{subfigure}{0.8\columnwidth}
  \includegraphics[width=\textwidth]{images/appendix/ttc/wide/wide_with2.png}
  \caption{with}
\end{subfigure}
\caption{The human and the robot cross each other in an open space. The addition of the TTCplus constraint makes the robot move aside quickly, showing its intention to give way and the choice of its side to move.}
\label{fig:open_space_ttc}
\end{figure} 

\hspace{\parindent} This scenario simulates a robot crossing a human in an open area where there is enough space to move away and not disturb the human. In Fig.~\ref{fig:open_space_ttc} (a), the robot and human move directly towards each other and only avoid each other at the last minute before the collision. This puts on an additional burden on the human to deviate from his path to avoid a collision with the robot. A more human-aware robot should avoid the occurrence of such path deviation, which is similar to what is seen in Fig.~\ref{fig:open_space_ttc} (b). Therefore, the TTCplus constraint not only shows its intention to move away early but also reduces the additional navigational burden that might be imposed on the human.  

\subsection{Corridor Crossing}
\begin{figure}[H]
\centering
\begin{subfigure}{0.5\columnwidth}
  \includegraphics[width=\textwidth]{images/appendix/ttc/corridor/without.png}
\end{subfigure}
\vspace{0.5cm}
\begin{subfigure}{0.8\columnwidth}
  \includegraphics[width=\textwidth]{images/appendix/ttc/corridor/coor_without2.png}
  \caption{without}
\end{subfigure}

\begin{subfigure}{0.5\columnwidth}
  \includegraphics[width=\textwidth]{images/appendix/ttc/corridor/with.png}
\end{subfigure}
% \hspace{-0.75cm}
\begin{subfigure}{0.8\columnwidth}
  \includegraphics[width=\textwidth]{images/appendix/ttc/corridor/coor_with2.png}
  \caption{with}
\end{subfigure}
\caption{The human and the robot cross each other in a corridor. TTCplus constraint not only makes the robot take a side early but also slows down the robot as it crosses the human.}
\label{fig:corridor_ttc}
\end{figure} 

\hspace{\parindent} This scenario is similar to the previous one, but the space is narrower. As there is not enough space to move away, the robot with the TTCplus constraint moves to a side as well as slows down, as shown in Fig.~\ref{fig:corridor_ttc} (b). Without this constraint, it behaves exactly like in the previous case (Fig.~\ref{fig:corridor_ttc} (a)). 

\section{Relative Velocity Constraint}
\subsection{Corridor Crossing}
\hspace{\parindent} The corridor crossing scenario is the same as the one shown in the previous section. In the case of the Relative Velocity constraint, the robot should try to move away as quickly as possible and provide more space for the human even when the line of travel is not the same. This can be clearly seen from the path and the velocity profile of the robot in Fig.~\ref{fig:corridor_rel}. The robot starts to move to one side very quickly and slows down as it crosses the human. If there was enough space, the robot could have moved with a larger velocity while crossing. This constraint addresses parallel travel better when compared to TTCplus.

\begin{figure}[H]
\centering
% \begin{subfigure}{0.4\columnwidth}
%   \includegraphics[width=\textwidth]{images/appendix/relvel/corridor/without.png}
% \end{subfigure}
% \vspace{0.5cm}
% \begin{subfigure}{0.8\columnwidth}
%   \includegraphics[width=\textwidth]{images/appendix/relvel/corridor/corr_without2.png}
%   \caption{without}
% \end{subfigure}

\begin{subfigure}{0.5\columnwidth}
  \includegraphics[width=\textwidth]{images/appendix/relvel/corridor/with.png}
\end{subfigure}
% \hspace{-0.75cm}
\begin{subfigure}{0.8\columnwidth}
  \includegraphics[width=\textwidth]{images/appendix/relvel/corridor/corr_with2.png}
  % \caption{with Relative Velocity Constraint}
\end{subfigure}
\caption{The human and the robot cross each other in a corridor. Relative Velocity constraint makes the robot clear the way quickly and move with a slower speed robot as it crosses the human at a small parallel distance.}
\label{fig:corridor_rel}
\end{figure} 

\subsection{Open Space Crossing}

\begin{figure}[H]
\centering
% \begin{subfigure}{0.4\columnwidth}
%   \includegraphics[width=\textwidth]{images/appendix/relvel/wide/without.png}
% \end{subfigure}
% \vspace{0.5cm}
% \begin{subfigure}{0.8\columnwidth}
%   \includegraphics[width=\textwidth]{images/appendix/relvel/wide/without2.png}
%   \caption{without}
% \end{subfigure}

\begin{subfigure}{0.5\columnwidth}
  \includegraphics[width=\textwidth]{images/appendix/relvel/wide/with.png}
\end{subfigure}
% \hspace{-0.75cm}
\begin{subfigure}{0.8\columnwidth}
  \includegraphics[width=\textwidth]{images/appendix/relvel/wide/with2.png}
  % \caption{with}
\end{subfigure}
\caption{The human and the robot cross each other in an open space. The addition of the Relative Velocity constraint makes the robot take a large deviation by exploiting the available space while showing its intention to give way. It also facilitates the robot to move at a larger velocity towards the goal.}
\label{fig:wide_rel}
\end{figure}

\hspace{\parindent} In this setting, the robot has enough space to move away and then travel with a larger velocity. Without the addition of the Relative Velocity constraint, the robot and human avoid each other moments before the collision, similar to Fig.~\ref{fig:open_space_ttc} (a). This constraint makes the robot move away very quickly and exploit the available space to move at full speed towards its goal. It can be observed from the path and the speed profile of the robot in Fig.~\ref{fig:wide_rel}. This clearly shows how the Relative Velocity constraint addresses the parallel travel better compared to the TTCplus constraint (see Fig.~\ref{fig:open_space_ttc} (b)).

\section{Visibility Constraint}
\hspace{\parindent} To show the advantage of adding the visibility constraint, an overtaking scenario is simulated. In this setting, the robot encounters a human moving very slowly and partially blocking its way. The robot has to over the human in order to move to its goal faster. In the first case, shown in Fig.~\ref{fig:visib_const} (a), without the addition of Visibility constraint, the robot overtakes the human very closely and also disturbs his navigation. With the addition of this constraint, however, the robot takes a large deviation and tries to enter the human's field of view as far as possible without disturbing him. This can be observed from the plots in Fig.~\ref{fig:visib_const} (b).
\begin{figure}[H]
\centering
\begin{subfigure}{0.5\columnwidth}
  \includegraphics[width=\textwidth]{images/appendix/vis/without.png}
\end{subfigure}
\vspace{0.5cm}
\begin{subfigure}{0.8\columnwidth}
  \includegraphics[width=\textwidth]{images/appendix/vis/without1.png}
  \caption{without}
\end{subfigure}

\begin{subfigure}{0.5\columnwidth}
  \includegraphics[width=\textwidth]{images/appendix/vis/with.png}
\end{subfigure}
% \hspace{-0.75cm}
\begin{subfigure}{0.8\columnwidth}
  \includegraphics[width=\textwidth]{images/appendix/vis/with2.png}
  \caption{with}
\end{subfigure}
\caption{The robot overtakes a human who is moving very slowly. The addition of the Visibility constraint makes the robot enter the human's field of view slowly without surprising or disturbing the human.}
\label{fig:visib_const}
\end{figure}

\section{Updated Invisible Humans Constraint}
\hspace{\parindent} As shown in chapter~\ref{chap:5}, the current version of the `Invisible Humans' constraint already addresses a lot of scenarios to proactively accommodate sudden human appearances. However, the defined formulation had some issues which needed to be addressed using passage detection and mode shifting. After testing the robot navigation in more complicated scenarios, we observed that the current formulation could lead to some deadlocks even after the passage detection. One such deadlock situation is shown in Fig.~\ref{fig:inv_fail}. Here, the robot faces opposing forces from the obstacles and the invisible humans and freezes before it can even detect an opening.
\begin{figure}[h]
    \centering
    \includegraphics[width=0.8\columnwidth]{images/appendix/inv/updated_inv.png}
    \caption{A situation where the current formulation of the Invisible Humans constraint could fail. The opposing forces from the obstacles and the invisible humans make the robot freeze without moving.}
    \label{fig:inv_fail}
\end{figure}
Therefore, we update our formulation by taking inspiration from the Relative Velocity constraint. Instead of using the velocity of the visible humans, we use the defined invisible humans' velocity in the formulation and update the formulation as follows:
\begin{equation}
\begin{split}
cost_{inv\_human} &= max\left(\frac{V-a\Delta t_n+\lVert\overrightarrow{V_r}\rVert+1}{d}, 0\right)\quad \text{if}\quad \Delta t_n> 0.5s\\
                &= \frac{V}{d}\quad otherwise
\end{split}
\label{updated_inv_eq}
\end{equation}

In the latest version of CoHAN, the above formulation is used instead of the previous one. The rigorous testing of the updated formulation is still pending, but we already see some improvements over the previous one. An example of the constrained door crossing is presented below.

\subsection{Testing the Updated Constraint}
\hspace{\parindent} The above formulation acts on the robot's velocity during the possible freezing scenarios and makes the robot move with lower velocities, and reduces the cost. Since it is an updated formulation of the Invisible Humans constraint, it should still hold the properties of the previous formulation. To show this, we have simulated the door crossing scenario again with a wall on the side that limits the space. 

In Fig.~\ref{fig:door_inv_new} (a), the robot moves without considering and accounting for the invisible humans in the environment. Therefore, it moves at almost full speed and takes the shortest path to reach the goal. With the formulation in chapter~\ref{chap:5}, the robot froze between the wall and the entry to the door and did not move. However, with the updated formulation, the robot moves away from the door as much as possible without colliding with the wall and also aligns itself to properly pass through the door. Further, while crossing the door, the robot moves very cautiously with a slower velocity, as seen in Fig.~\ref{fig:door_inv_new} (b) between $140-145 s$.
\begin{figure}[!ht]
\centering
\begin{subfigure}{0.3\columnwidth}
  \includegraphics[width=\textwidth]{images/appendix/inv/without.png}
\end{subfigure}
\vspace{0.5cm}
\begin{subfigure}{0.8\columnwidth}
  \includegraphics[width=\textwidth]{images/appendix/inv/without2.png}
  \caption{Without Invisible Humans constraint}
\end{subfigure}

\begin{subfigure}{0.3\columnwidth}
  \includegraphics[width=\textwidth]{images/appendix/inv/with.png}
\end{subfigure}
% \hspace{-0.75cm}
\begin{subfigure}{0.8\columnwidth}
  \includegraphics[width=\textwidth]{images/appendix/inv/with2.png}
  \caption{with the updated Invisible Humans constraint}
\end{subfigure}
\caption{Constrained door crossing scenario. The robot has to enter a door in a narrow space and try to accommodate humans as much as possible. The updated Invisible Humans constraint makes the robot move close to the wall before aligning itself towards the door and carefully entering it.}
\label{fig:door_inv_new}
\end{figure}

%%%%%%%%%%%Need to update this%%%%%%%
% \include{Annexe_fr_long}

\bibliographystyle{StyleThese}
% \bibliographystyle{plain}
% \bibliographystyle{unsrt}
%%%%%%%%%%%%Need to update this%%%%%%%%%%%%
\bibliography{These-refs}

% \cleardoublepage
% \begin{vcenterpage}
% \noindent\rule[2pt]{\textwidth}{0.5pt}

% % 1700 à 4000 caractères

% \textbf{Abstract:}
% %%%%%%%%%%%%%%%%%%%%%% ATTENTION ! SI MODIFICATION => MODIF SUR ADUM AUSSI !!!

% %%%%%%%%%%%%%Need to update this%%%%%%%%%%%%%%%%%%5
% \textcolor{red}{As robots begin to enter our daily lives, we need advanced knowledge representations and associated reasoning capabilities to enable them to understand and model their environments. Considering the presence of humans in such environments, and therefore the need to interact with them, this need comes with additional requirements. Indeed, knowledge is no longer used by the robot for the sole purpose of being able to act physically on the environment but also to communicate and share information with humans. Therefore knowledge should no longer be understandable only by the robot itself, but should also be able to be narrative-enabled. 
% In the first part of this thesis, we present our first contribution with Ontologenius. This software allows to maintain knowledge bases in the form of ontology, to reason on them and to manage them dynamically. We start by explaining how this software is suitable for \acrfull{hri} applications. To that end, for example to implement theory of mind abilities, it is possible to represent the robot’s knowledge base as well as an estimate of the knowledge bases of human partners. We continue with a presentation of its interfaces. This part ends with a performance analysis, demonstrating its online usability. 
% In a second part, we present our contribution to two knowledge exploration problems around the general topic of spatial referring and the use of semantic knowledge. We start with the route description task which aims to propose a set of possible routes leading to a target destination, in the framework of a guiding task. To achieve this task, we propose an ontology allowing us to describe the topology of indoor environments and two algorithms to search for routes. The second knowledge exploration problem we tackle is the \acrfull{reg} problem. It aims at selecting the optimal set of piece of information to communicate in order to allow a hearer to identify the referred entity in a given context. This contribution is then refined to use past activities coming from joint action between a robot and a human, in order to generate new kinds of Referring Expressions. It is also linked with a symbolic task planner to estimate the feasibility and cost of future communications. 
% We conclude this thesis by the presentation of two cognitive architectures. The first one uses the route description contribution and the second one takes advantage of our Referring Expression Generation contribution. Both of them use Ontologenius to manage the semantic Knowledge Base. Through these two architectures, we present how our contributions enable Knowledge Base to gradually take a central role, providing knowledge to all the components of the architectures.}

% \textbf{Keywords:} human-robot interaction, Human-Aware Navigation, Social Navigation
% \\
% \noindent\rule[2pt]{\textwidth}{0.5pt}
% \end{vcenterpage}

\end{document}
