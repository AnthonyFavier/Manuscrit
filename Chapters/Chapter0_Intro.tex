\chapter*{Introduction}
\addstarredchapter{Introduction}
\markboth{Introduction}{Introduction}

\newcommand{\mysection}[1]{%
    \section*{#1}%
    \addcontentsline{toc}{section}{#1}%
    \markright{#1}%
}


\minitoc

\acrfull{hrc} is a growing field in robotics and \acrfull{ai} research that aims to enable safe and effective teamwork between humans and robots.
This field concerns fully autonomous robots. Hence, it excludes exoskeletons or teleoperated robots such as surgery manipulators or remotely operated (aerial) vehicles. 

The work presented in this manuscript is assumed to be in the Joint Action context, which is described in \cite{sebanz_joint_2006} as ``any form of social interaction whereby two or more individuals coordinate their actions in space and time to bring about a change in the environment''. Various relationships can exist between humans and robots, e.g., collaborators, companions, guides, tutors, or social interaction partners. In all these cases, we think the robot should help and facilitate the human in terms of physical effort, mental workload, and even emotional state.

Industrial robots are already popular in factories because they are fast, accurate, reliable, and never tire, which makes them ideal for repetitive factory tasks. 
However, such robots are usually contained in dedicated areas where humans cannot enter for safety reasons. 
Hence, it is still an open challenge to endow robots with enough reliable reasoning capabilities and compliant motion control to allow efficient and trusted direct collaboration between humans and robots. 

This work aims to design autonomous robots able to make explainable, acceptable, and efficient decisions to collaborate with humans. Moreover, \acrshort{hrc} can also occur in various contexts that must be taken into account which range from co-worker robots in factories to householder robots for our everyday lives and include service robots in public places like restaurants and shops.


\mysection{Human-Robot Collaboration Challenges}

\acrshort{hrc} opens several challenges to address, each corresponding to a different aspect or functionality that the robot should be endowed with to fulfill its role. Hence, the ideal general collaborative robot should be the aggregation of all the notions below: 

\begin{itemize}
    \item \textbf{Navigation}: The robot should be able to acceptably and efficiently move in a human-populated environment. This implies being mechanically designed for it, anticipating and planning correct trajectories, and being able to adapt and follow these trajectories in real time. The robot should not move threateningly and should account for humans.

    \item \textbf{Manipulation}: The robot should be able to manipulate objects to interact with its environment. Hence, the robot should have an actuator like an arm and a gripper and should be able to exhibit motions that are efficient and safe to nearby humans.

    \item \textbf{Decision-making}: The robot should be able to make relevant decisions to be collaborative. This implies being able to plan its actions to solve a collaborative task. It also implies being able to supervise the task execution and make online decisions to adapt to uncertainties in real time, which includes human actions and commands. 

    \item \textbf{Communication}: To achieve congruent interaction and collaboration, collaborative agents must communicate. This implies that the robot should be able to communicate information to the human and understand the one received from the latter. These communications can be of various types, e.g., verbal, using natural language, prompting text on a screen, and signaling through robot movements (arms, head, mobile base). More innovative techniques can also be mentioned such as projecting arrows or desired paths on the ground, and using Augmented Reality to show internal information on the robot (decision, next action, etc...).

    \item \textbf{Perception}: Eventually, the robot must be able to perceive its environment. This is mandatory for the robot to have a reliable perception scheme to know the position of near objects, obstacles, and humans. First, relevant sensors must be used and placed on the robot or in the environment itself. After, from sensory data, some analysis and reasoning processes must extract relevant facts about the robot's environment such as objects' positions, spatial relations, reachable objects, human knowledge and intentions, the state of the current goal, and more. This constitutes the knowledge of the robot which may include an estimation of the near human knowledge. Perception is critical because most of the other challenges rely on the robot's knowledge.

\end{itemize}
    

\mysection{Contributions and manuscript organization}

The main challenge of HRC/HRI addressed in my PhD is \textbf{decision-making}. My work led to three main contributions concerning two distinguishable subfields. My two first contributions concern decision-making during task planning, to decide and plan the robot's action in order to solve a task collaboratively or simply in the presence of humans. My third contribution addresses decision-making in navigation, especially, how to simulate interactive social navigating agents endowed with decision-making processes to challenge robot navigation schemes.
As a result, the structure of my manuscript is as follows.

Chapter~\ref{chap:1} provides more details about the context of my PhD by discussing related fields and works of HRC. This chapter eventually highlights the gap in the literature that motivates my PhD work.

Part~\ref{part:1} gathers all my work concerning task planning for HRC. It represents a major portion of my PhD work and includes the chapters~\ref{chap:2}, \ref{chap:3}, \ref{chap:4}, \ref{chap:5}, and \ref{chap:6}.

Chapter~\ref{chap:2} familiarizes the reader with the HATP/EHDA task planner. Indeed, I participated in its development and this planner has been the keystone of my two contributions to task-planning for HRC. Hence, the reader should understand both this task planner's motivation and methods.

Chapter~\ref{chap:3} introduces my first main contribution which incorporates Theory of Mind concepts in HRC task-planning, more precisely, in the HATP/EHDA planning process.
Some models and algorithms are proposed and evaluated to better estimate human beliefs in order to better anticipate their potential actions. As a result, we can identify when the human has a false belief about a fact evaluated as relevant for the task. In such cases, the robot can proactively inform the human to correct the false belief or the robot can purposely delay its action to make sure the human sees its execution and infer the corresponding fact, avoiding verbal communication. 

In Chapter~\ref{chap:4}, as my second main contribution, I address the lack of concurrent actions of HATP/EHDA to improve fluency in collaboration. Inspired by the Joint Action literature, we designed a model of concurrent and compliant execution for HRC in the form of an automaton. We also propose to evaluate plans based on an estimation of the human inner preferences. A novel task planning approach taking into account the mentioned execution model and plan evaluation is proposed. This approach generates concurrent robot policies compliant with human online decisions and preferences. 

Chapter~\ref{chap:5} is a technical description of an interactive simulator I developed to execute the robot policy generated by the approach described in the previous chapter. This simulator proposes an execution scheme based on the model of execution to run and supervise the robot policy in a 3D simulator. It also allows a human operator to perform actions through intuitive mouse control. Hence, this simulator offers a way to collaborate with a robot following the approach we designed and is used in the next chapter to evaluate the approach.    

Chapter~\ref{chap:6} presents a user study validating the approach proposed in Chapter~\ref{chap:4} using the simulator described in Chapter~\ref{chap:5}. For this purpose, several scenarios have been designed using a BlocksWorld task, and human participants were asked to collaborate with the simulated robot to evaluate its behavior. We compared our approach with a baseline consisting of a robot following a simpler version of the model of execution. 

Conclusions regarding part~\ref{part:1} follow this chapter before starting the second part of this thesis. 

Part~\ref{part:2} concerns decision-making in navigation and how to simulate interactive social navigating agents. Despite not being my main research topic, this subject represents significant work in my PhD. This part includes the two last chapters: \ref{chap:7} and \ref{chap:8}.

Chapter~\ref{chap:7} describes my third contribution in which I address the decision-making challenge in navigation to allow the robot to move in a human-populated environment acceptably and efficiently. I developed a system producing an ``intelligent'' human avatar that, while being reactive, can make rational decisions about navigation tasks. This system serves as a benchmarking and testing tool for robot navigation systems to be challenged. This way, robot navigation systems can be evaluated, tuned, and stress tested in simulation allowing them to run mature real-life experiments faster.  

Chapter~\ref{chap:8} presents an additional work with the same rationales as the approach described in Chapter~\ref{chap:7} and is largely inspired by it. However, this work addresses a major limitation of the previous one which is not being able to simulate several human agents. This other approach can choreograph several agents with group movements and social behaviors. Despite having limited individual decision-making compared to the previous approach, this additional work generates pertinent and challenging intricate situations with several agents which is beneficial to the social robotic research.

Conclusions concerning both Chapter \ref{chap:7} and \ref{chap:8} end part~\ref{part:2}.

Eventually, I share general conclusions regarding all of my PhD work in a dedicated part before providing additional materials in the appendix.  

\mysection{List of Publications}

\subsubsection*{As main author}
\begin{itemize}

    \item Anthony Favier, Phani-Teja Singamaneni, Rachid Alami. Simulating Intelligent Human Agents for Intricate Social Robot Navigation. Social Robot Navigation workshop - Robotics: Science and Systems (RSS'21), Jul 2021, Washington, United States. 
    \item Anthony Favier, Phani-Teja Singamaneni, Rachid Alami. An Intelligent Human Avatar to Debug and Challenge Human-aware Robot Navigation Systems. Late Breaking Report - 2022 ACM/IEEE International Conference on Human-Robot Interaction (HRI'22), Mar 2022, Sapporo, Japan. 
    \item Anthony Favier, Shashank Shekhar, Rachid Alami. Robust Planning for Human-Robot Joint Tasks with Explicit Reasoning on Human Mental State. AI-HRI Symposium at AAAI Fall Symposium Series (FSS'22), Nov 2022, Arlington, United States. 
    \item Anthony Favier, Shashank Shekhar, Rachid Alami. Anticipating False Beliefs and Planning Pertinent Reactions in Human-Aware Task Planning with Models of Theory of Mind. PlanRob Workshop - International Conference on Automated Planning and Scheduling (ICAPS'23), Jul 2023, Prague, Czech Republic. 
    \item Anthony Favier, Shashank Shekhar, Rachid Alami. Models and Algorithms for Human-Aware Task Planning with Integrated Theory of Mind. IEEE International Conference on Robot and Human Interactive Communication (RO-MAN'23), Aug 2023, Busan, South Korea. 
    \item Anthony Favier, Phani Teja Singamaneni, Rachid Alami. Challenging Human-Aware Robot Navigation with an Intelligent Human Simulation System. Social Simulation Conference (SSC'23), Sep 2023, Glasgow, France. 
    \item Anthony Favier, Rachid Alami. Planning Concurrent Actions and Decisions in Human-Robot Joint Action Context. Symbiotic Society with Avatars workshop - ACM/IEEE International Conference on Human-Robot Interaction (HRI'24), Mar 2024, Boulder, United States.

\end{itemize}
    
\subsubsection*{As co-author}
\begin{itemize}
    
    \item Guilhem Buisan, Anthony Favier, Amandine Mayima, Rachid Alami. HATP/EHDA: A Robot Task Planner Anticipating and Eliciting Human Decisions and Actions. IEEE International Conference On Robotics and Automation (ICRA 2022), May 2022, Philadelphia, United States. ⟨10.1109/ICRA46639.2022.9812227⟩. 
    
    \item Phani-Teja Singamaneni, Anthony Favier, Rachid Alami. Towards Benchmarking Human-Aware Social Robot Navigation: A New Perspective and Metrics. IEEE International Conference on Robot and Human Interactive Communication (RO-MAN), 2023, Aug 2023, Busan, South Korea.
    \item Phani-Teja Singamaneni, Anthony Favier, Rachid Alami. Human-Aware Navigation Planner for Diverse Human-Robot Contexts. 2021 IEEE/RSJ International Conference on Intelligent Robots and Systems (IROS), Sep 2021, Prague (online), Czech Republic. 
    \item Phani-Teja Singamaneni, Anthony Favier, Rachid Alami. Invisible Humans in Human-aware Robot Navigation. IEEE International Conference on Robotics and Automation (ICRA 2022), May 2022, Philadelphia, United States.
    \item Phani-Teja Singamaneni, Anthony Favier, Rachid Alami. Watch out! There may be a Human. Addressing Invisible Humans in Social Navigation. 2022 IEEE/RSJ International Conference on Intelligent Robots and Systems (IROS 2022), Oct 2022, Kyoto, Japan. 

    \item Olivier Hauterville, Camino Fernández, Phani-Teja Singamaneni, Anthony Favier, Vicente Matellán, et al.. IMHuS: Intelligent Multi-Human Simulator. IROS2022 Workshop: Artificial Intelligence for Social Robots Interacting with Humans in the Real World, Oct 2022, Kyoto, Japan. 
    \item Olivier Hauterville, Camino Fernández, Phani-Teja Singamaneni, Anthony Favier, Vicente Matellán, et al.. Interactive Social Agents Simulation Tool for Designing Choreographies for Human-Robot-Interaction Research. ROBOT2022: Fifth Iberian Robotics Conference, Nov 2022, Zaragoza, Spain. 
\end{itemize}
    
    
    