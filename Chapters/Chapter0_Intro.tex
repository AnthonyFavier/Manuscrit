\cleardoublepage

\chapter*{Introduction}
\addstarredchapter{Introduction}
\markboth{Introduction}{Introduction}


\minitoc

\acrfull{hrc} is a growing field in robotics and \acrfull{ai} research that aims to enable safe and effective teamwork between humans and robots.
Exoskeletons, teleoperated robots such as surgery manipulators, and remotely operated (aerial) vehicles all have in common a human controlling and making decisions on behalf of the robot. In our context, autonomous robots must share tasks, decisions, and space with humans. For that, they need decisional abilities such as planning their actions appropriately and adapting online to the environment and human behavior. Hence, the \acrshort{hrc} field concerns fully autonomous robots endowed with decision-making capabilities and collaborating with humans.

The work presented is assumed to be in the Joint Action context, which is described in \cite{sebanz_joint_2006} as ``any form of social interaction whereby two or more individuals coordinate their actions in space and time to bring about a change in the environment''. Various relationships can exist between humans and robots, e.g., collaborators, companions, guides, tutors, or social interaction partners. In all these cases, we think the robot should help and facilitate the human in terms of physical effort, mental workload, and even emotional state.

Industrial robots are already popular in factories because they are fast, accurate, reliable, and never get tired, which makes them ideal for repetitive factory tasks. 
However, such robots are usually in dedicated areas where humans cannot enter for safety reasons. 
Hence, it is still an open challenge to endow robots with enough reliable reasoning capabilities and compliant motion control to allow efficient and trusted direct collaboration between humans and robots. 

Moreover, \acrshort{hrc} can also occur in various contexts that must be considered, ranging from co-worker robots in factories to householder robots for our everyday lives and including service robots in public places like restaurants and shops.

Focused on the decisional aspect of \acrshort{hrc}, this work aims to design autonomous robots able to make explainable, acceptable, and efficient decisions to collaborate with humans. 

\acrfull{hrc} is a multidisciplinary field that opens several technical challenges to address. The ideal collaborative robot should be the aggregation of solutions for each of the challenges listed below:

\begin{itemize}
    \item \textbf{Navigation}: The robot should be able to acceptably and efficiently move in a human-populated environment. This implies being mechanically designed for it, anticipating and planning correct trajectories, and being able to adapt and follow these trajectories in real time. The robot should not move threateningly and should account for humans.

    \item \textbf{Manipulation}: The robot should be able to manipulate objects to interact with its environment. Hence, the robot should have an actuator like an arm and a gripper and should be able to exhibit motions that are efficient and safe to nearby humans.

    \item \textbf{Communication}: To achieve congruent interaction and collaboration, collaborative agents must communicate. This implies that the robot should be able to communicate information to the human and understand the one received from the latter. These communications can be verbal or non-verbal.

    \item \textbf{Perception}: It is mandatory for the robot to have a reliable perception scheme to know the position of near objects, obstacles, and humans. First, relevant sensors must be used and placed on the robot or in the environment itself. After reasoning on the sensory data, relevant facts about the robot's environment, such as objects' positions, spatial relations, reachable objects, human knowledge, intentions, or the state of the current goal, must be extracted. 
    
    \item \uline{\textbf{Decision-making}}: This aspect is the focus of my work and implies that the robot should be able to make relevant decisions to be collaborative, including planning its actions to solve a collaborative task. It also implies being able to supervise the task execution and make online decisions to adapt to human actions, commands, and unexpected events. The listed skills are linked because the robot must reason and rely on each to make pertinent decisions. Consequently, I studied \textit{Perception}, \textit{Communication}, and \textit{Navigation} from the \textit{decisional} point of view during my PhD.

\end{itemize}
    
\mysection{Contributions}

My work addressed the \textbf{decision-making} aspect of \acrshort{hrc} and led to different contributions.
I began by participating in the development of a novel human-aware task planner called HATP/EHDA~\cite{buisan_hatpehda_icra}. This approach considers two distinct agent models: an uncontrollable one to estimate the human's behavior and a controllable one to plan the robot's actions accordingly. The agent models include distinct beliefs, agendas, and action models. This approach aims to make a clear distinction between the two agents and, most importantly, to model the distinct human decisional processes and to be able to reason on it. As a result, the planner can anticipate and even elicit human decisions and actions, but it never compels them. For these reasons, we believe this approach is promising to address \acrshort{hrc}, and I built two of my main contributions upon this approach.

My first contribution is to propose some models and algorithms of the Theory of Mind and to integrate them into the deliberation process of HATP/EHDA. Despite considering distinct beliefs, they were only updated according to the description of action effects provided in the domain and problem models. This means, for instance, that modeling the human observing and acquiring a new fact by entering a room had to be manually scripted in the `move' action description. With this contribution, we propose that an agent can acquire information by observing their surroundings or an action execution. Additionally, we propose a way to detect false human beliefs during the planning process, which may be detrimental to the task. Eventually, these relevant belief divergences are fixed by planning minimal verbal robot communication or delaying robot action that humans will initially not observe.

My second contribution brings planning and execution closer by addressing the turn-taking assumption of HATP/EHDA and exploring parallel executions. We formulated a step-based model of compliant and concurrent joint action execution. This model describes how the two agents should coordinate as well as four possible online human decisions about the execution: (1) the human decides to be passive and let the robot act alone, (2) the human acts alone and the robot is passive, (3) the human starts acting then the robot adapts and acts in parallel, and finally (4) the human deliberately let the robot decide and start acting before complying with it. This model guides the exploration of our proposed new human-aware task planning approach. After exploring all relevant courses of action, the robot's behavioral policy is extracted using a plan evaluation based on estimations of human preferences. Eventually, by following the produced policy, the robot can comply concurrently with any human online decisions and aims to satisfy human preferences, which can be updated online.

As another contribution validating the above approach, we implemented our proposed joint action model as an execution scheme into a dedicated simulator. Using this whole system, we conducted a user study where participants were invited to collaborate in several scenarios with a simulated robot following policies produced by our approach. In contrast with our approach, we used a baseline where the robot always imposes its decisions on the human. We showed through statistical analysis that our approach satisfies human preferences significantly more successfully than the baseline. Similarly, we have shown that our approach induces significantly more positive interaction, more adaptive and effective collaboration, and significantly more appropriate and accommodating robot decisions.

My last contributions concern simulating intelligent human agents. Endowing such simulated agents with decision-making capabilities can help to test, evaluate, and robustify interactive and collaborative robot systems. We propose a generic architecture called \acrfull{inhus} to simulate an intelligent autonomous human agent. Then, we present an implemented version for navigation use cases, permitting the generation of challenging intricate situations while recording and plotting relevant interaction data. This contribution simulates a single human agent endowed with complex reasoning processes navigating intricate environments.

Eventually, we present a last contribution simulating several navigating agents, namely \acrfull{imhus}. These agents can exhibit social group behaviors and be choreographed to challenge and benchmark robot systems.


\mysection{Manuscript Organization}

The above contributions are detailed in the rest of this manuscript, which is structured as follows.

Chapter~\ref{chap:1} provides more details about the context of my PhD. We describe the challenges linked to \acrshort{hrc} and how some of them can be tackled through task planning and simulation of human agents.

The rest of the manuscript is divided into two parts. Part~\ref{part:1} gathers all my work concerning task planning for \acrshort{hrc}. As a major portion of my PhD work, this part covers Chapters~\ref{chap:2} to \ref{chap:6}. Part~\ref{part:2} concerns decision-making in simulating intelligent autonomous agents. This part includes the two last chapters: \ref{chap:7} and \ref{chap:8}.

Part~\ref{part:1} begins with Chapter~\ref{chap:2}, which presents the HATP/EHDA task planner. This planner has been the keystone of most of my work. Hence, the reader should understand the motivation and methods of this task planner, which are described in this chapter.

Chapter~\ref{chap:3} describes my first main contribution, proposing models and algorithms to incorporate Theory of Mind concepts in \acrshort{hrc} task-planning. An empirical evaluation is provided and discussed, demonstrating how this contribution solves a broader class of problems than HATP/EHDA without systematic communication.

Chapter~\ref{chap:4} presents my second main contribution, proposing a new human-aware task planning approach based on a step-based compliant and concurrent joint action model. The approach's description is supported by empirical results proving its effectiveness in terms of the latitude of choice given to the human and the satisfaction of their internal preferences. We further validated this by developing an interactive simulator used for a user study, described in the following chapters.

Chapter~\ref{chap:5} presents our joint action model implemented as an execution scheme into a dedicated simulator. 
This simulator allows a human operator to perform actions through intuitive mouse control and collaborate with a simulated robot executing policies our planning approach produces. In addition to giving more details about our model, this chapter also provides technical details about the developed simulator. 

Chapter~\ref{chap:6} presents a user study validating the approach proposed in Chapter~\ref{chap:4} using the simulator described in Chapter~\ref{chap:5}. For this purpose, several scenarios were designed using a BlocksWorld task, and human participants were asked to collaborate with the simulated robot to evaluate its behavior using questionnaires. We compared our approach with a baseline behavior where the robot always imposes its decisions on humans. 

Part~\ref{part:2} begins with Chapter~\ref{chap:7}.
This chapter describes the \acrshort{inhus} system, addressing the challenge of simulating human agents endowed with decision-making capabilities. Its implementation in a navigation use case is presented. This chapter also compares two robot navigation systems using \acrshort{inhus}, proving that our approach effectively challenges robot schemes and allows measuring and comparing human-aware navigation properties.


Chapter~\ref{chap:8} presents \acrshort{imhus}, which complements the previous system to simulate and choreograph several agents with group movements and social behaviors. This system has been qualitatively evaluated in an elevator scenario.


Eventually, general conclusions regarding my overall PhD work will be shared, and additional materials will be provided in the appendix.  

\mysection{List of Publications}

\subsection*{As main author}
\begin{itemize}

    \item Anthony Favier, Phani-Teja Singamaneni, Rachid Alami. Simulating Intelligent Human Agents for Intricate Social Robot Navigation. Social Robot Navigation workshop - Robotics: Science and Systems (RSS'21), Jul 2021, Washington, United States. 
    \item Anthony Favier, Phani-Teja Singamaneni, Rachid Alami. An Intelligent Human Avatar to Debug and Challenge Human-aware Robot Navigation Systems. Late Breaking Report - 2022 ACM/IEEE International Conference on Human-Robot Interaction (HRI'22), Mar 2022, Sapporo, Japan. 
    \item Anthony Favier, Shashank Shekhar, Rachid Alami. Robust Planning for Human-Robot Joint Tasks with Explicit Reasoning on Human Mental State. AI-HRI Symposium at AAAI Fall Symposium Series (FSS'22), Nov 2022, Arlington, United States. 
    \item Anthony Favier, Shashank Shekhar, Rachid Alami. Anticipating False Beliefs and Planning Pertinent Reactions in Human-Aware Task Planning with Models of Theory of Mind. PlanRob Workshop - International Conference on Automated Planning and Scheduling (ICAPS'23), Jul 2023, Prague, Czech Republic. 
    \item Anthony Favier, Shashank Shekhar, Rachid Alami. Models and Algorithms for Human-Aware Task Planning with Integrated Theory of Mind. IEEE International Conference on Robot and Human Interactive Communication (RO-MAN'23), Aug 2023, Busan, South Korea. 
    \item Anthony Favier, Phani Teja Singamaneni, Rachid Alami. Challenging Human-Aware Robot Navigation with an Intelligent Human Simulation System. Social Simulation Conference (SSC'23), Sep 2023, Glasgow, France. 
    \item Anthony Favier, Rachid Alami. Planning Concurrent Actions and Decisions in Human-Robot Joint Action Context. Symbiotic Society with Avatars workshop - ACM/IEEE International Conference on Human-Robot Interaction (HRI'24), Mar 2024, Boulder, United States.

\end{itemize}
    
\subsection*{As co-author}
\begin{itemize}
    
    \item Guilhem Buisan, Anthony Favier, Amandine Mayima, Rachid Alami. HATP/EHDA: A Robot Task Planner Anticipating and Eliciting Human Decisions and Actions. IEEE International Conference On Robotics and Automation (ICRA 2022), May 2022, Philadelphia, United States. ⟨10.1109/ICRA46639.2022.9812227⟩. 
    
    \item Phani-Teja Singamaneni, Anthony Favier, Rachid Alami. Towards Benchmarking Human-Aware Social Robot Navigation: A New Perspective and Metrics. IEEE International Conference on Robot and Human Interactive Communication (RO-MAN), 2023, Aug 2023, Busan, South Korea.
    \item Phani-Teja Singamaneni, Anthony Favier, Rachid Alami. Human-Aware Navigation Planner for Diverse Human-Robot Contexts. 2021 IEEE/RSJ International Conference on Intelligent Robots and Systems (IROS), Sep 2021, Prague (online), Czech Republic. 
    \item Phani-Teja Singamaneni, Anthony Favier, Rachid Alami. Invisible Humans in Human-aware Robot Navigation. IEEE International Conference on Robotics and Automation (ICRA 2022), May 2022, Philadelphia, United States.
    \item Phani-Teja Singamaneni, Anthony Favier, Rachid Alami. Watch out! There may be a Human. Addressing Invisible Humans in Social Navigation. 2022 IEEE/RSJ International Conference on Intelligent Robots and Systems (IROS 2022), Oct 2022, Kyoto, Japan. 

    \item Olivier Hauterville, Camino Fernández, Phani-Teja Singamaneni, Anthony Favier, Vicente Matellán, et al.. IMHuS: Intelligent Multi-Human Simulator. IROS2022 Workshop: Artificial Intelligence for Social Robots Interacting with Humans in the Real World, Oct 2022, Kyoto, Japan. 
    \item Olivier Hauterville, Camino Fernández, Phani-Teja Singamaneni, Anthony Favier, Vicente Matellán, et al.. Interactive Social Agents Simulation Tool for Designing Choreographies for Human-Robot-Interaction Research. ROBOT2022: Fifth Iberian Robotics Conference, Nov 2022, Zaragoza, Spain. 
\end{itemize}
    
    
    