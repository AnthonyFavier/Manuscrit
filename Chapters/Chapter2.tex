
\ifdefined\included
\else
\setcounter{chapter}{1} %% Numéro du chapitre précédent ;)
\dominitoc
\faketableofcontents
\fi

\chapter{Related Work}
\minitoc
\chaptermark{State-of-Art}
\label{chap:2}



\section{chapter3}
Human-aware navigation involves navigation planning of a robot around humans. Humans follow certain social norms while navigating in an environment and expect the same from others, who are navigating in the same environment. Therefore, a robot cannot move along the shortest path while navigating around humans. Otherwise, it will create confusion and discomfort to humans. The theory of proxemics~\cite{hall_book_1966} provides a set of rules, that can be used to realize more human-like behaviour during robot motion and non-motion tasks \cite{rios2015proxemics}. Most state-of-art human-aware navigation planners add proxemics costs around humans in a grid-based map representation of the robot's working environment \cite{kruse_ras_2013}. There are other approaches like social force models \cite{helbing1995social} which also make use of proxemics. Nonetheless, proxemics alone may not be sufficient to completely generate a human acceptable trajectory for the robot. In the human-aware navigation planner proposed by Sisbot et al. \cite{sisbot_tr_2007}, other social criteria like visibility and hidden zones are considered along with proxemics. Another navigation framework proposed by Kruse et al. \cite{kruse_arso_2012} introduces the directional cost model, which attempts to solve the spatial conflict by adjusting velocity instead of the path when possible. This model has shown to increase the legibility of the robot motions and hence in HATEB-2, we introduced some new constraints that restrict the path change and adjust the velocity (Eq. \eqref{vel_eq}) based on the distance between human and the robot. The study conducted by Kruse et al. \cite{kruse2014evaluating} shows that humans prefer the robot to follow this strategy, especially in path crossing scenarios. 

Employing social constraints alone may not be sufficient to develop a socially acceptable navigation planner, and this arises the need for including human motion predictions into the framework \cite{kuderer_rss_2012}. Many methods based on social force model \cite{helbing1995social} predict homotopically distinct trajectories for humans and design planners that learn the navigation policies for the robot based on human demonstrations \cite{kuderer_rss_2012}. Although these methods involving independent human predictions work fine in large open spaces, they might require re-learning of parameters to handle situations such as long corridor crossing or passage through a door, where a cooperative behaviour is needed between human and the robot. Hence planning for humans along with the robot is required in such situations. The approach presented by Ferrer et al. \cite{ferrer2015multi}, uses the social force model for both to predict human paths and to control the robot motion. In this approach, the human predictions based on the previously planned path are used. Other approaches \cite{bordallo_iros_2015, nagariya_cdc_2015} try to predict the possible human goals based on some type of reasoning and generate locally optimal motion for the robot. One of the recent approaches \cite{fisac2018probabilistically} suggests the use of probabilistic human predictions to handle various uncertainties and plan robot motion on top of these probabilistic predictions. This approach is particularly useful in systems with unreliable sensors. All these approaches are effective in densely crowded environments as a virtue of remaining purely reactive but could lead to needless detours in intricate situations. Our previous work \cite{khambhaita2017viewing} is specifically developed to handle such intricate situations in semi-crowded environments. Such planning for humans along with the robot is usually required in robot-human handover scenarios, to know where to perform a task, and who performs a task \cite{mainprice_ro-man_2012,waldhart_iros_2015}. Similarly, in HATEB, the tightness of the elastic band can be adjusted to make either robot or the human take more load. 

The concept of modality shifting in human-aware navigation is discussed in works by Mehta et al. \cite{mehta2016autonomous} and Qian et al.\cite{qian2013decision}, where Partially Observable Markov Decision Process (POMDP) is used for decision making. In both of the works, different modalities necessary for human-aware navigation are proposed, assuming that the robot takes all the load of the navigation process. Hence these methods may also suffer problems like purely reactive planners in complex situations leading to unnecessary detours or long halts. HATEB-2 includes HATEB as one of the modalities and hence can handle both intricate situations as well as crowded scenarios, by switching between different modalities when needed. In this work, however, we focus mainly on different intricate situations involving cooperative motion between the human and the robot. 

\section{chapter4}

In the recent decade, more and more robots are entering into human environments. From the robotic vacuum cleaners to the human assisting robots in shops, malls \cite{kanda2009affective, foster2019mummer} and airports \cite{triebel2016spencer}, all of these robots are working in environments with humans moving around. To navigate in these places, the robot needs to be aware of the humans in the environment, and treating humans simply as obstacles may not be enough. Besides, the robot's motion should be safe, legible and acceptable to humans rather than being optimal from the sole point of view of the robot (time, energy etc.). Therefore, a new field of robotic navigation called human-aware (or social) navigation has emerged, which studies various human motion and social aspects for developing more human acceptable robotic navigation. 


There are a variety of human-aware navigation planners designed for different human-robot contexts. In the context of a crowd or robot navigation in the street, Ferrer. et al \cite{ferrer_social-aware_2013} presents a potential field based navigation using the social force model. The authors of \cite{truong_toward_2017} extended this to human-object and human group interactions by proposing the proactive social motion model. The work by Repiso et. al \cite{repiso2017line} shows the context of a robot accompanying a human. The authors of \cite{chen2017socially} address this crowd navigation problem by using reinforcement learning and, the works \cite{triebel2016spencer, okal2016learning} address the same with inverse reinforcement learning. Coming to other contexts, the works presented in \cite{sisbot2007human,truong2014dynamic} and \cite{kollmitz_time_2015} show some interesting costmap based approaches for planning paths in complex indoor scenarios that can occur at homes or offices. In this paper, we use a similar costmap based approach to handle static humans. Fernandez Carmona et. al \cite{carmona2019making} compares the performance of the existing navigation planners in a warehouse context and proposes an architecture to include humans in planning. The work of G\"{u}ldenring et. al \cite{guldenring2020learning} addresses the same context using reinforcement learning. Some other works like \cite{perez-higueras_robot_2014, perez2018teaching} use inverse reinforcement learning for confined and public space navigation contexts. Khambhaita and Alami \cite{khambhaita2017viewing} addressed the context of human-robot co-navigation based on an optimization-based approach. Note that none of the above planners was designed to handle multiple human contexts together. A multi-context human-aware navigation planning is a very new field, and not many works exist. Lu et. al \cite{lu_layered_2014} proposed a layered costmap based approach for handling different navigation contexts. A more recent work by Banisetty et. al \cite{banisetty2020deep} shows some promising results using a deep learning-based context classification and multi-objective optimization based navigation planner \cite{banisetty2019socially}. However, these results are validated only in indoor scenarios and, the authors do not present any results in a crowd, unlike the proposed system. 

In order to handle the dynamic humans in our navigation planner and plan a socially acceptable trajectory for the robot, a human motion prediction system is required. One of the classic approaches of human motion prediction is based on the social force model \cite{helbing1995social}. Ferrer et. al \cite{ferrer2015multi} uses this social force model both to predict human motions and to move the robot among the crowds. Kollmitz et. al \cite{kollmitz_time_2015} uses a simple linear prediction based on current human velocity. Instead of predicting the trajectory, a possible human goal can also be predicted using certain reasoning over a probable set of goals \cite{bordallo2015counterfactual}. Our proposed navigation system uses one such goal prediction system \cite{ferrer2014bayesian} as a part of the human path prediction module. Apart from this, our system offers three other human path prediction methods to handle different situations. In a recent work by Fisac et. al \cite{fisac2018probabilistically}, the authors suggest a probabilistic human model with confidence to handle the uncertainties in a system.

One of the key elements of the proposed system is the context-based shifting between different planning modes. This kind of modality shifting is discussed in the works of Qian et. al \cite{qian_decision-theoretical_2013}, and Mehta et. al \cite{mehta_autonomous_2016} based on Partially Observable Markov Decision Processes. Unlike these, our system uses situation assessment based modality shifting. In our previous work \cite{singamaneni2020hateb}, we introduced this modality shifting with three different modes of planning. In the current work, we extend this to handle a large number of humans and also introduce some elements, including a new planning mode. This modified \textit{HATEB local planner} is integrated into the proposed framework as the local planner.
\section{chapter5}
The motion planning problem in the presence of dynamic obstacles can be solved in several ways. Some works use grid-based planning \cite{kollmitz2015time, phillips2011sipp} while some others \cite{schoels2020nmpc, rosmann2021online} use optimizations like model predictive control. We use sparse graph optimization proposed in \cite{rosmann2013efficient} and add human-aware constraints as proposed in our previous works \cite{khambhaita2017viewing, singamanenihateb2020, singamaneni2021human}.

