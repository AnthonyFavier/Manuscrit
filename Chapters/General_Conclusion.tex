\newpage
\thispagestyle{empty}
\mbox{}


\chapter*{Conclusion}
\addstarredchapter{Conclusion} 
\markboth{Conclusion}{Conclusion}

\mysection{Contributions}

In this thesis, we addressed the challenge of enhancing the decision-making of robots to ensure seamless collaboration with a human partner. Several contributions have been proposed, contributing to robotic human-aware task planning on the one hand and to robot navigation by simulating intelligent social agents on the other. 

To better capture my contributions, we presented in Chapter~\ref{chap:1} the specificities and the multidisciplinarity of the Human-Robot Interaction field and the Human-Robot Collaboration subfield. Eventually, we presented state-of-the-art approaches in task planning and simulation for \acrshort{hrc}. 

\subsection*{Task Planning Conclusions}

Task planning for human-robot collaboration is a research topic of high interest to which numerous works have contributed. However, most of these works do not consider distinct agent models and, thus, do not consider different beliefs, action models, or goals. Moreover, works from the literature tend to produce a joint plan that must be shared and accepted by humans, assuming that humans are controllable and must follow the produced plan, which is an approach close to multi-robot planning. Finally, existing approaches usually assume the agents have already established a shared goal. However, exploring scenarios where the robot should decide `when' and `how' to start the collaboration is also relevant. 
Hence, there is a lack of task planning approaches that preserve the human latitude of online decisions while collaborating efficiently to solve both shared and individual tasks. 


In Chapter~\ref{chap:2}, we presented the HATP/EHDA human-aware task planner designed to address the highlighted gap in the literature. After participating in its development, this planning scheme became a laboratory to explore relevant human-aware task planning challenges, leading to two main contributions. 

In Chapter~\ref{chap:3}, we proposed models and algorithms to integrate concepts of Theory of Mind in the planning process of HATP/EHDA and plan the robot's actions appropriately. 
The main idea is that agents only update their beliefs from observable facts in their surroundings or by observing a co-present agent acting. 
More precisely, we modeled observability as follows. First, each fact describing the state of the world is associated with a place, which can be symbolic. The agents are situated and can move from one place to another. When situated in the same place, two agents are said to be co-present, and an agent and a fact are said to be co-located. Second, each fact is associated with an observability type, stating whether the fact is observable or not. For instance, the color of an object is observable and can be acquired if situated near the object. However, the presence of salt in water or the temperature of an object is not observable, but it can be inferred in certain situations.
Based on these models, we propose two rules to update agent beliefs. First, an agent acquires information from observing a co-located observable fact. These updates are done through a systematic Situation Assessment process inserted in the planning process of HATP/EHDA and executed after each planned action. Hence, the agent will automatically acquire information from its observable surroundings. For instance, when moving to another room, the agent's beliefs will be updated with (only) observable facts located in the other room.  There is no need to script this assessment in the `\textit{move}' action's effects. On the other hand, an agent acquires information from observing a co-present agent performing an action (including the agent itself). In this case, the agent's beliefs are directly updated with all action effects, including effects affecting non-observable facts. This rule models the inference that humans internally perform when observing their surroundings. For instance, if a human observes the robot adding some salt to the water, they can naturally infer that some salt is now in the water, although it is not observable.
Using those two rules, we can maintain human beliefs more precisely to predict their behavior better and detect false beliefs that may be detrimental to collaboration. Finally, we solve \textit{relevant} false human beliefs as follows. First, from the erroneous facts in human beliefs, we identify which ones must be corrected to solve the task. Then, a communication action correcting the minimal number of false beliefs is inserted into the robot's plan, potentially leaving some non-detrimental erroneous facts in the human beliefs. As a second possible solution, we check whether the false beliefs are due to a non-observed robot action. If so, we create another possible plan where the robot delays the relevant action until the human can observe its execution, avoiding the creation of a false belief. 
We evaluated this approach empirically using three domains, including shared tasks, several places, and non-observable facts. We showed that the proposed models and algorithms allow us to solve a broader class of problems than the original HATP/EHDA planner. Moreover, our relevant false beliefs detection and minimal communication already avoid systematic communication, and considering delaying robot actions reduces the communication rate of the solution plan produced even more. 


In Chapter~\ref{chap:4}, we proposed another contribution to bringing planning and execution closer. We proposed a step-based model of concurrent and compliant joint action execution. 
This model describes several possible coordination of the agents, including the four following cases: 1) the human decides to perform any desired action and the robot complies by executing its best non-conflicting action in parallel; 2) the human decides to be passive and let the robot act alone; 3) the human is acting alone while the robot remains passive; 4) the human decides to let the robot decide and start acting purposely, then the human accompany the robot with a concurrent non-conflicting action.
We use an abstracted version of this model to guide our search and explore further concurrent courses of action, allowing the robot to anticipate possible execution conflicts. 
The exhaustive exploration produces a directed acyclic graph where each path from the root to a leaf is a sequence of concurrent human-robot actions leading to the goal. From this graph produced offline, the robot's behavioral policy is extracted using an estimation of the human preferences regarding the joint task. These preferences can optimize various objectives, for instance, minimizing human efforts, finishing the task as soon as possible, or freeing the human as soon as possible.  
This policy indicates the best concurrent robot action to satisfy the estimated human preferences in every state and for each possible human action. This allows the robot to comply optimally with any human online decision.  
Moreover, this extraction is light and can be done online. As a result, as the execution progresses, the estimated human preferences can be reevaluated, and the robot policy can be updated accordingly on the fly.
To evaluate our approach, we used BlocksWorld scenarios where the human and the robot had to collaborate to stack colored cubes to match a given goal pattern. 
As a first step, we evaluate the approach by symbolically simulating possible executions, following the abstracted model of execution proposed. To highlight the compliance endowed to the robot, we simulated erroneous estimations of the human preferences, making the robot's behavior more or less adversarial to the human preferences. However, despite contributing to the shared task, the robot may not necessarily achieve the task as the human would have preferred. The results showed that despite erroneously estimated human preferences, the actual human preferences are correctly satisfied overall because the robot constantly adapts to and follows human decisions. 

As another contribution in Chapter~\ref{chap:5} and \ref{chap:6}, 
we implemented our proposed joint action model as an execution scheme into an interactive simulator permitting collaboration with a robot executing policies produced by our approach. 
Using this simulator, we conducted a user study with 25 participants and compared our approach with a contrasting baseline where the robot always imposes its decisions on humans. By recording execution data and requesting participants to answer a questionnaire, we showed that our approach performed and was appreciated significantly better than the baseline. Over the different scenarios, our approach satisfied the human preferences significantly better and the baseline. 
Moreover, the most significant differences using our approach are that the participant perceived the Interaction as more Positive, the Collaboration as more Adaptive and Efficient, and the Robot's Decisions as more Appropriate and Accommodating.    

\subsection*{Simulating Intelligent Human Agents Conclusions}

It is challenging to test and evaluate \acrshort{hri} systems. To tackle this issue, we proposed a generic architecture capable of simulating intelligent human agents to interact and challenge robot systems. We presented and evaluated an implementation of such architecture for the navigation use case. Various approaches address the human-aware navigation subject, but only a few recent works propose tools to challenge, test, debug, and evaluate robot human-aware navigation systems. Without such simulation tools, preliminary experimentation and debugging must be achieved using a real robot, which is slow and burdensome. 
Most existing simulated interactive agents are based only on reactive approaches, making them suitable for crowded scenarios but limited and unrealistic in intricate ones. 
Thus, there is a lack of simulated interactive agents endowed with some decision-making processes allowing them to resolve more realistically intricate scenarios, like path blockage or narrow corridor crossing.

In Chapter~\ref{chap:7}, we proposed a complete system to simulate agents endowed with decision-making capabilities. The InHuS architecture has been designed to be generic, but we implemented it as a first step for the navigation use case. 
This system aims to challenge and evaluate robot navigation in intricate human-populated scenarios. It simulates an environment including a robot and a human avatar. The robot is controlled by a navigation system that challenges and evaluates. The InHuS system controls the human agent. Scenarios, including start position and a goal for each agent, can easily be defined and reproduced. The avatar can be given compound goals corresponding to sequences of navigating and waiting actions to simulate complex human activities. 
The avatar uses the CoHAN human-aware navigation planner to navigate, producing proactive and legible trajectories. 
Additionally, the avatar detects when the robot blocks the shortest path to its goal. Instead of following another path, the system switches to a conflict mode, makes the avatar approach, and eventually stops near the blocking spot to show its intention to cross. Once the path is cleared, the avatar proceeds with nominal navigation.
To diversify the challenging situations, \textit{Attitudes} can be defined in the system. They are modes that can be activated anytime, influencing the avatar's goals and reactions to the robot. We can simulate curious, distracted, and non-cooperative humans through these modes. There is also an \textit{Endless} mode, making the agents repeatedly move to defined agent-specific positions. This is useful for generating unexpected situations and testing the robustness of the robot system over time. Another interesting feature is the ability to randomize all position goals. The final goal position is randomly sampled around the original within a given radius. 
To make the avatar's behavior more realistic, the system computes the visibility of the avatar and updates the robot's known position only when it is observable. 
Finally, the system records execution logs, such as agents' positions, speeds, and additional computed metrics. Among these metrics are the Time To Collision, the estimated time until the agents collide, and the Surprised metrics. The latter helps identify sudden close robot appearances from behind the avatar. All these data are plotted on an interactive visual interface, providing real time feedback while running the system.
We evaluated this system using two robot navigation systems. One is CoHAN, with human-aware properties. The other is close to the default \acrshort{ros} robot navigation stack, without human-aware properties.
After running both systems in a few challenging scenarios, the data gathered with InHuS indicates that, as expected, CoHAN exhibited significantly better human-aware properties than the default system. This suggests that InHuS can effectively challenge and evaluate the human-aware properties of a robot navigation system. 

Eventually, in Chapter~\ref{chap:8}, we proposed the IMHuS system, a complementary approach to the InHuS system. Indeed, InHuS is limited to simulating a single avatar. IMHuS permits the choreographing of several agents and the production of group and social behaviors. The IMHuS agents do not have individual decision-making capabilities. However, this framework allows producing intricate scenarios with multiple human agents to challenge robot navigation systems. Hence, it is an interesting solution between intricate single-agent scenarios and wide crowded scenarios. We showed the effectiveness of the approach in an elevator scenario. 


\mysection{Limitations and future works}

In this section, we discuss the limitations of the contributions presented in this thesis and the possible future works they suggest. 

\subsection*{Theory of Mind in Human-Aware Task Planning}

There are some limitations to the approach presented in Chapter~\ref{chap:3} that introduce Theory of Mind concepts in the HATP/EHDA planner.
First, we do not consider uncertainties in the robot's knowledge about the world. Despite being linked to agent knowledge, and thus chapter~\ref{chap:3}, this limitation is common to everything linked to HAPT/EHDA. Indeed, since the planner is part of the robot, there is no choice but to rely on the robot's knowledge about the world, hoping it is correct. An interesting future work could be to abstract the perception layer of the robot and associate each state variable to a confidence level. This confidence level would indicate how much the corresponding fact is sure. Then, the robot could rely only on the high-confidence ones. 

Secondly, we do not consider uncertainties in human beliefs. We assume to be in the worst-case scenario where humans do not know about facts they did not see. However, considering the pasta cooking example of Chapter~\ref{chap:3}, the human could assume that the robot effectively did its work while being away fetching the pasta. Thus, the human could believe that there is salt in the post even without seeing the robot's action. Keeping track of such different possible worlds is an approach used in epistemic task planning. It would be an interesting future work that we already started investigating in our lab. 

Another promising future work is to add more reasoning and inference processes to the planning process. We have already introduced the situation assessment process, updating agents' beliefs concerning their observable surroundings. However, adding logical reasoning and deduction would be interesting, like in the Dynamic Epistemic Logic in \cite{bolander_gentle_2017}. For instance, if it is known that A or B is true and the agent knows that B is false, then it can be deduced that A is true.
However, it would undoubtedly require using a knowledge representation different from state variables.

Inference processes based on causality could also be promising. For instance, if action \textit{a2} can only be performed after \textit{a1} and an agent notices that \textit{a2} has been performed, then the agent can infer that \textit{a1} has also been performed. 

Finally, we can identify the minimal relevant information to communicate in the proposed approach. However, it would be an interesting future work to explore `when' to communicate. Currently, the robot communicates at the last moment, just before the estimated false human beliefs have an effect. However, it could be wiser to plan the communication earlier, maybe even before the human leaves.   

\subsection*{Task Planning for Concurrent and Compliant Joint Action}

There are also a few limitations in my second contribution, each that can be relevant to address in future work.
First, we can mention the ``passive action pairs''. Indeed, when planning the robot's policy, the cases where both the robot and the human are passive are explicitly modeled but not explored because, without any action, there is no modification to the state and, thus, no need to create a new state. These situations could happen during execution but are not considered in the current plan evaluation to generate the robot's policy. Nevertheless, in some rare situations, it can be relevant for the robot to remain passive even if the human is passive. This would insist on and non-verbally communicate that it is better for humans to act than robots. 

Secondly, exploring and adding the possibility for the robot to take the initiative in the execution model would be interesting. A step starts with the robot waiting for a human signal before proceeding. The only exception is when the human cannot act, the robot directly starts acting. Nevertheless, when the robot's action does not depend on the human decision, it could be relevant to skip the initial synchronization and make the robot act since there is no possible conflict. 

Moreover, the model's step-based aspect can be considered a limitation. It benefits the search but constrains the execution with many synchronizations. Investigating a proper execution controller based on the model that can supervise the robot's policy execution in a flexible manner, avoiding systematic synchronizations, would be interesting.

Eventually, we conducted a user study to validate the approach using an interactive simulator. Simulation and mouse control could bias the results obtained. Hence, a pertinent future work is to make the simulation more immersive and natural, e.g., using Virtual Reality (VR). Implementing the scheme on a real robot is also interesting but would require a significant amount of engineering work to avoid bias created by a slow-moving robot or perception errors. 

\subsection*{Performances of the planning approach}

The overall planning approach proposed in this thesis is based on an exhaustive, expensive search, which does not scale. Collaboration scenarios are not usually very long, making our approach sufficient and adequate for this context. Nevertheless, despite the short scenarios, their complexity can be significantly high. In Chapter~\ref{chap:4}, we proposed switching from the AND/OR tree search space to a Directed Acyclic Graph, significantly improving performance. Still, the approach does not scale. As a result, it would be relevant to identify pertinent heuristics to avoid exhaustive exploration. One idea is to investigate the risk estimation of subtasks. It would allow starting the execution of a non-fully refined plan. As humans, we tend to apply this reasoning. For instance, consider a collaborative scenario where the task is to set the table. Due to the numerous types and number of objects to set, the possible orderings are considerable, making the exhaustive search infeasible. However, knowing that there are no risks of deadends or failure in this context, one could start with the closest objects without precisely knowing from the beginning the future sequence of actions to perform.  

\subsection*{Simulating Social Navigating Agents}

The InHuS was initially designed as a generic intelligent agent simulation architecture to solve and benchmark any collaborative task. However, implementing the entire generic architecture is very challenging, and as a first step, we limited the scheme to navigation. Hence, future work could be to integrate a complex task planner like HATP/EHDA inside InHuS as well as manipulation controllers. 

The major limitation of InHuS is that it simulates only a single agent. This limitation has been addressed with the IMHuS system, which allows the designing and reproducing of challenging navigation scenarios with several humans. However, in IMHuS, agents lack individual decision-making processes like in InHuS. Another approach to extending InHuS to multi-human scenarios could be switching agent control dynamically. We could run two instances of InHuS to control the two closest humans to the robot, and we would control the other agents using reactive approaches. This would require reasonable computational power but necessitate carefully keeping track, pausing, and resuming each agent's goal.

Finally, this work was done at the beginning of my PhD. Since then, other recent works have proposed stimulating social interactive navigating agents to benchmark robot navigation. An interesting future work is to look deeper into these recent approaches and compare them with InHuS to potentially combine them and propose a more refined and efficient version of InHuS. 

\subsection*{Unexplored features of HATP/EHDA}

When working on my contributions based on HATP/EHDA, I reimplemented the original planner several times to address specific challenges. As a result, I sometimes simplified some aspects of the planner and even removed some features irrelevant to my focus. However, it is worth indicating that all original HATP/EHDA features are pertinent to collaborative scenarios and could be integrated back into my contributions' planner versions. I provide a few details on two aspects not present in my thesis examples, but they would be worth investigating further.

First, I would like to mention the ``\textit{Trigger}'' mechanism implemented in HATP/EHDA. \textit{Triggers} model agent's reactions to particular situations rather than goal-oriented actions. For instance, whatever the human is doing, when handed over an object or asked a question, the human is likely to interrupt their activity to grab the object or answer the question. \textit{Triggers} are helpful to describe complex human action models, later used to estimate the next actions the human partner is likely to perform in a given state. For simplicity reasons, I did not implement \textit{Triggers} in my experimental version of the planner, but this feature does not conflict with my contributions. Adding this feature would only require a small amount of additional work.

Additionally, the use cases of my contributions always considered that a shared goal had been established priorly between the agents. However, it is worth mentioning that HATP/EHDA can manage scenarios without shared goals. Their creation is handled by the given agent action models, where \textit{Triggers} help model questions and answers. Scenarios without an initial shared goal were less illustrative for my examples, but the proposed approaches are not limited to this case. Moreover, designing and integrating reasoning processes indicating, in a principled way, `when' and `how' to create a shared goal with a human partner is an interesting work in the future.
