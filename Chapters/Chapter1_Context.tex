\ifdefined\included
\else
\setcounter{chapter}{0}
\dominitoc
\faketableofcontents
\fi

\chapter{Human Robot Collaboration Context}
\chaptermark{Human Robot Collaboration Context}
\label{chap:1}
\minitoc



%%%%%%%%%%%%%%%%%%%%%%%%%%%%%%%%%%%%%%%%%%%%%%%%%%%%%%%%%%%%%%%%%%%%%%%%%%%%%%%%%%%%%%
\section{HRI Interaction}

\subsection{human human interaction}
\subsection{human computer interaction}
\subsection{human robot interaction}

\subsection{dialogue}


%%%%%%%%%%%%%%%%%%%%%%%%%%%%%%%%%%%%%%%%%%%%%%%%%%%%%%%%%%%%%%%%%%%%%%%%%%%%%%%%%%%%%%
\section{HRC Collaboration}

\subsection{joint action}
\subsection{Whole architecture to work}
\subsection{execution policy, leader follower?}


%%%%%%%%%%%%%%%%%%%%%%%%%%%%%%%%%%%%%%%%%%%%%%%%%%%%%%%%%%%%%%%%%%%%%%%%%%%%%%%%%%%%%%
\section{Navigation}

\subsection{navigation techniques (global + local planning)}
\subsection{cohan}
\subsection{benchmark, pedsim, etc..}


%%%%%%%%%%%%%%%%%%%%%%%%%%%%%%%%%%%%%%%%%%%%%%%%%%%%%%%%%%%%%%%%%%%%%%%%%%%%%%%%%%%%%%
\section{Modeling Human agent}

\subsection{Task modeling / HTN}

%%%%%%%%%%%%%%%%%%%%%%%%%%%%%%%%%%%%%%%%%%%%%%%%%%%%%%%%%%%%%%%%%%%%%%%%%%%%%%%%%%%%%%
\section{Task Planning}

\subsection{Various techniques}
\subsection{Offline}
\subsection{Online}
\subsection{Human-aware task planning}
Human-Aware Task Planning for a robot refers to the process of considering the presence and behavior of humans in the planning and execution of robot tasks. It involves taking into account cues from the shared environment and the dynamics of human-robot interaction. The goal is to generate robot policies that are adaptable, robust, and efficient in crowded and dynamic environments. This field includes having an explicit shared task or common goal between the human and the robot, implying that the two agents will collaborate to reach the goal. It also includes not having an established shared goal, and thus, is closer to what I would call an interaction instead of a collaboration. Yet, both problems are interesting and must be addressed, with a unified approach or not. 

Several works have been done in human-aware task planning, using various techniques.
\textbf{Give some general approaches considering the human in robot planning. HA task planner estimating human behavior (planning for both) details more in chapter 2}

