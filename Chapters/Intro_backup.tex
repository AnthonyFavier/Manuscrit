\chapter*{Introduction}
\addstarredchapter{Introduction}
\markboth{Introduction}{}

Presently motion planning has applications in several fields \cite{latombe1999motion} like robotics, animation, video games, architectural design and the study of biological molecules. Nevertheless, it has a special place in robotics as it is majorly used and developed by roboticists to navigate a robot to a destination (or goal) or move a robotic manipulator from one point to another. Classical motion planning deals with the path and trajectory planning of a manipulator or mobile robot to go from the start position to the goal position avoiding the static and dynamic obstacles in the environment. Particularly, the 2D mobile robot navigation planning generally assumes the presence of a predefined map of the environment in which the robot moves. Therefore, the static obstacles in the map are avoided using path planning and the dynamic obstacles are avoided using trajectory planning or continuous path re-planning. Several methodologies are proposed by roboticists for robot navigation and they are promising as long as the humans are not considered. Treating humans as dynamic obstacles can solve the robot navigation problem, but does not offer an acceptable and comfortable solution to the humans co-existing in the environment. 

As humans are social beings and have certain notions and expectations about the environment around them, they expect the robot to respect their personal space and preferences while navigating. Consequently, a new field of navigation called Human-Aware (Robot) Navigation (or social robot navigation) has emerged, combining studies on humans and robot navigation. Human-Aware Robot Navigation originated in the field of \acrfull{hri} rather than robot navigation, and hence it shares some properties with HRI. It was not an independent field of research until recent years and used to be a part of the overall task that the robot needs to do in order to assist or coordinate with a human or humans. As time progressed, robots have become cheaper and are now being used in many indoor \cite{guldenring2020learning}, as well as outdoor \cite{ferrer2013social} settings, solely to move from one place to another to deliver things \cite{bogue2016growth} or to accompany a person \cite{repiso2017line}. The advent of autonomous vehicles \cite{rasouli2019autonomous} further soared the interest in this field. This thesis further explores human-aware navigation and presents a cooperative framework for robot navigation that is built on the principles of HRI. We also present some new ideologies that improve robot navigation and propose some new metrics for evaluation.     

% Quoting Issac Asimov's first law, ``A robot may not injure a human being or, through inaction, allow a human being to come to harm''. 

\section*{Human-Aware Robot Navigation}
Human-aware navigation (HAN) is a special case of robot navigation where the path planning or trajectory planning or both integrate humans into planning to generate paths and/or trajectories for the robot such that it reduces the discomfort to the humans while navigating. This could mean that the navigation motion executed by the robot should be legible and acceptable to the humans sharing the environment with the robot. The term discomfort could be ambiguous and could refer to different things in different settings. Hence it is important to understand and analyse the HRI scenario at hand to take appropriate action.     
\subsection*{Relation to HRI}

As mentioned previously, HAN developed as a part of HRI research and therefore, it inherits some of the properties of HRI like the principles of joint-action \cite{curioni2019joint} which include sharing a common perspective, coordinating, predicting others' contributions and communicating. In terms of human-aware robot navigation, these can be seen as:
%%%%%%%%%%%Paper%%%%%%%%%%%%%%%%%
% https://www.researchgate.net/publication/328209329_Joint_Action_in_Humans_A_Model_for_Human-Robot_Interaction
%%%%%%%%%%%Paper%%%%%%%%%%%%%%%%%
\begin{itemize}[leftmargin=*]
    \item \textit{Share a common perspective}: The robot should know the social norms and expectations from a human in a given environment. For example moving on the right, avoid collisions with humans and other objects etc.
    \item \textit{Coordination}: Robot and human should coordinate and co-operate with each other to reach their navigation goals. For example, robot should give way for the human to pass in a door crossing scenario instead of blocking.
    \item \textit{Predict others' contribution}: The robot should predict humans' motion (trajectory) and/or intentions to know how much the human is willing to contribute to the navigation task. For example, in a corridor, the robot facing a human can either continue with or change its path depending on whether the human is willing to change his/her path or not. 
    \item \textit{Communication}: The robot should be able to communicate to humans what action it is going to take or sometimes take permission or inform humans before taking an action. Communication in HAN can be split into two types, 1) Communicating navigation intention (for example, signals) and 2) Communication through speech or video to negotiate in a complex scenario.
\end{itemize}
These principles impose more restrictions on the robot's trajectory and require the inclusion of decision-making capabilities into robot navigation planning. This makes HAN a complicated problem to solve in the motion planning paradigm. 

\subsection*{Situation Analysis and Proactive Planning}
When a robot navigates around humans, it is important for a HAN planning system to analyse the situation of the robot to take a proper action that mitigates any deadlocks or freezing robot problem~\cite{trautman2010unfreezing}. This requires the HAN system to have decision-making capabilities on top of legible motion generation. Once a situation is identified, it may be possible that the current version (or parameters) of the navigation planning system cannot avoid the occurrence of a conflict. Therefore, it is required for a HAN to have different modalities and switch between depending on the context. 

Proactive planning could mean controlling and mitigating a situation at hand instead of responding to it when it happens. This allows the robot to respond quickly and minimizes the occurrence of the robot freezing problem. Moreover, this kind of planning complements situation analysis and lessens the burden on the decision-making system. Therefore, proactive planning offers a better framework to address HAN compared to reactive planning schemes like Social Force \cite{ferrer2013social}. This thesis explores how proactive planning combined with situation analysis can benefit HAN planning and then develop a HAN system capable of handling multiple human-robot navigation contexts. 

\section*{Our Contributions}
This thesis has four main contributions, which are briefly explained below. The first three contributions can be seen as three different versions of a HAN planning system in chronological order, and the new version inherits almost all the properties of the previous version with some exceptions. The core idea of all these contributions is to offer a better solution to HAN planning, respecting the principles of joint-action in HRI and assuming humans as a partner in navigation who can co-operate. The final contribution is a set of new metrics for HAN that are more relevant to multiple human-robot navigation contexts than the ones based on proxemics.   

\subsection*{Proactive Planning and Situation Assessment in HAN}
This is the first major contribution of this thesis that combines situation assessment with proactive planning. The idea of proactive planning in our HAN system is to actively plan for the robot and the other agents involved in the navigation, assuming a possible goal for each agent while controlling only the robot. The advantage of this kind of planning is that the planning system considers both the robot and the humans while planning for the robot, which makes the robot act proactively in many situations and avoid conflicts. In deadlock scenarios, this system elicits plans for all the agents, which, if followed, will resolve the deadlock. However, the deadlock scenarios cannot be solved by this kind of proactive planning. So, we introduce a simple situation assessment to detect such deadlocks and switch the planning modality with a different set of parameters that can resolve the deadlock. Apart from these, we introduce two new human-aware constraints (or social norms) to our HAN system that increase the legibility of the robot's motion. Further, we add a new goal prediction scheme for humans that improves the human planning part of proactive planning. The proposed architecture is tested under several simulated scenarios that highlight the importance of proactive planning and the new human-aware constraints. The use of situation assessment to solve the deadlocks is shown using both simulated as well as real-world experiments.

\subsection*{A HAN system to address multi-context navigation}
% After introducing the situation assessment and modality shifting into HAN, the system has been extended to take care of the different types of visible humans in the environment. To make the system scalable and more pertinent to real-world applications, an emulated visibility has been introduced into the system while limiting the proactive planning to the two nearest (moving) humans. For static humans, a new set of costmap layers are proposed that reduces the surprise appearance of the robot from behind apart from ensuring safety. To handle the interactions involving dynamic humans, a new set of human-aware constraints are introduced, and the situation assessment is improved by adding one more modality to the HAN system. The most interesting thing about the proposed system is that the parameters of the system are highly tunable. Depending on the type of goal prediction (for humans) selected and the allowed thresholds of various human-aware constraints, the robot's behaviour could be adjusted to handle different human-robot navigation contexts. This entire system is built over the ROS navigation stack and can easily be deployed on any real-world robot with minor changes. This part of the thesis presents this system in detail, along with a set of experiments and results in various human-robot contexts (simulated and real) that show the effectiveness of the system.

After introducing the situation assessment and modality shifting into HAN, the system has been extended to take care of the different types of visible humans in the environment. Numerous changes were made to make the system scalable and more pertinent to real-world applications. For static humans, a new set of costmap layers are proposed that reduces the surprise appearance of the robot from behind apart from ensuring safety. To handle the interactions involving dynamic humans, a new set of human-aware constraints are introduced, and the situation assessment is improved by adding one more modality to the HAN system. The most interesting thing about the proposed system is that the parameters of the system are highly tunable. Depending on the type of goal prediction (for humans) selected and the allowed thresholds of various human-aware constraints, the robot's behaviour could be adjusted to handle different human-robot navigation contexts. This entire system is built over the ROS navigation stack and can easily be deployed on any real-world robot with minor changes. This part of the thesis presents this system in detail, along with a set of experiments and results in various human-robot contexts (simulated and real) that show the effectiveness of the system.

\subsection*{Proactively addressing unseen humans in HAN}
The final version of the proposed system introduces a very new concept to HAN called the `invisible humans'. The intention behind this work is to address the possible future appearances of humans from the occluded or hidden regions of the environment in the navigation scene. Firstly, an algorithm to detect the locations of `invisible humans' in a 2D map is proposed. Then a new human-aware constraint for such unseen humans is proposed and added to our HAN system that makes the robot cautiously mitigate such locations avoiding deadlocks. However, this new constraint prevents the robot from entering doors and narrow passages by continuously anticipating the emergence of humans, inducing a new freezing robot problem. To address this, a door/passage detection scheme is added to the situation assessment of our HAN and a new modality is introduced that moves the robot cautiously while avoiding the freeze. Finally, the `invisible humans' detection algorithm is extensively tested, and the updated HAN system is validated in simulated and real-world scenarios that clearly show how this system proactively handles sudden human appearances.


\subsection*{New metrics for HAN}
The last contribution of this thesis is a set of metrics that can be applied to several human-robot navigation contexts. Unlike the proxemics based metrics, we believe that the proposed metrics are better suited to understand and evaluate HAN systems. \textcolor{blue}{More explanation needs to be added later}. 


\section*{Thesis Organisation}
A major part of this thesis is based on the published work (core publications). The chapters based on the published work are elaborated compared to papers, including more details and discussions. The supportive publications include the work that is complementary during the development of this thesis, and these are detailed in the Appendix part.

This thesis has six chapters which can be grouped into three different parts. In the first group, Chapters 1 and 2 provide the necessary background and related works that helped to build this thesis. Chapter 1 specifically presents details on robot navigation planning, HAN planning and the mathematical tools that are used to build the proposed system in this thesis. Chapter 2, on the other hand, discusses the related work. The following chapters are based on the core publications and form the second group. These chapters are presented in chronological order of the development of the proposed HAN system. Hence, Chapter 3 talks about combining situation assessment in HAN, followed by Chapter 4, in which a complete HAN system that can address multiple human-robot navigation contexts is presented. Next, Chapter 5 introduces the concept of `invisible humans' to HAN and talks about proactive avoidance of potential future collisions. Throughout these three chapters, the advantages of proactive planning in combination with situation assessment are discussed in different settings. The last group has only one chapter, Chapter 6, which proposes a set of new metrics for HAN and are validated in various conditions.

The final remarks, lessons learnt, and future perspectives are discussed in the Conclusions chapter. The supportive work presented in Appendix A shows how an intelligent human agent is developed for the case of HAN. Further, different methodologies employed to simulate human agents for testing our HAN system are also presented. Throughout this thesis, whenever we refer to robot navigation, it is always a mobile robot with either differential or omnidirectional drive navigating on a 2D plane.

\subsection*{List of Publications}
\markright{List of Publications}
\subsubsection*{Published : Core Publications}
\begin{itemize}
    \item Singamaneni, P. T., \& Alami, R. (2020, August). HATEB-2: Reactive Planning and Decision making in Human-Robot Co-navigation. In 2020 29th IEEE International Conference on Robot and Human Interactive Communication (RO-MAN) (pp. 179-186). IEEE.* \let\thefootnote\relax\footnotetext{*Nominated for Best Paper award}
    
    \item Singamaneni, P. T., Favier, A., \& Alami, R. (2021, October). Human-Aware Navigation Planner for Diverse Human-Robot Interaction Contexts. In 2021 IEEE/RSJ International Conference on Intelligent Robots and Systems (IROS) (pp. 5817-5824). IEEE.
    
    \item Singamaneni, P. T., Favier, A., \& Alami, R. (2022, May). Invisible Humans in Human-aware Robot Navigation. In Social Robot Navigation: Advances and Evaluation Workshop of IEEE International Conference on Robotics and Automation (ICRA 2022).

    \item Singamaneni, P. T., Favier, A., \& Alami, R. (2022, October). Watch out! There may be a Human Addressing Invisible Humans in Social Navigation. In 2022 IEEE/RSJ International Conference on Intelligent Robots and Systems (IROS). IEEE.

\end{itemize}
\subsubsection*{Published : Supportive Publications}
\begin{itemize}

\item Favier, A., Singamaneni, P. T., \& Alami, R. (2021, July). Simulating Intelligent Human Agents for Intricate Social Robot Navigation. In RSS Workshop on Social Robot Navigation 2021.

\item Favier, A., Singamaneni, P. T., \& Alami, R. (2022, March). An Intelligent Human Avatar to Debug and Challenge Human-aware Robot Navigation Systems. In Proceedings of the 2022 ACM/IEEE International Conference on Human-Robot Interaction (pp. 760-764).
\end{itemize}

\subsubsection*{Published : Other Publications}
\begin{itemize}

\item Singamaneni, P. T., Mayima, A., Sarthou, G., Sallami, Y., Buisan, G., Belhassein, K., ... \& Clodic, A. (2020, March). Guiding Task through Route Description in the MuMMER Project (Video Submission). In HRI'20: ACM/IEEE International Conference on Human-Robot Interaction (pp. 643-643). ACM.

\item Truc, J., Singamaneni, P. T., Sidobre, D., Ivaldi, S., \& Alami, R. (2022, May). KHAOS: a Kinematic Human Aware Optimization-based System for Reactive Planning of Flying-Coworker. In ICRA 2022-IEEE International Conference on Robotics and Automation 2022.
\end{itemize}

\subsubsection*{Submitted}

\begin{itemize}
\item Mayima, A., Sarthou, G., Buisan, G., Singamaneni, P., Sallami, Y., Belhassein, K., Waldhart, J., Clodic, A., \& Alami, R. Direction-giving considered as a Human-Robot Joint Action. Submitted to \textit{User Modeling and User-Adapted Interaction (UMUAI) Journal}.
\end{itemize}