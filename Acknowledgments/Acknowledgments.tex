\section*{Remerciemments}
 

% General thèse, LAAS

% Jean Charles Fabre: 
%     - m'a aidé pour divers pb pendant l'école
%     - M'a proposé un stage qui m'a fait découvrir le LAAS 
%     - Puis m'a finalement recommandé à Rachid Alami qui m'a lui même permis d'obtenir un stage au Japon ainsi qui qu'une thèse au dans l'équipe RIS
%     - Le covid a changé mes plans, le japon est tombé à l'eau, et j'ai fini par faire un stage avec Rachid Alami qui est finalement devenu d'un manière un peu flou le début de ma thèse


Le Docteur qui sort de cette aventure est bien différent de l'étudiant qui l'a commencé. Car oui, une thèse c'est une aventure dans laquelle on s'engage en quête de découverte par curiosité et passion.
Mais on s'engage également sans savoir où cela nous mène, sans savoir ce que l'on va trouver, sans savoir si l'on va dans la bonne direction, et sans savoir si l'on arrivera au bout.
Finalement et soudainement, cette aventure se termine pour moi, et je comprends que ça en valait la peine. 
Cette expérience m'a fait grandir, à la fois sur le plan scientifique, mais aussi et surtout sur le plan personnel.
J'ai énormément appris et j'ai rencontré des personnes formidables.
Ainsi, je souhaite remercier tous ceux qui ont participé de près ou de loin à mon aventure.


% RACHID %

Tout d'abord, je souhaite remercier mon directeur de thèse Rachid Alami qui a joué un rôle crucial dans ce parcours.
Merci de m'avoir fait confiance, que ce soit en mes compétences, mes idées, ma technique ou mon écriture.
Nous avons très vite été sur la même longueur d'onde ce qui a rendu nos échanges et notre travail assez naturel concernant la direction à prendre.
Ainsi, merci pour ces discussions et nombreuses idées qui elles même en font naître bien d'autres permettant toujours d'aller plus loin. 
Merci pour tes remarques et critiques, toujours pertinentes, qui permettent d'affiner et toujours mieux faire.
Merci pour ta disponibilité, qu'il soit midi ou minuit, que tu sois au labo ou à l'autre bout du monde, tu trouves toujours un moment pour répondre aux emails de dernière minute.
Merci d'avoir permis tous ces voyages en conférence, j'ai pu découvrir des lieux mémorables comme Barcelone, Philadelphie, Washington, Prague ou encore Busan ! 
Et enfin merci pour toutes les opportunités que tu m'as permis d'avoir.


% EVA %

Je tiens ensuite à remercier ma compagne Eva qui a également joué un rôle essentiel. 
Je me suis redécouvert en te rencontrant. Grâce à toi je me sens épanoui, naturel et complet. C'est grâce à toi que j'avais la force de me lever tous les jours et de continuer à avancer. 
Merci pour ta patience et de m'avoir supporté durant ces années où ma thèse était en permanence dans un coin de ma tête.  
Merci pour tes petites attentions et considérations. 
Merci d'avoir accepté que je ne sois pas toujours présent, que ce soit en soirée, en weekend ou en vacances. 
Malgré les dures épreuves que tu as traversées, tu as su être présente et m'accompagner dans cette aventure quand il le fallait.
Je t'en remercie du fond du cœur.
Sans toi je n'aurai pas eu la force ni le mental de fournir le travail que j'ai proposé, ni d'aller au bout de ce périple. 
Merci d'être avec moi et de faire partie de ce que je suis. 


% FAMILLE & AMIS %

Un grand merci aussi à ma famille, mes parents, ma sœur, mes grand-parents, mon arrière-grand-mère, mes tantes et oncles, mes cousines et cousins, qui m'ont tous suivis, supportés et complimentés pendant cette aventure. Vous m'avez tous donné du courage et une raison d'aller au bout de ce périple. Merci à tous.

Merci également à tous mes amis, que ce soit pour du gaming ou boire un coup, vous m'avez aidé à souffler un coup pour pouvoir mieux continuer d'avancer. Merci pour ça.


% TRUFFE %

Merci aussi à Truffe, ma petite boule de poil et d'énergie adorée. Tu ne sais pas lire et ne comprendra jamais ces mots, mais tu as égayé chacune de mes journées, que ce soit en photo, en câlin ou en faisant du ``nipôtekoi''.


% COLLEGUES DEBUT %

Je souhaite ensuite remercier tous ceux du labo qui m'ont accueilli dans l'équipe lors de mes débuts. Merci notamment à ceux que j'ai vu partir au cours de mon aventure comme Amandine, Antoine, Yannick, Kathleen, et Guilhem.

Un merci particulier à toi Guilhem pour ton legs d'HATP/EHDA et le temps que tu m'as consacré pour expliquer son fonctionnement. 
Cette approche m'a beaucoup inspirée, et sans vraiment le savoir, tu m'as fourni les fondations de la majorité de ma thèse. C'est aussi ça la recherche, avancer pas par pas, construire brique après brique. D'une certaine manière, tu m'as donné la première brique sur laquelle j'ai pu en poser beaucoup d'autres, et donc je t'en remercie.

Merci à ceux qui sont encore là comme toi Phani, tu es surement la première personne avec qui j'ai collaboré dans l'équipe. Merci de m'avoir guidé dans mes débuts et de m'avoir permis de me faire ma place. Ça a été un plaisir de travailler avec toi. C'est aussi avec toi que j'ai écrit mes premiers articles et c'est grâce à tes conseils et relectures que j'ai pu apprendre à en écrire d'autres. 

Merci aussi à toi Guillaume. 
De par ton costume et ton tact, tu es assez intimidant au premier abord. 
Mais c'est justement car tu n'as pas ta langue dans ta poche, allié à ta vision juste et critique ainsi que ta volonté sincère d'aider, que je souhaite te remercier. Une brève discussion ou question suffit pour que tu pointes du doigt une faiblesse dans un raisonnement ou une implémentation, permettant de se poser les bonnes questions et rechercher la bonne solution. Merci pour ça et tous tes conseils.


% COLLEGUE FIN %

Je souhaite aussi remercier ceux qui ont rejoint mon aventure en cours de route.  

Merci à toi Simon, tu resteras mon voisin d'open-space de cœur. Merci pour toutes ces rigolades à propos de tout et de rien. On a pu à la fois réfléchir à de l'algorithmique avancée, mais aussi créer une société gymnastique basée sur des personnes empilées les unes sur les autres où la hauteur offrirai des privilèges sociaux. En attendant de voir un telle société fondamentalement biaisée par les échelles, escaliers, ou encore par tes drones entubés, j'espère qu'on aura l'occasion de se recroiser, notamment sur les terres du fameux Hellfest.

Merci à toi Philippe. Merci pour ta gentillesse. Merci pour toutes ces discussions, souvent techniques, mais aussi sur des sujets très variés où ton point de vue est toujours très intéressant. Merci pour tes différents conseils vélos qui m'ont bien aidés. 
Je me suis toujours dit que tu en faisais trop avec tes backups synchronisés à ne plus en finir, mais d'expérience je peux maintenant te confirmer que j'avais tort et que tu as bien raison de continuer.   

Merci Jeremy pour ton rire communicatif et ton punch excellent dont les fruits sont le secret. 
Merci Stéphy pour ton soutien de la suprématie des Corgis. 
Merci Ilinka d'être aussi une fan Ghibli. 
Merci Laure pour ton sourire et ta gentillesse.
Merci William pour ces discussions toutous et parties de jeux de société. 
Merci Adrien pour ta disponibilité et ton aide de dernière minute qui m'ont fait gagner de nombreuses et précieuses heures.
Merci Shashank pour ton aide et ta vision singulière qui m'a fait voir les choses sous différentes perspectives et permis d'améliorer mes travaux. 

Merci également à tous les autres que j'apprécie et qui participent à la bonne ambiance présente dans ce labo. 
Merci Lou, Bastien, Smail, Léo, Rebecca, Marcel, Virgile, Maël, Valentin, Emile, et Roland. 
Et désolé à tous les derniers arrivants que je n'ai pas eus le temps d'apprendre à connaitre. 


% PERMANANT %

Je souhaite maintenant remercier les membres permanents de l'équipe qui participe à sa continuité et son fonctionnement.
Merci Simon Lacroix d'être à la tête de l'équipe.
Merci Arthur, notamment pour ton aide pour mes enseignements à l'INSA. 
Merci Félix pour ta gestion de notre précieux distributeur d'\textit{Élixir de Travail}.
Merci Aurélie pour ta gentillesse et bienveillance. 
Merci aussi à tous les autres.  


% JURY/RAPPORTEURS %

Je remercie également Julie Shah et Mohamed Chetouani d'avoir accepté d'être rapporteurs pour ma thèse et pour le temps qu'ils ont consacré à écrire leur rapport. 
Merci également au 


Julie SHASH et Mohamend Chetouani pour avoir accepté d'être rapporteurs

D'avoir accepté être membre du jury
Catherine Pelachaud
Luca Iocchi 
Simon Lacroix
Jean-Charles Fabre: particulier, qui m'as aidé pour divers problèmes en école d'ingénieur, puis m'a fait découvrir le LAAS a travers un stage dans son équipe, puis m'a recommandé à Rachid pour finalement obtenir un second stage au LAAS ainsi qu'une proposition de thèse. 


% SAPIN ETERNEL %

Merci au Sapin éternel et à ceux qui l'entretiennent de rendre le hall du batiment si joli.


% STRUCTURES %

Merci au LAAS-CNRS de m'avoir accueil et permis et travailler dans ses locaux. 

Merci à ANITI d'avoir permis de financer mon doctorat et au LAAS-CNRS d'avoir permis qu'il s'y déroule.
