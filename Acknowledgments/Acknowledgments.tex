\section*{Remerciemments}
 

% General thèse, LAAS

% Jean Charles Fabre: 
%     - m'a aidé pour divers pb pendant l'école
%     - M'a proposé un stage qui m'a fait découvrir le LAAS 
%     - Puis m'a finalement recommandé à Rachid Alami qui m'a lui même permis d'obtenir un stage au Japon ainsi qui qu'une thèse au dans l'équipe RIS
%     - Le covid a changé mes plans, le japon est tombé à l'eau, et j'ai fini par faire un stage avec Rachid Alami qui est finalement devenu d'un manière un peu flou le début de ma thèse


Le Docteur qui sort de cette aventure est bien différent de l'étudiant qui l'a commencé. Car oui, une thèse c'est une aventure dans laquelle on s'engage en quête de découverte par curiosité et passion.
Mais on s'engage également sans savoir où cela nous mène, sans savoir ce que l'on va trouver, sans savoir si l'on va dans la bonne direction, et sans savoir si l'on arrivera au bout.
Finalement et soudainement, cette aventure se termine et l'on comprend que ça en valait la peine. 
Cette expérience m'a fait grandir, à la fois sur le plan scientifique, mais aussi et surtout sur le plan personnel.
J'ai énormément appris et j'ai rencontré des personnes formidables.
Ainsi, je souhaite remercier tous ceux qui ont participé de près ou de loin à mon aventure.

Tout d'abord, je souhaite remercier mon directeur de thèse Rachid Alami qui a joué un rôle crucial dans ce parcours.
Merci de m'avoir fait confiance, que ce soit en mes idées, ma technique ou mon écriture.
Nous avons très vite été sur la même longueur d'onde concernant le sujet, ce qui a rendu nos échanges et notre travail assez naturel concernant la direction à prendre.
Ainsi, merci pour ces discussions et nombreuses idées qui elles même en font naître d'autres permettant toujours d'aller plus loin. 
Merci pour tes remarques et critiques, toujours pertinentes, qui permettent d'affiner et toujours mieux faire.
Merci pour ta disponibilité, qu'il soit midi ou minuit, que tu sois au labo ou à l'autre bout du monde, tu trouves toujours un moment pour répondre aux emails de dernière minute.

Je souhaite ensuite remercier tous ceux qui m'ont accueilli et intégré lors de mes débuts. Je pense notamment à Amandine, Antoine, Yannick, Kathleen, Phani, Guilhem, Guillaume.

Un merci particulier à Guilhem pour ton legs d'HATP/EHDA et le temps que tu m'as consacré pour expliquer son fonctionnement. 
Ce système m'a beaucoup inspiré, et sans vraiment le savoir, tu m'as fourni les fondations de la majorité de ma thèse. C'est aussi ça la recherche, avancer pas par pas, construire brique après brique. D'une certaine manière, tu m'as donné la première brique sur laquelle j'ai pu en poser beaucoup d'autres, et donc je t'en remercie.

Merci également à Phani, tu es surement la première personne avec qui j'ai collaboré dans l'équipe. C'est avec toi que j'ai écrit mes premiers articles et c'est grâce à tes conseils et relectures que j'ai pu apprendre à écrire dans ce format. 

Merci aussi à Guillaume. Tu as l'habitude mais je dois le répéter, impressionnant, toujours costume, pas langue dans sa poche. Mais c'est ça qui allié à ta vision juste et critique permet lors de brève discussion de ce rendre compte que l'on avance de travers et que l'on peut mieux faire.   

Guilhem, Amandine, Antoine, Kathleen, Yannick

Phani

Shashank

Guillaume: qui 

Simon

Philippe

Jeremy

Collègues en général, liste

Permanent (Simon Lacroix, Arthur, Felix, Aurélie)



Merci à ANITI d'avoir permis de financer mon doctorat et au LAAS-CNRS d'avoir permis qu'il s'y déroule.
