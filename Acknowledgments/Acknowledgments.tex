\section*{Remerciemments}
 

% General thèse, LAAS

% Jean Charles Fabre: 
%     - m'a aidé pour divers pb pendant l'école
%     - M'a proposé un stage qui m'a fait découvrir le LAAS 
%     - Puis m'a finalement recommandé à Rachid Alami qui m'a lui même permis d'obtenir un stage au Japon ainsi qui qu'une thèse au dans l'équipe RIS
%     - Le covid a changé mes plans, le japon est tombé à l'eau, et j'ai fini par faire un stage avec Rachid Alami qui est finalement devenu d'un manière un peu flou le début de ma thèse


Le Docteur qui sort de cette aventure est bien différent de l'étudiant qui l'a commencé. Car oui, une thèse c'est une aventure dans laquelle on s'engage en quête de découverte par curiosité et passion.
Mais on s'engage également sans savoir où cela nous mène, sans savoir ce que l'on va trouver, sans savoir si l'on va dans la bonne direction, et sans savoir si l'on arrivera au bout.
Finalement et soudainement, cette aventure se termine et l'on comprend que ça en valait la peine. 
Cette expérience m'a fait grandir, à la fois sur le plan scientifique, mais aussi et surtout sur le plan personnel.
J'ai énormement appris et rencontré des personnes formidables,  
Ainsi, je souhaite remercier tous ceux qui ont participer de près ou de loin à mon aventure.

Tout d'abord, je souhaite remercier mon directeur de thèse Rachid Alami qui m'a accompagné et conseillé. Merci de m'avoir fait confiance. 
Merci pour toutes tes remarques et critiques, toujours pertinentes. 
Merci pour ces discussions et idées qui permettent toujours d'aller plus loin et mieux faire, mais qu'il faut aussi savoir freiner de temps de temps pour éviter de s'y perdre.


Guilhem, Amandine, Antoine, 

Phani

Shashank

Guillaume,

Simon

Philippe 

Collègues en général, liste



Merci à ANITI d'avoir permis de financer mon doctorat et au LAAS-CNRS d'avoir permis qu'il s'y déroule.
