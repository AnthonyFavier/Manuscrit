\section*{Remerciemments}
 

% General thèse, LAAS

% Jean Charles Fabre: 
%     - m'a aidé pour divers pb pendant l'école
%     - M'a proposé un stage qui m'a fait découvrir le LAAS 
%     - Puis m'a finalement recommandé à Rachid Alami qui m'a lui même permis d'obtenir un stage au Japon ainsi qui qu'une thèse au dans l'équipe RIS
%     - Le covid a changé mes plans, le japon est tombé à l'eau, et j'ai fini par faire un stage avec Rachid Alami qui est finalement devenu d'un manière un peu flou le début de ma thèse


Le Docteur qui sort de cette aventure est bien différent de l'étudiant qui l'a commencé. Car oui, une thèse c'est une aventure dans laquelle on s'engage en quête de découverte par curiosité et passion.
Mais on s'engage également sans savoir où cela nous mène, sans savoir ce que l'on va trouver, sans savoir si l'on va dans la bonne direction, et sans savoir si l'on arrivera au bout.
Finalement et soudainement, cette aventure se termine pour moi, et je comprends que ça en valait la peine. 
Cette expérience m'a fait grandir, à la fois sur le plan scientifique, mais aussi et surtout sur le plan personnel.
J'ai énormément appris et j'ai rencontré des personnes formidables.
Ainsi, je souhaite remercier tous ceux qui ont participé de près ou de loin à mon aventure.


% RACHID %

Tout d'abord, je souhaite remercier mon directeur de thèse Rachid Alami qui a joué un rôle crucial dans ce parcours.
Merci de m'avoir fait confiance, que ce soit en mes compétences, mes idées, ma technique ou mon écriture.
Nous avons très vite été sur la même longueur d'onde ce qui a rendu nos échanges et notre travail assez naturel concernant la direction à prendre.
Ainsi, merci pour ces discussions et nombreuses idées qui elles même en font naître bien d'autres permettant toujours d'aller plus loin. 
Merci pour tes remarques et critiques, toujours pertinentes, qui permettent d'affiner et toujours mieux faire.
Merci pour ta disponibilité, qu'il soit midi ou minuit, que tu sois au labo ou à l'autre bout du monde, tu trouves toujours un moment pour répondre aux emails de dernière minute.
Merci d'avoir permis tous ces voyages en conférence, j'ai pu découvrir des lieux mémorables comme Barcelone, Philadelphie, Washington, Prague ou encore Busan ! 
Et enfin merci pour toutes les opportunités que tu m'as permis d'avoir.


% EVA %

Je tiens ensuite à remercier ma compagne Eva qui a joué un rôle essentiel. 
Je me suis redécouvert en te rencontrant. Grâce à toi je me sens épanoui, naturel et complet. C'est grâce à toi que j'avais la force de me lever tous les jours et de continuer à avancer. 
Merci de m'avoir supporté durant ces années où ma thèse était en permanence dans un coin de ma tête.  
Merci pour tes petites attentions et considérations. 
Merci d'avoir accepté que je ne sois pas toujours présent, que ce soit en soirée, en weekend ou en vacances. 
Malgré les dures épreuves que tu as traversées, tu as su être présente et m'accompagner dans cette aventure quand il le fallait.
Je t'en remercie du fond du cœur.
Sans toi je n'aurai pas eu la force ni le mental de fournir le travail que j'ai proposé, ni d'aller au bout de ce périple. 
Merci d'être avec moi et de faire partie de ce que je suis. 


% JURY/RAPPORTEURS %

Julie SHASH et Mohamend Chetouani pour avoir accepté d'être rapporteurs

D'avoir accepté être membre du jury
Catherine Pelachaud
Luca Iocchi 
Simon Lacroix
Jean-Charles Fabre: particulier, qui m'as aidé pour divers problèmes en école d'ingénieur, puis m'a fait découvrir le LAAS a travers un stage dans son équipe, puis m'a recommandé à Rachid pour finalement obtenir un second stage au LAAS ainsi qu'une proposition de thèse. 

% COLLEGUES DEBUT %

Je souhaite ensuite remercier tous ceux qui m'ont accueilli dans l'équipe lors de mes débuts. Merci notamment à ceux à ceux que j'ai vu partir au cours de mon aventure comme Amandine, Antoine, Yannick, Kathleen, et Guilhem.

Un merci particulier à toi Guilhem pour ton legs d'HATP/EHDA et le temps que tu m'as consacré pour expliquer son fonctionnement. 
Ce système m'a beaucoup inspiré, et sans vraiment le savoir, tu m'as fourni les fondations de la majorité de ma thèse. C'est aussi ça la recherche, avancer pas par pas, construire brique après brique. D'une certaine manière, tu m'as donné la première brique sur laquelle j'ai pu en poser beaucoup d'autres, et donc je t'en remercie.

Merci à ceux qui sont encore là comme toi Phani, tu es surement la première personne avec qui j'ai collaboré dans l'équipe. Merci de m'avoir guidé dans mes débuts et de m'avoir permis de me faire ma place. C'est aussi avec toi que j'ai écrit mes premiers articles et c'est grâce à tes conseils et relectures que j'ai pu apprendre à en écrire d'autres. 

Merci aussi à toi Guillaume. 
Tu le sais très bien, mais il est vrai que de par ton costume et tes remarques, tu es assez intimidant au premier abord. 
Mais c'est justement car tu n'as pas ta langue dans ta poche, allié à ta vision juste et critique, que je souhaite te remercier. Une brève discussion ou question suffit pour que tu pointes du doigt une faiblesse dans un raisonnement ou une implémentation, permettant de se poser les bonnes questions et rechercher la bonne solution. Merci pour tous tes conseils.


% COLLEGUE FIN %

Je souhaite remercier tous ceux qui ont rejoint mon aventure en cours de route.   
Simon: restera toujours mon collège d'open space, rigolade, hellfest (hate de se croiser sur les terres du hellfest), entubage du dronage.  
Philippe: discussion vélo, backupeur fou mais qui avait raison
Jeremy: Ton rire communicatif, et évidement donc punch, ou plus specifiquement, les fruits qui baignent au fond. 
Stéphy: Discussion toutou et corgi 
Ilinka: Ghibli
William: Toutou et jeux de société 

Shashank: De par ton parcours, ta vision assez différente de la mienne du problème m'a permis de me poser des questions différentes et pertinentes qui ont aidé à améliorer mes travaux.  

Adrien: Le sauveteur. Ma ouvert la porte un matin de pluie prévenu à la dernière minute pour diagnostiquer et réparer mon pc qui m'a lacher une semaine avant de rendre mon manuscrit. Ta disponibilité et ton aide m'a fait gagner de nombreuses et précieuses heures, donc merci.  

Merci également à tous les autres que j'apprécie et qui participent à la bonne ambiance présente dans ce labo. Merci à
Laure, Lou, Bastien, Smail, Léo, Rebecca, Marcel, Virgile, Maël, Valentin, Emile, et Roland.


% TRUFFE %

Merci aussi à Truffe, ma petite boule de poil et d'énergie adorée. Tu ne sais pas lire et ne comprendra jamais ces mots, mais tu as égayé chacune de mes journées, que ce soit en photo, en câlin ou en faisant du ``nipôtekoi''.


% PERMANANT %

Merci aux permanents qui permettent de faire tourner l'équipe.   

Simon Lacroix: responsable équipe, organisation séminaire, 
Arthur: 
Felix: café, ou "Élixir de travail", sans ça et donc sans Félix on n'aurai pas les même conditions de travail. 
Aurélie: Gentillesse et bienvaillance. Bien guidé au debut merci. 


% FAMILLE %

Famille


% AMIS %

Amis

% ANITI %

Merci à ANITI d'avoir permis de financer mon doctorat et au LAAS-CNRS d'avoir permis qu'il s'y déroule.
