\chapter{User Study results}
\label{ap:study}

This appendix provides more details and numbers about the User Study described in chapter~\ref{chap:6}.

\section{Participants information}

\textbf{TODO: give information gathered, remove names to anonymize, only participant id.}

\section{Questionnaire answers}

\textbf{TODO: Add questionnaire answers (12 answer X 6 scenarios)}

\section{Execution metrics extracted}

\textbf{TODO: Add Execution metrics. Will be huge, specific format or just several pages?}

\section{Participants comments}

This section gathers the participants' comments about the experiment. The comments have been transcribed using the participant's words as much as possible and then translated as most of them were in French. They are listed below, classed by subject, and redundant comments are combined in one line.

\textbf{TODO: remove that and replace with actual comment, not translated.}

\textbf{SIMULATION}
\begin{itemize}
\setlength\itemsep{-0.3em}
\item Good overall experience.
\item Simulation is clear, simple, pleasant, intuitive, captivating, and funny.
\item It's like a video game.
\item I feel committed and active.
\end{itemize}

\textbf{DISPLAY}
\begin{itemize}
\setlength\itemsep{-0.3em}
\item Robot is well seen
\item There is too much information on the screen (goal + scene + text prompt).
\item The text prompt was hard to read.
\end{itemize}

\textbf{GENERAL}
\begin{itemize}
\setlength\itemsep{-0.3em}
\item Robot giving indications and guidance would be great.
\item Sometimes I made mistakes and could have acted better.
\item I didn't see much difference between HF and RF.
\item At first, the difference between HR and RF is unclear but then ok.
\item There is no scenario that I prefer.
\item I dislike robot taking initiative because it might be wrong.
\end{itemize}

\textbf{STEPS}
\begin{itemize}
\setlength\itemsep{-0.3em}
\item Steps set the tempo and synchronize us which is useful.
\item Steps are confusing and frustrating. I have to wait before acting again.
\end{itemize}

\textbf{TASK}
\begin{itemize}
\setlength\itemsep{-0.3em}
\item In simple tasks human thinks they know better what to do, thus the robot should follow human decisions (the notion of hierarchy has been mentioned).
\item Using common resources first is really a wrong decision. 
\item Performing the last action is satisfying, and the robot adapting to allow it is great.
\item The fact that both agents must perform actions makes the collaboration relevant and useful. 
\item The task is clear.
\item Having to drop a cub (pink bar) on the table is unpleasant.
\item In (5) I felt a double satisfaction: the task is achieved, and my objective is fulfilled.
\item In my daily life I would prefer the robot doing time-consuming tasks, thus, I like the HFE objective.
\item I felt a lot of emotions (satisfaction/frustration), so a more practical and real task would enhance these feelings. 
\end{itemize}


\textbf{Robot-First regime}
\begin{itemize}
\setlength\itemsep{-0.3em}
\item RF can be very frustrating, confusing, constraining, unpredictable, inefficient, and adversarial.
\item RF can be great for finishing the task fast.
\item RF isn't good for freeing the human.
\item RF doesn't consider my objective/preferences.
\item RF with correct estimation is great, fast, efficient, and less cognitively demanding.
\item With RF, the robot acts first but still adapts to my action in the next step.
\item With RF I'm more focused on preventing the robot's mistakes than on the actual task.
\item It is very unpleasant when the robot steals cubes I wanted to pick.
\item with RF we are not in control, even if RF is doing well the lack of control is uncomfortable.
\item I disliked 
\item RF robots felt like my enemies.
\end{itemize}

\textbf{Human-First regime}
\begin{itemize}
\setlength\itemsep{-0.3em}
\item I enjoyed HF robots.
\item HF able to fulfill my objectives.
\item I prefer to be in control and have the robot as a follower.
\item I feel in control with HF.
\item HF adapts to my actions.

\item Once used to it, HF is smooth and efficient for completing the stack quickly.
\item HF is more efficient than RF depending on the objective.
\item HF is sometimes inconsistent (pink bar in (5)).
\item HF is slower than RF.
\item HF is more interesting and predictable than RF.
\item HF is less efficient and frustrating than RF.

\item HF makes less wrong choices.
\item I enjoy being able to predict the robot's behavior/actions.
\end{itemize}

\textbf{ACTIONS}
\begin{itemize}
\setlength\itemsep{-0.3em}
\item Not being able to pick cubes in advance isn't natural, confusing, frustrating, and a bit complicated at first. But we get used to it quite fast.
\item I felt obliged to act at every step.
\item The objective of "trying to be free early" is a bit frustrating since I would like to act even when not necessary. It's hard to consider this objective as my personal preference.
\item The actions feel stiff and unnatural.
\item Robot movements seem real.
\item The lack of collision with cubes is unrealistic but ok.
\end{itemize}

\begin{sidewaystable}[]
    \begin{tabular}{|c|l|}
    \hline
    ID & \multicolumn{1}{c|}{Comments}                                                                                                                                                                                                                                                                                                                                                                                                                                                                                                                                                                                                                          \\ \hline
    1  & \begin{tabular}[c]{@{}l@{}}Quand robot défie notre logique perturbant et frustant. Globalement bien aimé. Notion de hierarchie, en fonction de la  \\ tâche (notamment simple)  on se supérieur (on sait mieux faire) et donc le robot doit nous suivre.   (2) bien pour tâche         \\  au + vite car va vite mais moins bien pour être lib,  dépend de la tâche, pire (4)\end{tabular}                                                                                                                                                                                                                                                             \\ \hline
    2  & \begin{tabular}[c]{@{}l@{}}RF est bien mais les mauvais choix sont pénible, frustrant, il ne concidère pas mon objectif/preferences. Cependant il  \\ s'adapte quand même une fois l'action faite. (2) RF trop cool, va + vite pour faire la tâche mais marche pas bien pour           \\ lib. HF est plus efficace en fonction de l'objectif. pire (4)\end{tabular}                                                                                                                                                                                                                                                                                   \\ \hline
    3  & \begin{tabular}[c]{@{}l@{}}Chrono TO stressant. ne pas pouvoir prendre les cubes à l'avance n'est pas naturel, pertubant et rend un peu compliqué  \\ mais devient simple une fois habitué. Son de fin de tâche. Se sent obligé de faire quelque chose à  chaque étape, et             \\ donc meme en RF clique sur la main pour confirmer que  je serai passif. Rappeller à la fin  de tache le regime et obj.       \\ (5) et (1)  HF  pour finir la tâche une fois habitué va relativement vite et fluide. HF pour  etre libéré aussi, apres             \\ rien a faire.\end{tabular}                                                             \\ \hline
    4  & \begin{tabular}[c]{@{}l@{}}Bon moment, Clair preference pour HF. RF est une catastrophe. En RF on  n'a pas vraiment son mot à dire. En plus, quand \\  RF commence à faire une  erreur on est plus concentré à l'empecher de continuer à faire des erreurs  plutot que sur la          \\ tâche, très frustant. Meme quand RF fait bon choix on se  sent obligé de l'ecouter et impuissant.(5) HF et lib, car          \\ fluide en controle  et j'ai pu atteindre mon objectif. Pire (6)\end{tabular}                                                                                                                                                 \\ \hline
    5  & \begin{tabular}[c]{@{}l@{}}Intuitif, marche bien, efficace. Etre libere un peu frustant car on a envie  d'agir, regarder le robot faire etre       \\ penible.. Sol: Donner une tache  auxiliaire à l'humain a faire uniquement quand se désengage de la tache  principale ?           \\ (3) HF et tache wrong. Car H agit le plus, R agit seulement  quand necessaire. De plus, le robot s'adapte. Il a              \\ anticipé si jamais je ne  prend pas le bleu en prenant le vert, mais une fois qu'il m'a vu prendre  le bleu il n'a pas             \\ pris le sien. pire (4)\end{tabular}                                                    \\ \hline
    6  & \begin{tabular}[c]{@{}l@{}}Globalement positif. Aurai aimé que le robot donne plus d'indication,  guide plus les actions. (2) vif, mieux pour la   \\ tache, mais moins prévisible.  (1) bien aussi mais plus lent/passif. pire le (6)\end{tabular}                                                                                                                                                                                                                                                                                                                                                                                                    \\ \hline
    7  & \begin{tabular}[c]{@{}l@{}}Certain scenario vraiment contraignant et frustant. Uiliser ressource  commune en 1er est vraiment un mauvaix choix.    \\ Globalement ok.  2 efficace une fois compris. Faire la dernière action est gratifiant, donc  que le robot s'adapte pour          \\ le permettre s'est positif.Preféré (2), rapide  et bon choix, pire 4\end{tabular}                                                                                                                                                                                                                                                                          \\ \hline
    \end{tabular}
    \caption{Comments from participants given after the experiment. Part 1}
    \label{tab:ap:commets_1}
\end{sidewaystable}

\begin{sidewaystable}[]
    \begin{tabular}{|c|l|}
    \hline
    ID & \multicolumn{1}{c|}{Comments}                                                                                                                                                                                                                                                                                                                                                                                                                                                                                                                                                                                                                          \\ \hline
    8  & \begin{tabular}[c]{@{}l@{}}Parfois frustant a cause des mauvais choix du robot. Les actions sont  "rigide", pas naturel (attraper cube bleu sous   \\ vert, le faisant "tomber"  le vert). Bien dans l'ensemble, le fait que chacun ai sa part rend la  collaboration                  \\ pertinente et utile. Je prefère que la priorité des choix et  actions soit à l'humain. (3) car s'adapte à ce que je          \\ fais et c'est clair.  Pire (4) un mauvais choix du robot a impliqué un mauvais choix de ma  part, frustrant                        \\  (1) simplement frustrant, le robot semble contre moi.\end{tabular}                    \\ \hline
    9  & \begin{tabular}[c]{@{}l@{}}RF pas efficace, mais follow est plus simple, moins compliqué. HF mieux mais parfois incohérent (barre rose). (1) est   \\ le plus satisfaisant. (4) est le pire\end{tabular}                                                                                                                                                                                                                                                                                                                                                                                                                                               \\ \hline
    10 & \begin{tabular}[c]{@{}l@{}}ne pas pouvoir attraper en avance un peu agaçant. Le fait de devoir regarder le but à gauche + la scène + lire le       \\ prompt text un peu complex =\textgreater mieux si robot dit quoi faire.Prefère quand le robot est "passif", dans le sens         \\  follower. Que le robot attende qu'on decide puis agisses. N'aime pas quand le robot "prend des initiatives" car             \\ possible mauvais choix:  vole les cubes très agaçant.. + prend cube commun en 1er. Pref 1 Pire 4\end{tabular}                                                                                                                \\ \hline
    11 & \begin{tabular}[c]{@{}l@{}}Bonne simu, claire. Mvt ont l'air réel tache claire et globalement se  passe bien. Les steps cadense bien, pratique     \\ Meilleur (1) pire (4)\end{tabular}                                                                                                                                                                                                                                                                                                                                                                                                                                                               \\ \hline
    12 & \begin{tabular}[c]{@{}l@{}}Temps d'attente (step) frustrant. Serait bien de pouvoir prendre avant  pour indiquer intention, donner info. HF +      \\ interessant + de control,  - efficace mais - frustrant, - imprevu et donc de mauvais choix.(+) 1  (-) 6 car on est               \\ obligé de reposer le cube.\end{tabular}                                                                                                                                                                                                                                                                                                                      \\ \hline
    13 & \begin{tabular}[c]{@{}l@{}}Interaction simple, comme un jeu video. Les mauvais choix du robot sont  assez frustrant. Simple et plaisant.(+) 3 car  \\ s'est bien adapté malgré  erreur humaine (-) 4 car vole les cubes\end{tabular}                                                                                                                                                                                                                                                                                                                                                                                                                   \\ \hline
    14 & \begin{tabular}[c]{@{}l@{}}Tres bien dans l'ensemble. Géné par l'affichage, du mal à lire. Certain  scenario efficace d'autre non. Simple, sauf    \\ lecture/texte. (+) 5 fini  rapidement, double satisfaction de finir sa part vitre puis voir la pile  fini par le robot           \\ (-) 6 degouté, frustrant\end{tabular}                                                                                                                                                                                                                                                                                                                        \\ \hline
    15 & Bien, a volé 2 fois les cubes, pas agréable. HF + facile.(+)  2 (-) 4, 3                                                                                                                                                                                                                                                                                                                                                                                                                                                                                                                                                                               \\ \hline
    16 & \begin{tabular}[c]{@{}l@{}}Simple, agréable. Le manque de collision avec cube réduit le réalisme  mais ok. Un peu confu/pertubant et un peu        \\ frustrant de devoir attendre  que robot finisse action avant d'agir à nouveau.(+) 2, RF car rapide (-) 6\end{tabular}                                                                                                                                                                                                                                                                                                                                                                            \\ \hline
\end{tabular}
    \caption{Comments from participants given after the experiment. Part 2}
    \label{tab:ap:commets_2}
\end{sidewaystable}

\begin{sidewaystable}[]
    \begin{tabular}{|c|l|}
    \hline
    ID & \multicolumn{1}{c|}{Comments}                                                                                                                                                                                                                                                                                                                                                                                                                                                                                                                                                                                                                          \\ \hline
    17 & \begin{tabular}[c]{@{}l@{}}certain scenario ok =\textgreater c'est plutot H qui a mal agit, 2 sce avec mauvais  choix de R. HF/RF pas tellement    \\ choix !=(-) 6 RF etre lib (+) 2\end{tabular}                                                                                                                                                                                                                                                                                                                                                                                                                                                     \\ \hline
    18 & \begin{tabular}[c]{@{}l@{}}pas pouvoir prendre cube en avance frustrant. En RF, R devient  prévisiblement génant, on prévoit et réfléchi pour      \\ s'adapter et  anticiper mauvais choix (defensif). HF mieux.(+) HF tache plus vite,  plus interessant que lib rapidement,         \\  ennuyeux. (-) 4, R imprévisible,  devient ennemi\end{tabular}                                                                                                                                                                                                                                                                                               \\ \hline
    19 & \begin{tabular}[c]{@{}l@{}}Globalement ok reagit bien, comprend bien ce que je faisait (pink),  avant dernier tres frustrant, vole les cubes..     \\ Sinon bien passé  (+) 2 (-) 4\end{tabular}                                                                                                                                                                                                                                                                                                                                                                                                                                                       \\ \hline
    20 & \begin{tabular}[c]{@{}l@{}}Simu claire, voit bien le R et quand il agit, interaction bonne et claire,  4 mauvais mais globalement benefique        \\ interaction. Pref au quotidien  laisser R faire les tache chronophage, donc bien aimé obj lib au plus  vite.(+) 5 puis 2         \\  (-) 4\end{tabular}                                                                                                                                                                                                                                                                                                                                          \\ \hline
    21 & R prévisible cool, on peut preshot et anticiper. (+) 1 3 (-) 4                                                                                                                                                                                                                                                                                                                                                                                                                                                                                                                                                                                         \\ \hline
    22 & \begin{tabular}[c]{@{}l@{}}Sympa, simu bien faite. interactif, on est pris dedans et acteur. Beaucoup d'émotion déjà (satisfaction / frustration)  \\ donc si c'était une tache plus concrete ça serait encore plus frustrant.(+) 1 3 (-) 4\end{tabular}                                                                                                                                                                                                                                                                                                                                                                                               \\ \hline
    23 & \begin{tabular}[c]{@{}l@{}}Simu bien faite, s'imagine bien interaction. Au debut un peu confu diff entre HF/RF puis ok. Confusion cube blanc et    \\ gris(+) aucun (-) 6\end{tabular}                                                                                                                                                                                                                                                                                                                                                                                                                                                                 \\ \hline
    24 & \begin{tabular}[c]{@{}l@{}}Interessant, jamais une gène, interaction amusante, sympa de prédire RA(+) 1, 3 (-) 6\end{tabular}                                                                                                                                                                                                                                                                                                                                                                                                                                                                                                                          \\ \hline
    25 & \begin{tabular}[c]{@{}l@{}}HF ok, qd RF on peut rien faire. Simulation moins naturelle, pas  parfaitement réaliste. Notion etape/synchronisation   \\ perturbant un peu (+) 5 HF lib (-) 4\end{tabular}                                                                                                                                                                                                                                                                                                                                                                                                                                                \\ \hline
    \end{tabular}
    \caption{Comments from participants given after the experiment. Part 3}
    \label{tab:ap:commets_3}
\end{sidewaystable}

\section{Scenario preference}

This section provides the results obtained after asking participants which scenario they preferred the most and the least. 

\begin{table}[]
    \center
    \begin{tabular}{c|c|c|c|c|c|c|}
    \cline{2-7}
                                                                                                       & \textbf{S1}         & \textbf{S2}         & \textbf{S3}         & \textbf{S4}          & \textbf{S5}         & \textbf{S6}         \\ \hline
    \multicolumn{1}{|c|}{\textbf{\begin{tabular}[c]{@{}c@{}}Times preferred \\ the most\end{tabular}}} & \textit{\textbf{9}} & \textit{\textbf{7}} & \textit{\textbf{3}} & \textit{0}           & \textit{\textbf{5}} & \textit{0}          \\ \hline
    \multicolumn{1}{|c|}{\textbf{\begin{tabular}[c]{@{}c@{}}Times preferred\\ the least\end{tabular}}} & \textit{0}          & \textit{0}          & \textit{0}          & \textit{\textbf{16}} & \textit{0}          & \textit{\textbf{9}} \\ \hline
    \end{tabular}
    \caption{Number of times each scenario has been respectively preferred the most and the least. HF scenarios are never disliked and RF scenarios with erroneous estimations are never preferred.}
    \label{tab:ap:scenario_preferred}
    \end{table}