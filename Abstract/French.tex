\chapter*{R\'esum\'e}

La navigation d'un robot mobile dans un environnement humain est appelée Human-Aware Navigation (HAN) ou parfois Social (Robot) Navigation. Nous présentons l'évolution de la HAN au fil des ans et les défis qui lui sont associés avant de procéder à l'explication de notre approche. Avec un intérêt croissant dans le domaine, de nouveaux cadres sont nécessaires pour mieux aborder ce problème, et comme contribution, nous proposons un système de planification proactive multi-contexte pour HAN. 

Le cadre de planification proactive proposé pour le HAN est basé sur l'approche Human-Aware Timed Elastic Band (HATEB). Ce cadre ajoute une boucle de décision au niveau de la planification de la trajectoire et passe d'un mode de planification à l'autre en fonction de la situation évaluée. En utilisant ce cadre, nous proposons une pile HAN basée sur ROS appelée planificateur Co-operative Human-Aware Navigation (CoHAN). CoHAN est conçu pour aborder la navigation de robot multi-contexte parmi des humains statiques et dynamiques avec des cartes de coûts mises à jour et de nouvelles contraintes sociales homme-robot. Ces contraintes d'optimisation visent à rendre le mouvement du robot lisible et à éviter les apparitions surprises par derrière. CoHAN propose quatre mécanismes différents de prédiction de trajectoire pour les humains et trois modes de planification différents, ainsi qu'un nouveau mode de récupération. L'analyse qualitative effectuée sur des scénarios simulés homme-robot met en évidence les avantages de la planification proactive et basée sur la situation. Elle montre également comment les contraintes sociales nouvellement ajoutées améliorent la lisibilité du robot. Les comparaisons quantitatives avec d'autres planificateurs HAN montrent que CoHAN peut résoudre des scénarios plus complexes de manière acceptable.

Dans la partie suivante, CoHAN est étendu pour traiter de manière proactive l'apparition soudaine d'humains dans des régions occluses de la carte. Nous proposons le concept d'“humains invisibles” pour estimer l'emplacement de telles émergences humaines. Ces estimations sont ajoutées à CoHAN via une autre contrainte d'optimisation, et un nouveau mode de planification pour passer les portes et les passages étroits avec précaution est introduit. Les expériences en simulation et sur le robot réel démontrent les avantages des “humains invisibles” dans le HAN et montrent que le robot maintient une plus grande distance avec les humains occultés par rapport aux autres approches.

Les mesures actuelles basées sur les violations de zones proxémiques ne peuvent pas rendre justice à l'évaluation de scénarios complexes. Par conséquent, nous proposons un ensemble de métriques prenant en compte les vitesses et la visibilité de l'humain, qui peuvent être pertinentes dans différents contextes de navigation homme-robot. Cette partie de la thèse présente la formulation mathématique de ces métriques, suivie de l'évaluation de quatre différents contextes de navigation homme-robot en les utilisant. Ces métriques, combinées aux profils de vitesse et aux trajectoires, distinguent clairement un planificateur HAN d'un simple planificateur de navigation. Nous concluons cette thèse par une discussion sur la contribution, suivie par les défis et les perspectives futures du HAN.




%%%%%%%%%%%%%%%%%% First version one %%%%%%%%%%%%%%%%%%%%%%%%%%
% Aujourd'hui, plus que jamais, des robots mobiles et des drones parcourent les espaces de travail et de vie des humains. En particulier, les robots mobiles sont déployés ou en voie de déploiement dans de nombreux endroits, des aéroports aux restaurants en passant par les rues. Dans un cadre classique de planification du mouvement, tout est obstacle que le robot doit éviter pour atteindre sa destination. Cependant, cette approche ne peut pas être directement utilisée pour la navigation des robots dans les environnements humains. Les humains peuvent ne pas être à l'aise de voir un robot se déplacer très près d'eux ou de ne pas savoir si le robot est prêt à leur céder le passage ou non. Si les humains ne sont pas pris en compte de manière explicite dans la planification de la navigation, le robot peut les perturber et ne pas être accepté. Ainsi, un nouveau domaine de la navigation robotique se concentrant sur ces aspects, appelé `Human-Aware Robot Navigation (HAN) (ou Social Robot Navigation)', se développe rapidement de nos jours. Ce travail explore l'approche HAN dans le cas des robots mobiles et propose quelques nouveaux facteurs et systèmes qui peuvent rendre un robot plus acceptable par les humains. 

% L'idée centrale de ce travail est que le robot doit éviter ou atténuer les interactions homme-robot inconfortables qui se produisent pendant la navigation. Ainsi, dans la première partie de la thèse, nous explorons l'évaluation de situation et la planification proactive dans HAN pour planifier des trajectoires de robot lisibles et acceptables. Nous introduisons également de nouvelles contraintes sociales homme-robot et une nouvelle méthode de prédiction de la trajectoire des humains au voisinage du robot. Le système proposé a été validé dans plusieurs contextes, et une analyse détaillée en est présentée. La partie suivante développe cette idée et propose un système HAN qui peut gérer des humains aussi bien statiques qu'en mouvement dans plusieurs circonstances. Nous proposons un système HAN basé sur la pile de navigation ROS pour résoudre le problème de la navigation multi-contexte. Ce système est hautement ajustable et possède un mécanisme de changement de modalité qui permet au robot d'adapter, en fonction du contexte, plusieurs paramètres d'interaction homme-robot. Nous introduisons des contraintes plus sensibles à l'homme concernant les normes sociales pour rendre la navigation du robot plus vive et réactive. Enfin, le système a été testé dans plusieurs scénarios simulés et réels et des analyses en sont fournies. Comparé à un système HAN déjà existant, notre système a donné des résultats meilleurs et plus satisfaisants tant sur le plan qualitatif que quantitatif. Bien que ce système puisse gérer plus d'un type de scénario avec des humains visibles par le robot, il ne peut pas traiter les apparitions soudaines d'humains ou préparer le robot à de telles occurrences. Ainsi, dans une partie suivante de la thèse, une méthodologie pour détecter de telles apparitions potentielles est proposée. Ces estimations sont ensuite intégrées au système HAN proposé précédemment afin de permettre au robot de manœuvrer avec précaution autour des lieux de ces possibles apparitions. L'algorithme proposé a été largement testé, et les avantages de cet ajout sont démontrés par plusieurs expériences. 

% Tout au long du développement de cette thèse, l'évaluation du système HAN a été un défi car il n'existe pas de métriques suffisamment bonnes et acceptée par la communauté. Par conséquent, nous avons utilisé certaines métriques existantes et en avons proposé de nouvelles qui pourraient être pertinentes dans de nombreux contextes humains-robots. La dernière partie de cette thèse présente ces nouvelles métriques et leur évaluation dans différents contextes. Enfin, nous concluons cette thèse par une discussion sur l'état actuel du domaine, les défis rencontrés au cours du développement de cette thèse et les perspectives futures.



%%%%%%%%%Previous one updated by Rachid%%%%%%%%%%%%%

% Aujourd'hui, plus que jamais, des robots mobiles et des drones parcourent les espaces de travail et de vie des humains. En particulier, les robots mobiles sont déployés ou en voie de déploiement dans de nombreux endroits, des aéroports aux restaurants en passant par les rues. Dans un cadre classique de planification du mouvement, tout est obstacle que le robot doit éviter pour atteindre sa destination. Cependant, cette approche ne peut pas être directement utilisée pour la navigation des robots dans les environnements humains. Les humains peuvent ne pas être à l'aise de voir un robot se déplacer très près d'eux ou de ne pas savoir si le robot est prêt à leur céder le passage ou non. Si les humains ne sont pas pris en compte de manière explicite dans la planification de la navigation, le robot peut les perturber et ne pas être accepté. Ainsi, un nouveau domaine de la navigation robotique se concentrant sur ces aspects, appelé `Human-Aware Robot Navigation (HAN) (ou Social Robot Navigation)', se développe rapidement de nos jours. Ce travail explore l'approche HAN dans le cas des robots mobiles et propose quelques nouveaux facteurs et systèmes qui peuvent rendre un robot plus acceptable par les humains. 

% L'idée centrale de ce travail est que le robot doit éviter ou atténuer les interactions homme-robot inconfortables qui se produisent pendant la navigation. Ainsi, dans la première partie de la thèse, nous explorons l'évaluation de situation et la planification proactive dans HAN pour planifier des trajectoires de robot lisibles et acceptables. Nous montrons également comment la planification proactive pourrait être une meilleure alternative à la planification réactive pour l'approche HAN. Nous introduisons également de nouvelles contraintes sociales homme-robot et une nouvelle méthode de prédiction de la trajectoire des humains au voisinage du robot. Le système proposé a été validé dans plusieurs contextes, et une analyse détaillée en est présentée. La partie suivante développe cette idée et propose un système HAN qui peut gérer des humains aussi bien statiques qu'en mouvement dans plusieurs circonstances. Nous proposons un système HAN basé sur la pile de navigation ROS pour résoudre le problème de la navigation multi-contexte. Ce système est hautement ajustable et possède un mécanisme de changement de modalité qui permet au robot d'adapter, en fonction du contexte, plusieurs paramètres d'interaction homme-robot. Nous introduisons des contraintes plus sensibles à l'homme concernant les normes sociales pour rendre la navigation du robot plus vidve egt réactive. Enfin, le système a été testé dans plusieurs scénarios simulés et réels et des analyses en sont fournies. Comparé à un système HAN déjà existant, notre système a donné des résultats meilleurs et plus satisfaisants tant sur le plan qualitatif que quantitatif. Bien que ce système puisse gérer plus d'un type de scénario avec des humains visibles par le robot, il ne peut pas traiter les apparitions soudaines d'humains ou préparer le robot à de telles occurrences. Ainsi, dans une partie suivante de la thèse, une méthodologie pour détecter de telles apparitions potentielles est proposée. Ces estimations sont ensuite intégrées au système HAN proposé précédemment afin de permettre au robot de manœuvrer avec précaution autour des lieux de ces possibles apparitions. L'algorithme proposé a été largement testé, et les avantages de cet ajout sont démontrés par plusieurs expériences. 

% Tout au long du développement de cette thèse, l'évaluation du système HAN a été un défi car il n'existe pas de métriques suffisament bonnes et acceptée par la communauté. Par conséquent, nous avons utilisé certaines métriques existantes et en avons proposé de nouvelles qui pourraient être pertinentes dans de nombreux contextes humains-robots. La dernière partie de cette thèse présente ces nouvelles métriques et leur évaluation dans différents contextes. Enfin, nous concluons cette thèse par une discussion sur l'état actuel du domaine, les défis rencontrés au cours du développement de cette thèse et les perspectives futures. 