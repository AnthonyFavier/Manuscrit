\chapter*{Résumé}

\textbf{TODO: Too long}

La collaboration et la coordination entre l'homme et le robot doivent être transparentes pour que les robots deviennent des éléments importants de notre vie quotidienne. 
Bien que la collaboration homme-robot se soit avérée bénéfique, la plupart des robots actuels travaillent dans un espace de travail physiquement séparé de l'homme, ou leurs capacités sont sévèrement limitées à proximité de l'homme.
L'objectif de la collaboration homme-robot, un domaine en plein essor de la recherche en robotique et en intelligence artificielle, est de permettre aux humains de collaborer efficacement et en toute sécurité avec des robots pour accomplir n'importe quelle tâche industrielle, de service ou domestique.
Ce travail vise à combler le fossé entre les capacités robotiques et les attentes humaines, en favorisant une nouvelle ère de collaboration transparente et intuitive entre les humains et les robots dans des environnements partagés. Plus précisément, ce manuscrit présente une exploration de la prise de décision dans le contexte de la collaboration homme-robot, en particulier dans les sous-domaines de la navigation consciente de l'homme et de la planification des tâches.

Tout d'abord, nous discutons de divers domaines et travaux connexes de la CRH afin de mieux comprendre le contexte de mon travail. 
Après m'être familiarisé avec le planificateur de tâches HATP/EHDA, je présente ma première contribution qui incorpore certains concepts de la théorie de l'esprit dans la planification de tâches HRC. Certains modèles et algorithmes sont proposés et évalués pour mieux estimer et maintenir les croyances humaines afin de mieux anticiper leur comportement. En conséquence, nous pouvons identifier quand l'humain a une fausse croyance à propos d'un fait évalué comme pertinent pour la tâche. Dans ce cas, le robot peut informer l'humain de manière proactive pour corriger la fausse croyance ou le robot peut retarder volontairement son action pour s'assurer que l'humain voit son exécution et déduit le fait correspondant, en évitant la communication verbale. Les résultats montrent que ce schéma permet de maintenir efficacement les croyances humaines dans les scénarios de fausses croyances et permet de résoudre une classe plus large de problèmes que HATP/EHDA tout en ne communiquant pas systématiquement.

Ma deuxième contribution est une nouvelle approche de planification des tâches produisant une politique comportementale du robot assurant une collaboration harmonieuse où l'humain a toujours une latitude de décision totale et où le robot se conforme toujours simultanément aux actions de l'humain.
Cette approche est basée sur un modèle d'action conjointe simultanée et conforme que nous avons conçu. Ce modèle, sous la forme d'un automate, tient compte du facteur d'incontrôlabilité des humains et des signaux sociaux. Nous proposons également une nouvelle méthode d'évaluation et de sélection des plans basée sur l'estimation des préférences internes de l'homme concernant la tâche. Les résultats empiriques montrent que cette approche permet un comportement simultané du robot conforme aux décisions et préférences en ligne de l'homme.

Pour valider cette approche, nous avons mené une étude auprès des utilisateurs à l'aide d'un simulateur de collaboration interactive spécialement développé à cet effet. Les participants ont été invités à collaborer dans plusieurs scénarios de BlocksWorld avec un robot simulé en suivant les politiques produites par notre approche. Pour la comparaison, nous avons utilisé une approche opposée à la nôtre, dans laquelle l'humain est contraint de se conformer aux choix du robot. Nous avons montré par une analyse statistique que notre approche permettait de satisfaire des préférences humaines nettement meilleures. De même, nous avons également montré, par rapport à la base, que notre approche induit une interaction significativement plus positive, une collaboration adaptative et efficace, et des décisions du robot significativement plus adaptatives et accommodantes.

Enfin, ma troisième contribution concerne la prise de décision en matière de navigation. Je propose un système produisant un avatar humain "intelligent" qui, en plus d'être réactif, peut prendre des décisions rationnelles sur les tâches de navigation. Ce système sert d'outil de référence et de test pour les systèmes de navigation robotisés. De cette manière, les systèmes de navigation robotique peuvent être évalués, réglés et soumis à des tests de résistance en simulation, ce qui leur permet de réaliser plus rapidement des expériences matures dans la vie réelle.