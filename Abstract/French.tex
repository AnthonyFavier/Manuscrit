\chapter*{Résumé}

\textbf{TODO: Too long}

Bien qu'il ait été montré que la collaboration humain-robot puisse être bénéfique, la plupart des robots actuels travaillent dans des espaces physiquement séparés de l'humain ou leurs capacités sont sévèrement limitées à proximité de l'humain. Ce travail vise à combler le fossé entre les capacités robotiques et les attentes humaines, en favorisant une nouvelle ère de collaboration transparente et intuitive entre les humains et les robots dans des environnements partagés pour réaliser à la fois des tâches industrielles, de services ou domestiques. Plus précisément, ce manuscrit présente une étude sur la prise de décision dans le contexte de la collaboration humain-robot, en particulier dans les domaines de la navigation et de la planification des tâches.

Tout d'abord, nous discutons de divers domaines et travaux en lien avec la collaboration humain-robot afin de mieux comprendre le contexte de mon travail. Après une familiarisation avec le planificateur de tâches HATP/EHDA, je présente ma première contribution qui incorpore certains concepts de la théorie de l'esprit dans la planification de tâches. Certains modèles et algorithmes sont proposés et évalués pour mieux estimer et maintenir les connaissances de l'humain lors d'une collaboration afin de mieux anticiper son comportement. En conséquence, nous pouvons identifier quand l'humain a une fausse connaissance d'un fait évalué comme pertinent pour la tâche. Dans ce cas, le robot peut informer l'humain de manière proactive pour corriger la fausse information ou le robot peut retarder volontairement ses actions afin qu'elles soient vu par l'humain. Les résultats montrent que ce schéma permet de maintenir efficacement les connaissances de l'humain et permet de résoudre une classe plus large de problèmes que HATP/EHDA tout en ne communiquant pas systématiquement.

Ma deuxième contribution est une nouvelle approche de planification des tâches produisant une politique comportementale du robot assurant une collaboration fluide où l'humain a toujours une latitude de décision totale et où le robot se conforme toujours en parallèle à ces décisions. Cette approche est basée sur un modèle d'action conjointe simultanée et accommodante que nous avons conçu. Ce modèle, sous la forme d'un automate, tient compte de l'incontrôlabilité de l'humain et des signaux sociaux. Nous proposons également une nouvelle méthode d'évaluation et de sélection des plans basée sur l'estimation des préférences internes de l'humain concernant la tâche. Les résultats empiriques montrent que cette approche permet un comportement concourant du robot qui se conforme aux décisions et aux préférences en temps réel de l'humain.

Pour valider l'approche précédente, nous avons mené une étude utilisateur à l'aide d'un simulateur spécialement développé à cet effet. Les participants ont été invités à collaborer dans plusieurs scénarios avec un robot simulé suivant les politiques produites par notre approche. Nous avons utilisé comme référence une approche opposée à la nôtre dans laquelle l'humain est forcé de se conformer aux choix du robot. Nous avons montré par une analyse statistique que notre approche permettait de satisfaire les préférences des humains de manière nettement plus satisfaisante. De même, nous avons montré que notre approche induit une interaction significativement plus positive, une collaboration plus adaptative et efficace, et des décisions du robot significativement plus adéquates et accommodantes.

Enfin, ma troisième contribution concerne la prise de décision dans le domaine de la navigation. Je propose un système simulant un avatar humain qui, en plus d'être réactif, prend des décisions rationnelles sur les tâches de navigation. Ce système sert d'outil de test et d'évaluation pour les systèmes de navigation robotique. Ainsi, ces derniers peuvent être évalués, ajustés et robustifiés en simulation pour réaliser plus rapidement des expériences matures dans la vie réelle.