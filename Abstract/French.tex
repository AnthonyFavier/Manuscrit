\chapter*{Résumé}

% 4000 char

Bien que la collaboration humain-robot puisse être bénéfique, la plupart des robots actuels travaillent dans des espaces physiquement séparés de l'humain ou alors leurs capacités sont drastiquement limitées à proximité d'un humain. Ce travail vise à combler le fossé entre les capacités robotiques et les attentes humaines, en favorisant une nouvelle ère de collaboration transparente et intuitive entre les humains et les robots dans des environnements partagés pour réaliser à la fois des tâches industrielles, de services ou domestiques. Plus précisément, ce manuscrit présente une étude sur la prise de décision dans le contexte de la collaboration humain-robot, en particulier dans les domaines et de la planification des tâches et de la simulation d'agents intelligents.

D'abord, nous discutons de divers travaux en lien avec la Collaboration Humain-Robot afin de mieux comprendre le contexte de mon travail. Après une familiarisation avec le planificateur de tâches HATP/EHDA, je présente ma première contribution qui incorpore certains concepts de la Théorie De l'Esprit dans la planification de tâches. Certains modèles et algorithmes sont proposés et évalués pour mieux estimer et anticiper les connaissances de l'humain et son comportement. Ainsi, nous pouvons identifier les potentiellement néfastes fausses croyances de l'humain and ainsi l'informer proactivement pour corriger les fausses informations, ou volontairement retarder les actions du robot pour qu'elles soient vu par l'humain. Les résultats montrent que ce schéma permet de maintenir efficacement les connaissances de l'humain et permet de résoudre une classe de problèmes plus large que HATP/EHDA tout en ne communiquant pas systématiquement.

Ma deuxième contribution est une nouvelle approche de planification des tâches produisant une politique comportementale du robot assurant une collaboration fluide où l'humain a toujours une latitude de décision totale et où le robot se conforme toujours en parallèle à ces décisions. Cette approche est basée sur un modèle d'action conjointe simultanée et accommodante que nous avons conçu. Ce modèle, sous la forme d'un automate, tient compte de l'incontrôlabilité de l'humain et des signaux sociaux. Nous proposons également une nouvelle méthode d'évaluation et de sélection des plans basée sur l'estimation des préférences internes de l'humain concernant la tâche. Les résultats empiriques montrent que cette approche permet un comportement concourant du robot qui se conforme aux décisions et aux préférences en temps réel de l'humain.

Dans une autre contribution validant l'approche précédente, nous avons implémenté notre modèle d'action conjointe en tant que schéma d'exécution dans un simulateur. Nous avons ainsi mené une étude utilisateur où les participants ont collaboré dans plusieurs scénarios avec un robot simulé suivant les politiques produites par notre approche. En opposition avec notre approche, notre avons utilisé comme référence un robot imposant continuellement ses décisions à l'humain. Nous avons montré par une analyse statistique que notre approche permettait de nettement mieux satisfaire les préférences des humains. De plus, les differences les plus significatives sont que les participants ont perçu une interaction plus positive, une collaboration plus adaptative et efficace, et des décisions du robot plus adéquates et accommodantes.   

Enfin, mes dernières contributions concernent la simulation d'agents humains intelligents. Ses agents simulés dotés de processus de prise de décision peuvent aider à tester, évaluer et robustifier des systèmes robotiques intéractifs et collaboratifs. Nous proposons une architecture générique visant à simuler d'un tel agent intelligent et nous présentons une version implémentée pour le cas de la navigation. Nous présentons aussi une contribution capable de simuler plusieurs agents sociaux mobiles.