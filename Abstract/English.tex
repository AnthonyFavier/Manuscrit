\chapter*{Abstract}

% 4000 caractères

Although human-robot collaboration can be beneficial, most of today's robots work in spaces physically separated from humans, or their capabilities are severely limited in close proximity to humans. This work aims to bridge the gap between robotic capabilities and human expectations, fostering a new era of seamless and intuitive collaboration between humans and robots in shared environments to perform industrial, service or domestic tasks. More specifically, this manuscript presents a study of decision-making in the context of human-robot collaboration, particularly in the areas of navigation and task planning.

First, we discuss various fields and works related to human-robot collaboration in order to better understand the context of my work. After an introduction to the HATP/EHDA task planner, I present my first contribution, which incorporates some concepts from the Theory Of Mind into task planning. Some models and algorithms are proposed and evaluated to better estimate and maintain human knowledge during collaboration, in order to better anticipate human behavior. As a result, we can identify when the human has a false belief about a fact evaluated as relevant to the task. In this case, the robot can proactively inform the human to correct the false information, or the robot can deliberately delay its actions so that they can be seen by the human. The results show that this scheme effectively maintains the human's beliefs and solves a wider class of problems than HATP/EHDA, while not systematically communicating.

My second contribution is a new approach to task planning producing a robot behavioral policy ensuring smooth collaboration where the human always has full decision latitude and the robot always conforms in parallel to these decisions. This approach is based on a model of concurrent and compliant joint action that we have designed. This model, in the form of an automaton, takes into account human incontrollability and social cues. We also propose a new method of plan evaluation and selection based on the estimation of the human's internal preferences regarding the task. Empirical results show that this approach enables concurrent robot behavior that conforms to the human's real-time decisions and preferences.

To validate the above approach, we conducted a user study using a specially developed simulator. Participants were invited to collaborate in several scenarios with a simulated robot following the policies produced by our approach. We used as a reference an approach opposite to ours, in which the human is forced to conform to the robot's choices. We showed through statistical analysis that our approach satisfies human preferences significantly more successfully. Similarly, we have shown that our approach induces significantly more positive interaction, more adaptive and effective collaboration, and significantly more appropriate and accommodating robot decisions.

Finally, my third contribution concerns decision-making in navigation. I propose a system simulating a human avatar which, in addition to being reactive, makes rational decisions about navigation tasks. This system serves as a test and evaluation tool for robotic navigation systems. In this way, they can be evaluated, adjusted, and robustified in simulation, so that mature real-life experiments can be carried out more quickly. An additional work capable of simulating several avatars is also presented.