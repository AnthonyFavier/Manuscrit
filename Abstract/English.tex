\chapter*{Abstract}
% \addstarredchapter{Abstract}

The navigation of a mobile robot in human environments is referred to as \acrfull{han} or sometimes as Social (Robot) Navigation. We present the evolution of HAN over the years and the challenges associated with it before we proceed with the explanation of our approach. With a growing interest in the field, new frameworks are required to address this problem better, and as a contribution, we propose a proactive multi-context HAN planning system. 

The proposed proactive planning framework for HAN is based on the \acrfull{hateb} approach. This framework adds a decision-making loop at the trajectory planning level and switches between different planning modes based on the assessed situation. Using this framework, we propose a \acrshort{ros} based HAN Stack called \acrfull{cohan} planner. CoHAN is designed to address multi-context robot navigation among static and dynamic humans with updated costmaps and new human-robot social constraints. These optimization constraints aim to make the robot’s motion legible and avoid surprise appearances from behind. CoHAN offers four different path prediction mechanisms for humans and three different planning modes, along with a new recovery mode. The qualitative analysis performed on simulated human-robot scenarios highlights the advantages of proactive and situation based planning. They also show how the newly added social constraints improve the robot’s legibility. Quantitative comparisons with other HAN planners show that CoHAN can solve more intricate scenarios in an acceptable manner.

In the next part, CoHAN is extended further to proactively address sudden human appearance from occluded regions on the map. We propose the concept of ‘invisible humans’ to estimate the locations of such human emergences. These estimations are added to CoHAN via another optimization constraint, and a new mode of planning to pass through doors and narrow passages cautiously is introduced. The experiments in simulation and on the real robot demonstrate the benefits of ‘invisible humans’ in HAN and show that the robot maintains a greater distance from occluded humans compared to other approaches.

The current metrics based on proxemic zone violations cannot do justice in evaluating intricate scenarios. Therefore, we propose a set of metrics taking into consideration the velocities and the visibility of the human, which may be pertinent to various human-robot navigation contexts. This part of the thesis presents the mathematical formulation of these metrics, followed by the evaluation of four different human-robot navigation contexts using them. These metrics, combined with the velocity profiles and paths, clearly distinguish a HAN planner from a simple navigation planner. We conclude this thesis with a discussion on the contribution followed by the challenges and the future perspectives of HAN.



%%%%%%%%%%%%%%%%%%%%%% First Version %%%%%%%%%%%%%%%%%%%%%%%%%%%
% Today, more than ever, mobile robots and drones are roaming human workspaces. In particular, mobile robots are being deployed in many places, from airports to restaurants to streets. In a classical motion planning setting, everything is an obstacle, and the robot has to avoid all the obstacles and reach its destination. However, this cannot be directly employed in robot navigation planning in human environments. Humans may not be comfortable seeing a robot move very close to them or not knowing if the robot is ready to give them the way or not. Unless the humans are considered in navigation planning, the robot can confuse the humans and may not be accepted to be around them. Hence a new field of robot navigation concentrating on these aspects, called ‘Human-Aware Robot Navigation (HAN) (or Social Robot Navigation)’, is evolving at a rapid pace these days. This work explores HAN in the case of mobile robots and proposes some new factors and systems that can make a robot more ‘human aware’. 

% The core idea behind this work is that the robot has to avoid or mitigate uncomfortable human-robot interactions that occur during navigation. So, we explore situation assessment and proactive planning in HAN to plan legible and acceptable trajectories for the robot in the first part of the thesis. We also introduce some new human-robot social constraints and a new human path prediction methodology. The proposed system has been validated under several settings, and a detailed analysis is presented. The next part elaborates on this idea and moves on to propose a HAN system that can handle static and dynamic humans under several circumstances. We propose a HAN system based on the ROS navigation stack to address the problem of multi-context navigation. This system is highly tunable and has a modality switching mechanism that allows the robot to mitigate several human-robot interaction settings. We introduce some more human-aware constraints pertaining to social norms to make the robot’s navigation vivid. Finally, it has been tested in several simulated and real-world scenarios and analyses are provided. When compared with an already existing HAN system, our system yielded better and more satisfactory results both qualitatively and quantitatively. Even though this system can handle more than one kind of scenario with visible humans, it cannot address sudden human appearances or prepare the robot ready for such occurrences. So in the next part of the thesis, a methodology to detect such possible appearances is proposed. These estimations are then integrated with the previous proposed HAN system to allow the robot to manoeuvre around the places of such possible emergences cautiously. The proposed algorithm has been extensively tested, and the advantages of this addition are shown through several experiments. 

% Throughout the development of this thesis, the evaluation of the HAN system has been a challenge as there are no good enough and well-accepted metrics currently. Therefore, we have used some existing ones and proposed some new metrics that could be pertinent to many human-robot contexts. The last part of this thesis presents these proposed metrics and their evaluations in different settings. Finally, we conclude this thesis with a discussion on the current state of the field, the challenges faced during the development of this thesis and the future perspectives. 





%%%%%%% Old and Bigger %%%%%%%%%%%%%%

% Today, more than ever, mobile robots and drones are roaming human workspaces and environments. In Particular, mobile robots are being deployed in many places, from airports to restaurants to streets. Planning the motion of the robot in such places poses more issues than what is considered by the classical robot motion planning algorithms. In a classical setting, everything is an obstacle, and the robot has to avoid all the obstacles and reach its destination. However, this cannot be directly employed in robot motion planning, specifically, robot navigation in human environments. Humans may not be comfortable seeing a robot move very close to them or not knowing if the robot is ready to give them the way or not. Unless the humans are considered in navigation planning, the robot can confuse the humans and may not be accepted to be around humans. Hence a new field of robot navigation concentrating on these aspects, called Human-Aware Robot Navigation (HAN), has been evolving at a rapid pace these days. It is also referred to as `Social Robot Navigation' sometimes. This work explores HAN in the case of mobile robots and proposes some new factors and systems that can make a robot more `human aware'. 

% The core idea behind this work is that the robot has to avoid or mitigate uncomfortable human-robot interactions that occur during the navigation. So, we explore situation assessment and proactive planning in HAN to plan legible and acceptable trajectories for the robot in the first part of the thesis. It was also shown how proactive planning could be a better alternative to reactive planning for HAN. We also introduce some new human-robot social constraints (or social norms) as well as a new human path prediction methodology. The proposed system has been validated under several settings, and a detailed analysis is presented at the end of this part. The next part elaborates on this idea and moves on to propose a HAN system that can handle static and dynamic humans under several circumstances. We propose an open-source human-aware navigation system based on the ROS navigation stack to address the problem of multi-context navigation. This system is highly tunable and has modality switching mechanisms that allow the robot to mitigate several human-robot interaction settings. We also introduce some more human-aware constraints pertaining to social norms to make the robot’s navigation vivid. Finally, this system has been tested in several simulated and real-world scenarios and analyses are provided. When compared with an already existing HAN system, our system yielded better and more satisfactory results both qualitatively and quantitatively. Even though the proposed system can handle more than one kind of scenario with visible humans, it cannot address the sudden human appearances or prepare the robot ready for such occurrences. So, in the next part of the thesis, a methodology to detect such possible appearances is proposed. These estimations are then integrated with the previous proposed HAN system to allow the robot to manoeuvre around the places of such possible emergences cautiously. The proposed algorithm has been extensively tested, and the advantages of this addition are shown through simulated and real-world experiments. 

% Throughout the development of this thesis, the evaluation of the HAN system quantitatively has been a challenge as there are no good metrics currently. Therefore, we have used some existing ones and proposed some new metrics that could be pertinent to many human-robot contexts. The last part of this thesis presents these new metrics and their evaluations in different settings. Finally, we conclude this thesis with a discussion on the current state of the field, the challenges faced during the development of this thesis and the future perspectives. 
